\documentclass[a4paper,10pt]{article}
\usepackage[utf8]{inputenc}

\usepackage{hyperref}

\usepackage{amsmath,amsthm}
\usepackage{ dsfont }
\usepackage{lipsum}
\usepackage{hyperref}

\usepackage{tikz}
\usepackage{verbatim}
%\usepackage{simpler-wick}
%\usepackage[lf]{MinionPro}
\usepackage[scr=rsfso,calscaled=.96]{mathalfa}


\newcommand{\One}{\hat{\mathbf{1}}} \newcommand{\eff}{\text{eff}}
\newcommand{\Heff}{\hat{H}_\text{eff}}
\newcommand{\Veff}{\hat{V}_\text{eff}}
\newcommand{\braket}[1]{\langle#1\rangle}
\newcommand{\Span}{\operatorname{sp}}
\newcommand{\tr}{\operatorname{trace}}
\newcommand{\diag}{\operatorname{diag}}
\newcommand{\bra}[1]{\left\langle #1 \right|}
\newcommand{\ket}[1]{\left| #1 \right\rangle} \newcommand{\element}[3]
{\bra{#1}#2\ket{#3}}

\usepackage{amsmath}
\theoremstyle{definition}
\newtheorem{definition}{Definition}
\newtheorem{theorem}{Theorem}
\newtheorem{fact}{Fact}
\newtheorem{observation}{Observation}
\newtheorem{example}{Example}

% Title Page
\title{Project 2 Fyss4480, The CCD approximation}
\author{}


\begin{document}
\maketitle

\subsection*{Project 2 introduction}
In this project we will study the Coupled Cluster method. In particular we will explore the Coupled Cluster Doubles (CCD)
approximation. The first part of the project will be to derive the relevant energy and amplitude equations. The second part 
will be to write a CCD program and apply it to the Helium and Beryllium atom as in Project1 and compare the resulting 
energies with results from configuration interaction, Hartree-Fock and many-body perturbation theory.

In the Coupled Cluster method the ansatz for the ground state is given by 
\begin{equation}
 \ket{\Psi_\text{CC}} = e^{\hat{T}}\ket{c},
\end{equation}
where $\hat{T}$ is the cluster operator
\begin{align*}
 \hat{T} &= \hat{T}_1 + \hat{T}_2 + \dots \\
 \hat{T}_n &= \left( \frac{1}{n!} \right)^2 \sum_{abc \dots} \sum_{ijk \dots} t^{abc\dots}_{ijk\dots}a_a^\dagger a_b^\dagger a_c^\dagger \dots a_k a_j a_i. 
\end{align*}
The energy is given by 
\begin{equation}
 E_\text{CC} =  \braket{c|\bar{H}|c}
\end{equation}
where $\bar{H}$ is the \textit{similarity transformed} Hamiltonian
\begin{equation}
 \bar{H} = e^{-\hat{T}}\hat{H}e^{\hat{T}}.
\end{equation}

The amplitudes $t^{abc\dots}_{ijk\dots}$ are given by the solution of the non-linear amplitude equations 
\begin{equation}
 \braket{\Phi^{abc \dots}_{ijk \dots} | \bar{H} | c} = 0.
\end{equation}
In order to setup the above equations, the similarity transformed Hamiltonian is expanded using the Baker-Campbell-Hausdorff (BCH)
expansion, 
\begin{align}
 \bar{H} = \hat{H} &+ [\hat{H},\hat{T}] \nonumber \\
 &+ \frac{1}{2} [[\hat{H},\hat{T}],\hat{T}] \nonumber \\
 &+ \frac{1}{6} [[[\hat{H},\hat{T}],\hat{T}],\hat{T}] \nonumber \\
 &+ \frac{1}{24} [[[[\hat{H},\hat{T}],\hat{T}],\hat{T}],\hat{T}] \nonumber \\
 &+ \dots.
\end{align}
When $\hat{H}$ is a two-body Hamiltonian this series truncates after the four nested commutators listed above.

In this project we will consider the so-called Coupled Cluster Doubles approximation where we have 
\begin{equation}
\hat{T} = \hat{T}_2 = \frac{1}{4} \sum_{ab}\sum_{ij} t^{ab}_{ij} a_a^\dagger a_b^\dagger a_j a_i 
\end{equation}

\paragraph{Part a) The CCD energy}
Let the Hamiltonian be given on normal ordered form as discussed in the lectures 
\begin{align}
 \hat{H} &= E_\text{ref} + \hat{F} + \hat{H}_I \\
 &= E_\text{ref} + \sum_{pq} \braket{p|\hat{f}|q}\{a_p^\dagger a_q\} + \frac{1}{4}\sum_{pqrs} \braket{pq|\hat{v}|rs}_\text{AS} \{ a_p^\dagger a_q^\dagger a_s a_r \}.
\end{align}
The subscript AS will be dropped in the remainder of the text, but the matrix elements are still antisymmetrized.

Using Wicks generalized theorem, show that 
\begin{align}
 \braket{c|\hat{H}|c} &= E_\text{ref} \\
 \braket{c|[\hat{H},\hat{T}_2]|c} &= \frac{1}{4} \sum_{abij} \braket{ij|\hat{v}|ab} t^{ab}_{ij}
\end{align}
and argue that 
\begin{equation}
 \braket{c|[[\hat{H},\hat{T}_2],\hat{T}_2]|c} = 0.
\end{equation}
Thus, we have that the CCD energy is given by 
\begin{equation}
 E_\text{CCD} = E_\text{ref} + \frac{1}{4} \sum_{abij} \braket{ij|\hat{v}|ab} t^{ab}_{ij} \label{CCDenergy}
\end{equation}

\paragraph{Part b) Deriving the CCD amplitde equation}
The next step is to obtain an algbraic expressions for the amplitude equation 
\begin{equation}
 \braket{\Phi^{ab}_{ij}|\bar{H}|c} = 0 \label{amplitudeEquation}.
\end{equation}

We will break this up into several steps. First, show that 
\begin{align}
 \braket{\Phi^{ab}_{ij}|\hat{F}|c} &= 0 \\
 \braket{\Phi^{ab}_{ij}|\hat{H}_I|c} &= \braket{ab|\hat{v}|ij} 
\end{align}

Secondly, show that 
\begin{align}
 \braket{\Phi^{ab}_{ij}|[\hat{F},\hat{T}_2]|c} &= \hat{P} (ij) \sum_ k  f^{k}_{i} t^{ab}_{jk}  - \hat{P} (ab) \sum_c f^{a}_{c} t^{bc}_{ij} \\
 \braket{\Phi^{ab}_{ij}|[\hat{H}_I,\hat{T}_2]|c} &= \frac{1}{2} \sum_{kl} \braket{kl|\hat{v}|ij} t^{ab}_{kl} + \hat{P}(ab) \hat{P}(ij) \sum_{kc} \braket{ak|\hat{v}|ic} t^{bc}_{jk}  \nonumber \\
 &+ \frac{1}{2} \sum_{cd} \braket{ab|\hat{v}|cd} t^{cd}_{ij}
\end{align}
Here we have defined 
\begin{align}
 \hat{P}(ab) &= \left( 1 - \hat{P}_{ab} \right) \\
 \hat{P}(ab) \hat{P}(ij) &= \left( 1 - \hat{P}_{ab} \right)\left( 1 - \hat{P}_{ij} \right)
\end{align}
where $\hat{P}_{ab}$ permutes the indices $a$ and $b$ (or $i$ and $j$). 

Thirdly, argue that 
\begin{equation}
 \braket{\Phi^{ab}_{ij}|[[\hat{F},\hat{T}_2],\hat{T}_2]|c} = 0
\end{equation}
and show that 
\begin{align}
 \frac{1}{2} \braket{\Phi^{ab}_{ij}|[[\hat{H}_I,\hat{T}_2],\hat{T}_2]|c} =  &\frac{1}{4} \sum_{klcd} \braket{kl|\hat{v}|cd} t^{ab}_{kl} t^{cd}_{ij} -  \frac{1}{2}\hat{P} (ij) \sum_{klcd} \braket{kl|\hat{v}|cd} t^{ab}_{il} t^{cd}_{kj} \nonumber \\
 - & \hat{P} (ab) \sum_{klcd} \braket{kl|\hat{v}|cd} t^{ac}_{ik} t^{bd}_{lj}  - \frac{1}{2} \hat{P}(ab) \sum_{klcd} \braket{kl|\hat{v}|cd} t^{ac}_{ij}t^{bd}_{kl} .
\end{align}

Finally, argue that 
\begin{equation}
 \braket{\Phi^{ab}_{ij}|[[[\hat{H},\hat{T}_2],\hat{T}_2],\hat{T}_2]|c} = 0.
\end{equation}

Collecting all terms we have that 
\begin{align}
 \braket{\Phi^{ab}_{ij}|\bar{H}|c} = &\braket{ab|\hat{v}|ij} + \hat{P} (ij) \sum_ k  f^{k}_{i} t^{ab}_{jk}  - \hat{P} (ab) \sum_c f^{a}_{c} t^{bc}_{ij} \nonumber \\
 &+\frac{1}{2} \sum_{kl} \braket{kl|\hat{v}|ij} t^{ab}_{kl} + \hat{P}(ab) \hat{P}(ij) \sum_{kc} \braket{ak|\hat{v}|ic} t^{bc}_{jk}  \nonumber \\
 &+ \frac{1}{2} \sum_{cd} \braket{ab|\hat{v}|cd} t^{cd}_{ij} + \frac{1}{4} \sum_{klcd} \braket{kl|\hat{v}|cd} t^{ab}_{kl} t^{cd}_{ij} -  \frac{1}{2}\hat{P} (ij) \sum_{klcd} \braket{kl|\hat{v}|cd} t^{ab}_{il} t^{cd}_{kj} \nonumber \\
 - & \hat{P} (ab) \sum_{klcd} \braket{kl|\hat{v}|cd} t^{ac}_{ik} t^{bd}_{lj}  - \frac{1}{2} \hat{P}(ab) \sum_{klcd} \braket{kl|\hat{v}|cd} t^{ac}_{ij}t^{bd}_{kl} \label{amplitudeExpression}.
\end{align}

\paragraph{Part c) Setting up an iterative scheme}
We want to solve Eq. \ref{amplitudeEquation} with the left-hand side given by Eq. \ref{amplitudeExpression} 
for the unknown amplitudes $t^{ab}_{ij}$. 

Define $\epsilon_p = f^p_p$ and $D^{ij}_{ab} = \left( \epsilon_i+\epsilon_j-\epsilon_a-\epsilon_b \right) $ 
and show that 
\begin{equation}
 t^{ab}_{ij}D^{ij}_{ab} = \braket{ab|\hat{v}|ij} + \hat{P} (ij) \sum_ {k\neq i}  f^{k}_{i} t^{ab}_{jk}  - \hat{P} (ab) \sum_ {c\neq a} f^{a}_{c} t^{bc}_{ij} + \left( \dots \right) \label{amplitudeExplicit}\\
\end{equation}
where $\left( \dots \right)$ refers to the rest of Eq. \ref{amplitudeExpression}. Furtheremore, argue that 
\begin{equation}
 \hat{P} (ij) \sum_ {k\neq i}  f^{k}_{i} t^{ab}_{jk}  - \hat{P} (ab) \sum_ {c\neq a} f^{a}_{c} t^{bc}_{ij} = 0
\end{equation}
if the single-particle functions are given by the Hartree-Fock solution.

Equation \ref{amplitudeExplicit} is a non-linear equation in the amplitudes and must be solved iteratively. 
The way we have rewritten the equation suggests that we define an iterative scheme as 
\begin{equation}
 (t^{ab}_{ij})^{(k+1)}  = \frac{g\left((t^{ab}_{ij})^{(k)}\right)}{D^{ij}_{ab}} \ \ k = 0,1,2,\dots \label{IterativeScheme}
\end{equation}
for some initial guess $(t^{ab}_{ij})^{(0)}$. Here we have defined
\begin{align}
 g\left((t^{ab}_{ij})^{(k)}\right) = &\braket{ab|\hat{v}|ij} + \hat{P} (ij) \sum_ {k\neq i}  f^{k}_{i} (t^{ab}_{jk})^{(k)}  - \hat{P} (ab) \sum_ {c\neq a} f^{a}_{c} (t^{bc}_{ij})^{(k)} \nonumber \\
 &+\frac{1}{2} \sum_{kl} \braket{kl|\hat{v}|ij} (t^{ab}_{kl})^{(k)} + \hat{P}(ab) \hat{P}(ij) \sum_{kc} \braket{ak|\hat{v}|ic} (t^{bc}_{jk})^{(k)}  \nonumber \\
 &+ \frac{1}{2} \sum_{cd} \braket{ab|\hat{v}|cd} (t^{cd}_{ij})^{(k)} + \frac{1}{4} \sum_{klcd} \braket{kl|\hat{v}|cd} (t^{ab}_{kl}))^{(k)} (t^{cd}_{ij}))^{(k)} \nonumber \\
 &-  \frac{1}{2}\hat{P} (ij) \sum_{klcd} \braket{kl|\hat{v}|cd} (t^{ab}_{il}))^{(k)} (t^{cd}_{kj}))^{(k)} - \hat{P} (ab) \sum_{klcd} \braket{kl|\hat{v}|cd} (t^{ac}_{ik}))^{(k)} (t^{bd}_{lj})^{(k)} \nonumber \\
 &- \frac{1}{2} \hat{P}(ab) \sum_{klcd} \braket{kl|\hat{v}|cd} (t^{ac}_{ij})^{(k)} (t^{bd}_{kl})^{(k)} \label{gFunction}.
\end{align}

\paragraph{Part d) The initial guess}
Iterative solvers need a guess for the amplitudes. A good starting point is to use
the correlated wave operator from perturbation theory to first order in the interaction.
This means that we define the zeroth approximation to the amplitudes as
\begin{equation}
 (t^{ab}_{ij})^{(0)} = \frac{\braket{ab|\hat{v}|ij}}{D^{ij}_{ab}} \label{MBPTinit}. 
\end{equation}
Insert this expression into the CCD energy \ref{CCDenergy} and compare with the energy expression from second-order 
perturbation theory.

\paragraph{Part e) Writing a program that solves the amplitude equations}
The final part of the program is write a program to solve the amplitude equations. That is, write 
a program that computes the amplitudes according to the scheme \ref{IterativeScheme} with the initial 
guess given by Eq. \ref{MBPTinit}. 

One particular nice way of writing the program in Python is to use numpy's einsum 
function \href{https://docs.scipy.org/doc/numpy-1.15.1/reference/generated/numpy.einsum.html}{np.einsum} to evaulate 
the contributions to Eq. \ref{gFunction}. This is by no means mandatory and you can equally well write the program 
as you want.

Compute the CCD energy for Helium and Beryllium using the same basis functions as in Project1 and compare the resulting 
energies with energies obtained with the Hartree-Fock and configuration interaction method. You can use the code template 
found at the github page \href{https://github.com/ManyBodyPhysics/FYS4480/tree/master/doc/Projects/2018/Project1/code_template}{code template} as a starting point for 
your program. 

\paragraph{Part f) Computing in Hartree-Fock basis}
Finally, using the program you wrote in Project1 you can do a basis transformation to the Hartree-Fock basis given 
by 
\begin{align*}
 \braket{p|\hat{f}|q}   &= \epsilon_p \delta_{pq} \\
 \braket{pq|\hat{v}|rs} &= \sum_{\alpha \beta \gamma \delta} C^*_{\alpha p} C^*_{\beta q} C_{\gamma r} C_{\delta s} \braket{\alpha \beta | \hat{v} | \gamma \delta }.
\end{align*}
Perform this transformation and repeat the calcultations in part e) and compare the resulting energies with those you already 
have computed.




\end{document}          
