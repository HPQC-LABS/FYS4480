\documentclass[a4paper,10pt]{article}
\usepackage[utf8]{inputenc}

\usepackage{hyperref}

\usepackage{amsmath,amsthm}
\usepackage{ dsfont }

\usepackage{tikz}
\usepackage{verbatim}
\usepackage{simpler-wick}
%\usepackage[lf]{MinionPro}
\usepackage[scr=rsfso,calscaled=.96]{mathalfa}


\newcommand{\One}{\hat{\mathbf{1}}} \newcommand{\eff}{\text{eff}}
\newcommand{\Heff}{\hat{H}_\text{eff}}
\newcommand{\Veff}{\hat{V}_\text{eff}}
\newcommand{\braket}[1]{\langle#1\rangle}
\newcommand{\Span}{\operatorname{sp}}
\newcommand{\tr}{\operatorname{trace}}
\newcommand{\diag}{\operatorname{diag}}
\newcommand{\bra}[1]{\left\langle #1 \right|}
\newcommand{\ket}[1]{\left| #1 \right\rangle} \newcommand{\element}[3]
{\bra{#1}#2\ket{#3}}

\usepackage{amsmath}
\theoremstyle{definition}
\newtheorem{definition}{Definition}
\newtheorem{theorem}{Theorem}
\newtheorem{fact}{Fact}
\newtheorem{observation}{Observation}
\newtheorem{example}{Example}
\newtheorem{warning}{Warning}
% Title Page
\title{Summary of lectures FYS4480 autumn 2018}
%\author{Morten Hjorth-Jensen and Håkon Kristiansen}

\begin{document}
\maketitle

\section*{Disclaimer}
The purpose of this note is to summarize what has been covered so far in the course and provide a general overview over 
important results and methods. Statements are given without proof and one should consult the lecture notes for more in depth 
discussion of the various topics.

\section{Introduction}

\subsection{Particle statistics}
The fact that elementary particles are identical, or indistinguishable is expressed by 
demanding that the probability density $|\Psi(x_1,x_2,\dots,x_N)|^2$ must be permutation invariant. That is, 
if we exchange any coordinates we must get the same probability
\begin{equation}
 |\Psi(x_1,x_2,\dots,x_N)|^2 = |\Psi(x_{\sigma(1)},x_{\sigma(2)},\dots,x_{\sigma(N)})|^2 .
\end{equation}
Here $\sigma$ is any permutation of coordinate indices. 

To make this somewhat precise we write down the definition
of permutations and define the so-called permutation operator.

\begin{definition}[Permutation]
 Let $\mathcal{I}$ be the finite set $\mathcal{I} = \{ 1,2,3,\cdots,N \}$. A 
 permutation is a function, $\sigma: \mathcal{I} \rightarrow \mathcal{I}$, that takes every number in 
 $\mathcal{I}$ and maps it to another unique number in $\mathcal{I}$. 
 Informally, a permutation is any reordering of $N$ indices.
 
 The collection of all perumations of $\mathcal{I}$ is denoted $S_N$ and is called the symmetric group on $N$ 
 letters. In particular $S_N$ has $N!$ elements, i.e there are $N!$ possible reorderings of $N$ indices. We write 
 $\sigma \in S_N$ for an arbitrary permutation.
\end{definition}

\begin{definition}[Transposition]
 A transposition is a permutation $\sigma \in S_N$, where only a single pair of indices $(i,j)$ in 
 $\{ 1, 2,\cdots, i, \dots, j,\dots, N \}$ are exchanged. 
 That is to say, $\sigma(i) = j$ and $\sigma(j) = i$, while $\sigma(k) = k$ for all $k \neq i,j$.
\end{definition}

\begin{theorem}
 Any permutation $\sigma \in S_N$ can be obtained by a series of transpositions. The sign of the permutation is 
 given by $(-1)^{|\sigma|}$, where $|\sigma|$ is the number of transpositions needed to obtain the permutation. 
\end{theorem}

\begin{example}[A permutation]
 Let $\mathcal{I} = \{1,2,3 \}$. Then $\sigma(1) = 3, \sigma(2) = 1, \sigma(3) = 2$ is 
 a permutation. In other words, we have a reordering of three indices given by
 $\{ \sigma(1), \sigma(2), \sigma(3) \} = \{ 3,1,2 \}$.
 
 Note that in terms of tranpositions we could perform the reordering as follows 
 \begin{align*}
  \{ 1,2,3 \} \underbrace{\rightarrow}_\text{transpose $3$ and $2$} \{ 1,3,2 \} \underbrace{\rightarrow}_\text{transpose $1$ and $3$} \{ 3,1,2 \}.
 \end{align*}
 Here we needed two transpositions so $|\sigma| = 2$. 
\end{example}

\begin{definition}[The permutation operator]
 The permutation operator $\hat{P}_\sigma$ is defined as the operator that evaluates a wavefunction 
 at permuted coordinates
 \begin{equation}
  \hat{P}_\sigma \Psi(x_1,\cdots,x_N) = \Psi(x_{\sigma(1)},\cdots,x_{\sigma(N)}) \label{PermutationOperator}.
 \end{equation}
\end{definition}

Now, the condition
\begin{equation*}
 |\Psi(x_1,x_2,\dots,x_N)|^2 = |\Psi(x_{\sigma(1)},x_{\sigma(2)},\dots,x_{\sigma(N)})|^2 
\end{equation*}
is equivalent to
\begin{equation}
 \Psi(x_1,x_2,\dots,x_N) = e^{i \alpha(\sigma)}\Psi(x_{\sigma(1)},x_{\sigma(2)},\dots,x_{\sigma(N)})
\end{equation}
which again is equivalent to saying that $\Psi(x_1,x_2,\cdots,x_N)$ must be an eigenfunction of 
the permutation operator \ref{PermutationOperator} 
\begin{equation}
 \hat{P}_\sigma \Psi(x_1,x_2,\cdots,x_N) = e^{i \alpha(\sigma)} \Psi(x_{\sigma(1)},x_{\sigma(2)},\dots,x_{\sigma(N)}).
\end{equation}
One can show that either $\hat{P}_\sigma \Psi = \Psi$ for every $\sigma \in S_N$, or $\hat{P}_\sigma \Psi = (-1)^{|\sigma|} \Psi$
for every $\sigma \in S_N$ and $|\sigma|$ is the number of transpositions. 
In the former case, $\Psi$ is "totally symmetric with respect to permutations", and in the latter
case, "totally anti-symmetric". Bosons have totally symmetric wavefunctions only, and fermions have totally
anti-symmetric wavefunctions only.

In this course, we will restrict our attention to fermions, i.e we must look for 
anti-symmetric wavefunctions. 

\subsection{Slater Determinants}
\begin{definition}[Slater determinant]
 Let $\psi_{\alpha_1},\psi_{\alpha_2},\cdots,\psi_{\alpha_N}$ be arbitrary single-particle functions 
 (not necessarily orthonormal). The Slater determinant defined by these functions is denoted 
 $\Phi(x_1,\cdots,x_N; \alpha_1,\cdots,\alpha_N)$, and is defined by the formula
 \begin{equation*}
  \Phi(x_1,\cdots,x_N; \alpha_1,\cdots,\alpha_N) = \frac{1}{\sqrt{N!}}
  \begin{vmatrix}
  \psi_{\alpha_1}(x_1) & \psi_{\alpha_1}(x_2) & \dots & \dots & \psi_{\alpha_1}(x_N) \\ 
  \psi_{\alpha_2}(x_1) & \psi_{\alpha_2}(x_2) & \dots & \dots & \psi_{\alpha_2}(x_N) \\ 
  \dots & \dots & \dots & \dots & \dots \\
  \psi_{\alpha_N}(x_1) & \psi_{\alpha_N}(x_2) & \dots & \dots & \psi_{\alpha_N}(x_N) \\ 
  \end{vmatrix}
 \end{equation*}
 where $x_i$ stand for the coordinates and spin values of a particle $i$ and $\alpha_1,\alpha_2,\dots,\alpha_N$
 are quantum numbers needed to describe remaining quantum numbers. 
 
 It is common to use the notation 
 \begin{equation}
  \Phi(x_1,\cdots,x_N; \alpha_1,\cdots,\alpha_N) = \ket{\alpha_1 \alpha_2 \dots \alpha_N}
 \end{equation} 
 for the Slater determinant.
\end{definition}

A useful way to express the Slater determinant is via the antisymmetrization operator $\hat{A}$.
\begin{definition}[Antisymmetrization operator]
 The antisymmetrization operator $\hat{A}$ is defined as 
 \begin{equation}
  \hat{A} = \frac{1}{N!} \sum_{\sigma \in S_N} (-1)^{|\sigma|} \hat{P}_\sigma
 \end{equation}
 where the notation 
 $\sum_{\sigma \in S_N}$ means that we sum over all possible permutations of $N$ indices.
 
The expression for the Slater determinant given above, can now be written in terms of the 
antisymmetrization operator as follows
\begin{align}
  \Phi(x_1,\cdots,x_N; \alpha_1,\cdots,\alpha_N) &= \underbrace{\frac{1}{\sqrt{N!}} \sum_{\sigma \in S_N} (-1)^{|\sigma|} \hat{P}_\sigma}_\text{$= \hat{A}$} \left(\psi_{\alpha_1}(x_1) \psi_{\alpha_2}(x_2) \dots \psi_{\alpha_N}(x_N) \right) \nonumber \\
  &= \frac{1}{\sqrt{N!}} \sum_{\sigma \in S_N} (-1)^{|\sigma|}  \psi_{\alpha_1}(x_{\sigma(1)}) \psi_{\alpha_2}(x_{\sigma(2)}) \dots \psi_{\alpha_N}(x_{\sigma(N)}) \nonumber \\
  &= \sqrt{N!} \hat{A} \Phi_H(x_1,\dots,x_N)
\end{align}
where $\Phi_H(x_1,\dots,x_N) \equiv \psi_{\alpha_1}(x_1) \psi_{\alpha_2}(x_2) \dots \psi_{\alpha_N}(x_N)$.
\end{definition}

\section{Second quantization}

\subsection{Creation and annihilation operators}
\begin{definition}[Vacuum state]
 The state $\ket{0}$ represents the state containing no particles and is referred to as
 the vacuum state. Another common notation is to use $\ket{-}$. The vacuum state satisfies 
 $\braket{0|0} = 1$. It should be noted that $\ket{0} \neq 0$.
\end{definition}

\begin{definition}[The creation operator]
 We define the fermion creation operator $a_\alpha^\dagger$, creating a particle in the single-particle state $\phi_\alpha$, as
 \begin{equation}
  a_\alpha^\dagger \ket{0} \equiv \ket{\alpha}
 \end{equation}
 and 
 \begin{equation}
  a_\alpha^\dagger \ket{\alpha_1 \alpha_2 \cdots \alpha_n} = \ket{\alpha \alpha_1 \alpha_2 \cdots \alpha_n},
 \end{equation}
 if $\alpha \neq \alpha_i$ for all single-particle states in the determinant. 
 Furthermore, insertion of $\alpha$ at position $j+1$ will induce a sign change $(-1)^j$
 \begin{equation}
  a_\alpha^\dagger \ket{\alpha_1 \alpha_2 \cdots \alpha_j \alpha_{j+1} \cdots \alpha_n} = (-1)^j\ket{\alpha_1 \alpha_2 \cdots \alpha_j \alpha \alpha_{j+1} \cdots \alpha_n}.
 \end{equation}
 If $\alpha = \alpha_i$ for some single-particle state, then 
 \begin{equation}
  a_\alpha^\dagger \ket{\alpha_1 \alpha_2 \cdots \alpha_n} = \ket{\alpha \alpha_1 \alpha_2 \cdots \alpha_n} = 0
 \end{equation}
\end{definition}

\begin{definition}[The annihilation operator]
The annihilation operator, $a_\alpha$, is the Hermitian adjoint of $a_\alpha^\dagger$, i.e 
\begin{equation}
a_\alpha \equiv (a_\alpha^\dagger)^\dagger. 
\end{equation}
The annihilation operator gives the zero result when acting on the vacuum state 
\begin{equation}
 a_\alpha \ket{0} = 0.
\end{equation}
If $\ket{\Phi} = \ket{\alpha_1 \alpha_2 \cdots \alpha_{i-1} \alpha_{i} \alpha_{i+1} \cdots \alpha_n}$ is an 
arbitrary Slater determinant and $\alpha = \alpha_i$
\begin{equation}
 a_\alpha \ket{\alpha_1 \alpha_2 \cdots \alpha_{i-1} \alpha_{i} \alpha_{i+1} \cdots \alpha_n} = (-1)^{i-1}\ket{\alpha_1 \alpha_2 \cdots \alpha_{i-1}  \alpha_{i+1} \cdots \alpha_n},
\end{equation}
i.e it removes particle $i$ from the determinant. Otherwise $a_\alpha \ket{\Phi} = 0$, that is if 
$\alpha \neq \alpha_i$ for all single-particle states in the determinant.
\end{definition}

\begin{definition}[Anticommutator of operators]
The anticommutator of two operators $\hat{A}$ and $\hat{B}$ is defined as
 \begin{equation}
 \{\hat{A},\hat{B}\} \equiv \hat{A}\hat{B}+\hat{B}\hat{A}.                      
\end{equation}
\end{definition}
\begin{theorem}[The fundamental anticommutator relations]
The creation and annihilation operators satisfy the so-called "fundamental anticommutator" relations
\begin{align}
 \{a_p^\dagger,a_q^\dagger \} &= 0\\
 \{a_p,a_q \} &= 0\\
 \{a_p,a_q^\dagger \} &= \delta_{pq}.\\
\end{align}
\end{theorem}

\subsection{Representation of operators}
The matrix representation of a one-body operator, relative to a single particle basis $\{ \phi_i \}_{i=1}$, is given by 
\begin{equation}
 \braket{\phi_p|\hat{h}|\phi_q} \equiv \int \phi^*_p(\mathbf{r}) \hat{h} \phi_q(\mathbf{r}) d\mathbf{r}.
\end{equation}
Common notations are $\braket{\phi_p|\hat{h}|\phi_q}=\braket{p|\hat{h}|q}=h^p_q$. The second quantized form a one-body operator 
is given by
\begin{equation}
 \hat{H}_0 = \sum_{pq}\braket{p|\hat{h}|q} a_p^\dagger a_q.
\end{equation}
Likewise the matrix representation of a two-body operator, typically the coulomb interaction, is given by 
\begin{equation}
 \braket{\phi_p \phi_q|\hat{v}|\phi_r \phi_s} = \int \phi^*_p(\mathbf{r_1}) \phi^*_q(\mathbf{r_2}) \hat{v}(\mathbf{r}_1,\mathbf{r}_2) \phi_q(\mathbf{r_1}) \phi_s(\mathbf{r_2}) d \mathbf{r}_1 d \mathbf{r}_2.
\end{equation}
It is common to write $\braket{\phi_p \phi_q|\hat{v}|\phi_r \phi_s} = \braket{pq|\hat{v}|rs} = v^{pq}_{rs}$. Furthermore, it is customary 
to introduce the \textit{anti-symmetric} matrix element
\begin{equation}
 \braket{pq|\hat{v}|rs}_{AS} \equiv \braket{pq|\hat{v}|rs} - \braket{pq|\hat{v}|sr}.
\end{equation}
Here it should be noted that many sources drop the AS subscript, which can be a source of confusion. The second quantized form a 
two-body operator is given by
\begin{align}
 \hat{H}_I &= \frac{1}{2} \sum_{pqrs} \braket{pq|\hat{v}|rs} a_p^\dagger a_q^\dagger a_s a_r \\
 &= \frac{1}{4}\sum_{pqrs} \braket{pq|\hat{v}|rs}_{AS} a_p^\dagger a_q^\dagger a_s a_r 
\end{align}
Thus, the full Hamiltonian $\hat{H} = \hat{H}_0 + \hat{H}_I$, using anti-symmetric matrix elements is given by 
\begin{equation}
 \hat{H} =  \sum_{pq}\braket{p|\hat{h}|q} a_p^\dagger a_q + \frac{1}{4}\sum_{pqrs} \braket{pq|\hat{v}|rs}_{AS} a_p^\dagger a_q^\dagger a_s a_r.
\end{equation}

\subsection{Wicks theorem and related concepts}
\begin{definition}[Operator string]
 A sequence of creation and annihilation operators is called an operator string. Generally an operator 
string of $n$ creation and annihilation operators are on the form 
\begin{equation}
 A_1 A_2 \cdots A_n, \ \ A_i \in \{a_p,a_p^\dagger\}.
\end{equation}

For example
$$A_1A_2A_3A_4 = a_p a_q^\dagger a_r a_s^\dagger,$$
where $A_1 = a_p, A_2 = a_q^\dagger, A_3 = a_r$ and $A_4 = a_s^\dagger$.
\end{definition}
\begin{definition}[Vacuum expecatation value]
The number 
\begin{equation}
 \braket{-|A_1 A_2 \cdots A_n|-}
\end{equation}
is referred to as a vacuum expecatation value, where $A_1A_2 \cdots A_n$ is an operator string. 
\end{definition}
\begin{definition}[Normal ordered product]
The normal-ordered product form of an operator 
string $A_1A_2\cdots A_n$ is defined as a rearrangement,
\begin{equation}
 \{A_1A_2\cdots A_n \} \equiv (-1)^{|\sigma|}A_{\sigma(1)}A_{\sigma(2)}\cdots A_{\sigma(n)},
\end{equation}
where $\sigma$ is a permutation such that all the creation operators in the operator string is to the left of all the
annihilation operators, i.e.,
\begin{equation}
 \{A_1A_2\cdots A_n \} \equiv (-1)^{|\sigma|}[\text{creation operators}]\cdot[\text{annihilation operators}].
\end{equation}
\end{definition}
The permutation $\sigma$ is in general not unique, since we may permute the creation and annihilation
operators separately without affecting the total expression. For example,
\begin{equation}
 \{a_p a_q^\dagger a_r^\dagger a_s \} = a_q^\dagger a_r^\dagger a_p a_s = -a_r^\dagger a_q^\dagger a_p a_s = a_r^\dagger a_q^\dagger a_s a_p = -a_q^\dagger a_r^\dagger a_s a_p
\end{equation}
To find the permutation $\sigma$ that normal-orders an operator product, it is usually simplest to count the
number f of anticommutations necessary to achieve the rearrangement, and set $(-1)^{|\sigma |} = (-1)^f$.
Note that the string $A_1\cdots A_n \neq \{ A_1 \cdots A_n \}$ in general, since by reordering creation and annihilation
operators we neglect the extra terms arising from the Kronecker delta in the anti-commutator relation
$\{ a_p , a_q^\dagger \} = \delta_{pq}$.

\begin{definition}[Contraction]
A contraction between two arbitrary creation and annihilation operators $X$
and  $Y$, is the number defined by
\begin{equation}
 \wick{\c X \c Y} \equiv \braket{-|X Y|-}. 
\end{equation}
\end{definition}
We list the possible contractions, relative to the vacuum state $\ket{-}$,
\begin{align}
 \wick{\c a_p^\dagger \c a_q^\dagger} &= \braket{-|a_p^\dagger a_q^\dagger|-} = 0 \\
 \wick{\c a_p \c a_q} &= \braket{-|a_p a_q|-} = 0 \\
 \wick{\c a_p^\dagger \c a_q} &= \braket{-|a_p^\dagger a_q|-} = 0 \\ 
 \wick{\c a_p \c a_q^\dagger} &= \braket{-|a_p a_q^\dagger|-} = \delta_{pq}
\end{align}

Wick’s theorem states that every string of creation and annihilation operators can be written as a sum of
normal-ordered products where we perform every possible contraction. 
\begin{theorem}[Wicks theorem]
Let $A_1 \cdots A_n$ 
be an operator string of creation and annihilation operators. Then,
\begin{align*}
 A_1A_2\cdots A_n &= \{ A_1 A_2 \cdots A_n \} + \sum_{(1)} \{ A_1 \wick{ \c... \c... ... } A_n \}
 &+ \sum_{(2)} \{ A_1 \wick{ .\c1.\c2. .\c1.\c2. ... } \} + \cdots \\
 &+ \sum_{\left \lfloor \frac{n}{2} \right \rfloor } \underbrace{\{ A_1 \wick{ .\c1.\c3. \c5.\c2.\c4. \c4.\c3.\c5. \c1... .\c2.. ...} A_n \}}_\text{$\left \lfloor \frac{n}{2} \right \rfloor$ contractions}
\end{align*}
The notation $\sum_{(m)}$ signifies that we sum over all combinations of $m$ contractions.
When n is even, the last sum signifies that we sum over $n/2$ contractions, i.e., all operators are
contracted. If n is odd, there is one uncontracted operator left in each term of the last sum.
\end{theorem}

Evaluation of vacuum expecatation values are greatly simplified by Wicks theorem. First note that for any string with at least one factor 
\begin{equation}
 \braket{-|\{ A_1 \cdots A_n \}|-} = 0,
\end{equation}
because in the normal-order product, the annihilation operators are to the right, and the creation
operators are on the left. 
\begin{theorem}[Wicks theorem for vacuum expecatation values]
For odd $n$, Wicks theorem gives
\begin{equation}
 \braket{-|A_1 \cdots A_n|-} = 0,
\end{equation}
while for even $n$,
\begin{equation}
 \braket{-|A_1 \cdots A_n|-} = \sum_{\left \lfloor \frac{n}{2} \right \rfloor } \underbrace{\{ A_1 \wick{ .\c1.\c3. \c5.\c2.\c4. \c4.\c3.\c5. \c1... .\c2.. ...} A_n \}}_\text{all contracted}.
\end{equation}
\end{theorem}

\begin{fact}
 A useful fact is that the sign of a fully contracted operator product is $(-1)^k$, where $k$ is the
number of contraction line crossings.
\end{fact}

Consider an operator string on the form 
\begin{equation}
 \{ A_1 A_2 \cdots A_p \} \{B_1 B_2 \cdots B_q \} \cdots \{ Z_1 Z_2 \cdots Z_r \}
\end{equation}
where each substring is normal ordered.  Then the generalized Wicks theorem states that we only have 
to consider contractions between different substrings. That is, contractions involving two operators from the same 
substring -referred to as self contractions- do not contribute.
\begin{theorem}[The generalized Wicks theorem]
 Let $$\{ A_1 A_2 \cdots A_p \} \{B_1 B_2 \cdots B_q \} \cdots \{ Z_1 Z_2 \cdots Z_r \}$$
be an operator string consisting of normal ordered substrings. Then,
\begin{align*}
 &\{ A_1 A_2 \cdots A_p \} \{B_1 B_2 \cdots B_q \} \cdots \{ Z_1 Z_2 \cdots Z_r \} = \{  A_1 A_2 \cdots A_p \vdots B_1 B_2 \cdots B_q \vdots \cdots \vdots Z_1 Z_2 \cdots Z_r  \} \\
 &+ \sum_{(1)}^{'} \{  \wick{ A_1 \c A_2  \cdots A_p \vdots B_1 B_2 \cdots \c B_q } \vdots \cdots \vdots Z_1 Z_2 \cdots Z_r  \} \\
 &+ \sum_{(2)}^{'}  \{  \wick{ A_1 \c1 A_2  \cdots A_p \vdots B_1 \c2 B_2 \cdots \c1 B_q  \vdots \cdots \vdots \c2 Z_1 Z_2 \cdots Z_r } \}  \\
 &+ \cdots \\
 &+ \sum_{\left \lfloor \frac{n}{2} \right \rfloor }^{'} \underbrace{ \{  \wick{ \c5 A_1 \c1 A_2  \c4 \cdots \c3 A_p \vdots \c4 B_1 \c2 B_2 \cdots \c1 B_q  \vdots \cdots \vdots \c2 Z_1 \c3 Z_2 \c5 \cdots Z_r } \}}_\text{$\left \lfloor \frac{n}{2} \right \rfloor$ contractions}
\end{align*}
The notation $\sum_{(m)}^{'}$ signifies that we sum over all combinations of $m$ contractions that each involve
operators from different substrings. The vertical dots are only a bookkeeping device to remind ourselves
which operators belong to which substrings.
\end{theorem}

\subsection{Particle-hole formalism}
Consider a $N$-particle system and the single-particle functions $\{ \phi_p \}_{p=1}^L$, with $L$ denoting 
the number of single-particle functions. In particle-hole formalism we divide the single-particle functions 
into two sets. The first $N$ single-particle functions, $\{ \phi \}_{i=1}^N$, are referred to as 
particle/occupied-states/functions while the $N+1,\cdots,L$ states, $\{ \phi \}_{a=N+1}^L$, are referred to as 
hole/unoccupied/virtual states. The indices $i,j,k,\cdots$ are reserved for particle states and indices $a,b,c,\cdots$
are reserved for hole states. General states, particle or hole, are indexed by $p,q,r,\cdots$.

\begin{definition}[The reference state]
 We define the reference Slater determinant $\ket{c}$ to be the determinant constructed from the particle states, i.e 
 \begin{equation}
  \ket{c} = \prod_{i=1}^N a_i^\dagger \ket{0} = \ket{\phi_1 \phi_2 \cdots \phi_N}.
 \end{equation}
Other common notations for the reference state are $\ket{c} = \ket{\Phi_0} = \ket{\Phi}_\text{ref}$.
\end{definition}

\begin{definition}[Excited determinants]
 From the reference determinant, we can construct excited determinants using the creation and annihilation operators,
 substituting one or more particle states with hole states.
 \begin{align*}
  \ket{\Phi_i^a} &= a_a^\dagger a_i \ket{c}, \ \ \text{single excitation} \\
  \ket{\Phi_{ij}^{ab}} &= a_b^\dagger a_j a_a^\dagger a_i \ket{c}, \ \ \text{ double excitation}, \\
  &\vdots
 \end{align*}
\end{definition}

\begin{definition}[Quasiparticle creation and annihilation operators]
 Quasiparticle creation and annihilation operators $b_p,b_p^\dagger$ are defined as follows
 \begin{align}
  b_i &= a_i^\dagger, \ \ b_a = a_a \\
  b_i^\dagger &= a_i, \ \ b_a^\dagger = a_a^\dagger.
 \end{align}
 Thus, for hole indices, quasiparticle creation operators are the ordinary creation operators, but for
particle states, creating a quasiparticle is the same as destroying a particle in a particle state. One says
that $b_i^\dagger$ creates a hole, while $b_a^\dagger$ creates a particle.

Furthermore, the fundamental anticommutator relations are preserved for quasiparticle creation and annihilation operators
\begin{align}
 \{b_p^\dagger,b_q^\dagger \} &= 0, \\
 \{b_p,b_q \} &= 0, \\
 \{b_p,b_q^\dagger \} &= \delta_{pq}.
\end{align}
\end{definition}

\begin{definition}[The new vacuum state]
Making the observation that 
\begin{equation}
 b_p \ket{c} = 0 \text{ for all } p,
\end{equation}
we see that the reference acts for quasiparticle creation/annihilation as the vacuum 
did for the ordinary creation/annihilation operators. Thus, we define the reference $\ket{c}$ to be
vacuum state for the quasiparticle operators. We will often refer to $\ket{c}$ as the new vacuum state in 
the following.
\end{definition}
\begin{fact}
The definition of normal-ordering, contractions, and ultimately Wick's
theorem, depended only on the anticommutator relations and the property $$a_p \ket{0} = .0$$ Thus, Wick's
theorem is valid also for quasiparticles.
\end{fact}

Notice that relative to the new vacuum state, there are now two non-zero contractions
\begin{align} 
 \wick{\c b_a \c b_b^\dagger} = \wick{\c a_a \c a_b^\dagger} &= \braket{c|a_a a_b^\dagger|c} = \delta_{ab} \\
 \wick{\c b_i \c b_j^\dagger} = \wick{\c a_i^\dagger \c a_j} &= \braket{c|a_i^\dagger a_j|c} = \delta_{ij}. \\
\end{align}

\subsection{Normal ordering relative to the new vacuum state}
If we normal order relative to the new vacuum $\ket{c}$ the one-body Hamiltonian can be written 
\begin{equation}
 \hat{H}_0 = H_0^{(1\text{-body})} + H_0^{(0\text{-body})} 
\end{equation}
where we have defined 
\begin{align}
 H_0^{(0\text{-body})} &\equiv \sum_i \braket{i|\hat{h}|i} \\
 H_0^{(1\text{-body})} &\equiv \sum_{pq} \braket{p|\hat{h}|q} \{a_p^\dagger a_q \}.
\end{align}
The notation $H^{n\text{-body}}$ means that the operator acts as a $n$-body operator, with $0$-body being the same 
as a constant.

The two-body Hamiltonian can be written as 
\begin{align}
 \hat{H}_I &= \hat{H}_I^{(2\text{-body})} + \hat{H}_I^{(1\text{-body})} + \hat{H}_I^{(0\text{-body})} \\
\end{align}
where 
\begin{align}
 \hat{H}_I^{(2\text{-body})} &= \frac{1}{4}\sum_{pqrs} \braket{pq|\hat{v}|rs}_{\text{AS}}\{a_p^\dagger a_q^\dagger a_s a_r \}, \\
 \hat{H}_I^{(1\text{-body})} &= \sum_{pqi} \braket{pi|\hat{v}|qi}_{\text{AS}} \{a_p^\dagger a_q \}, \\
 \hat{H}_I^{(0\text{-body})} &= \frac{1}{2} \sum_{ij} \braket{ij|\hat{v}|ij}_{\text{AS}}
\end{align}
We now define 
\begin{align*}
 E_0^{\text{ref}} &\equiv H_0^{(0\text{-body})} + \hat{H}_I^{(0\text{-body})} = \sum_i \braket{i|\hat{h}|i} + \frac{1}{2} \sum_{ij} \braket{ij|\hat{v}|ij}_{\text{AS}} \\
 \{ \hat{F} \} &\equiv H_0^{(1\text{-body})} + \hat{H}_I^{(1\text{-body})} = \sum_{pq} \braket{p|\hat{f}|q} \{a_p^\dagger a_q\} \\
 \{ \hat{H}_I \} &\equiv \hat{H}_I^{(2\text{-body})}
\end{align*}
where we defined the so-called fock matrix element 
\begin{equation}
 \braket{p|\hat{f}|q} = \braket{p|\hat{h}|q} + \sum_i \braket{pi|\hat{v}|qi}_{\text{AS}}
\end{equation}
$E_0^\text{ref}$ is referred to as the reference energy and $\{ \hat{F} \}$ is the Fock-operator.

Thus, we can write the full Hamiltonian normal ordered relative to the new vacuum as 
\begin{equation}
 \hat{H} =  E_0^{\text{ref}} + \{ \hat{F} \} + \{ \hat{H}_I \}.
\end{equation}

When computing expectation values using the normal ordered Hamiltonian we can now use Wicks generalized theorem, considering 
only contractions between normal ordered substrings. Recall carefully, that there are two possible non-zero contractions 
relative to the new vacuum state.
\begin{example}
 Consider the computation of the matrix element $\braket{c|\{ \hat{F} \}| \Phi_i^a}$. By Wicks generalized theorem we have 
 \begin{align*}
  \braket{c|\{ \hat{F} \}| \Phi_i^a} = \sum_{\left \lfloor \frac{n}{2} \right \rfloor }^{'} \braket{c|\left( \sum_{pq}\braket{p|\hat{f}|q}\{a_p^\dagger a_q\}a_a^\dagger a_i \right)|c},
 \end{align*}
 With $\sum_{\left \lfloor \frac{n}{2} \right \rfloor }^{'}$ meaning that we sum over all possible contractions between 
different normal ordered substrings. 

Note that $a_a^\dagger a_i$ already is on normal order form. The computation is 
also simplified by noting that there is only one possible way of contracting all operators and get a non-zero result, that is 
\begin{equation}
 \wick { \{ \c1 a_p^\dagger \c2 a_q \} \c2 a_a^\dagger \c1 a_i } = \delta_{pi}\delta_{qa}
\end{equation}
with a positive sign since there are two line crossings. Thus,
\begin{align*}
  \braket{c|\{ \hat{F} \}| \Phi_i^a} &= \braket{c|\left(\sum_{pq}\braket{p|\hat{f} |q}\delta_{pi}\delta_{qa} \right)|c} \\
  &= \braket{i|\hat{f}|a} \braket{c|c} \\
  &= \braket{i|\hat{f}|a}.
\end{align*}
\end{example}

\section{Many-body methods}

\subsection{Configuration Interaction Theory}

\subsection{Hartree-Fock theory}
\subsubsection{The Hartree-Fock equations}
In Hartree-Fock theory the underlying idea is to approximate the true wavefunction of a $N$-particle system 
$\ket{\Psi}$ by a \textbf{single} Slater determinant 
\begin{equation}
 \ket{\Phi} = \ket{\psi_1 \psi_2 \dots \psi_N}
\end{equation}
where the $\psi_i$'s are orthonormal single-particle functions. 

In Hartree-Fock we know the form of the wavefunction 
which by assumption is a single Slater determinant. The unknowns are the single-particle functions $\psi_i$.

How do we go about finding these single-particle functions? The idea is to expand the $\psi_i$'s in a fixed 
known (finite in practice) basis $\{ \phi_\lambda \}_{\lambda=1}^L$ which we take to be orthonormal for 
simplicity ($\braket{\phi_p|\phi_q} = \delta_{pq}$)
\begin{equation}
 \psi_p = \sum_\lambda C_{p \lambda} \phi_\lambda.
\end{equation}
Typically the basis functions $\phi_\lambda$ are chosen as the eigenfunctions of the single-particle operator $\hat{h}_0$
such that
\begin{equation}
 \hat{h}_0 \phi_\lambda = \epsilon_\lambda \phi_\lambda.
\end{equation}

Now, the unknowns are the expansion coefficients $\{ C_{p\lambda} \}$ which determines $\psi_i$ in terms of 
the $\phi_\lambda$'s. The constraint that the $\psi_i$'s must be orthonormal gives that we must have 
\begin{equation}
 \braket{\psi_i|\psi_j} = \delta_{ij} = \sum_{\alpha} C^*_{i\alpha} C_{i \alpha}. 	
\end{equation}

The equation for the expansion coefficients is obtained by minimizing the energy of a single Slater determinant
\begin{align*}
 E[\Phi] &= \sum_i \braket{i|\hat{h}_0|i} + \frac{1}{2} \sum_{ij} \braket{ij|v|ij}_{AS} \\
 &= \sum_i \sum_{\alpha \beta} C^*_{i \alpha}C_{i \beta} \braket{\phi_\alpha|\hat{h}_0|\phi_\beta} + \frac{1}{2} \sum_{ij} \sum_{\alpha \beta \gamma \delta} C^*_{i \alpha} C^*_{j \beta} C_{i \gamma} C_{j \delta} \braket{\phi_\alpha \phi_\beta|v|\phi_\gamma \phi_\delta}_{AS},
\end{align*}
with respect to the expansion coefficients subject to the constraint that the $\psi_i$'s must be orthonormal. Here the notation 
$\braket{\phi_\alpha|\hat{h}_0|\phi_\beta}$ and $\braket{\phi_\alpha \phi_\beta|v|\phi_\gamma \phi_\delta}_{AS}$ are meant 
as a reminder that matrix elements are with respect to the fixed basis functions $\phi_\lambda$.

The constraint is incorporated via so-called Lagrange multipliers such that in order to minimize $E[\Phi]$ we have to solve 
\begin{equation}
 \frac{d}{d C^*_{i \alpha}} \left[ E[\Phi] - \underbrace{\sum_j \epsilon_j \sum_\alpha C^*_{j \alpha} C_{j \alpha}}_\text{The constraint} \right] = 0.
\end{equation}
Taking derivatives one obtains 
\begin{equation}
 \sum_\beta \left\{\braket{\alpha|\hat{h}|\beta} + \sum_j \sum_{\gamma \delta} C^*_{j \gamma} C_{j \delta} \braket{\alpha \gamma | \hat{v} | \beta \delta}_\text{AS} \right\}C_{i \beta} = \epsilon_i C_{i \alpha} \label{HfEquation}.
\end{equation}
Defining the so-called Fock matrix  
\begin{equation}
 h_{\alpha \beta}^{\text{HF}} \equiv \braket{\alpha|\hat{h}|\beta} + \sum_j \sum_{\gamma \delta} C^*_{j \gamma} C_{j \delta} \braket{\alpha \gamma | \hat{v} | \beta \delta}_\text{AS}
\end{equation}
we can rewrite \ref{HfEquation} as
\begin{equation}
 \sum_{\beta} h_{\alpha \beta}^{\text{HF}} C_{i \beta} = \epsilon_i^{\text{HF}} C_{i \alpha} \label{HfEquation2}.
\end{equation}

Now let $C$ be the matrix that consists of the $C_{i \beta}$'s. By definition $[C^\dagger]_{\beta i} = [C]_{i \beta}$, 
i.e element $(\beta,i)$ of the \textbf{matrix} $C^\dagger$ is equal to the matrix element $C_{i \beta}$.
Then \ref{HfEquation2} can be written as 
\begin{align}
 \sum_{\beta} h_{\alpha \beta}^{\text{HF}} [C^\dagger]_{\beta i} = [C^\dagger]_{\alpha i}\epsilon_i^{\text{HF}} \\
\end{align}
where we recognize the left-hand side as a matrix-matrix product $h^\text{HF} C^\dagger$ where $h^\text{HF}$ is the matrix 
having $h_{\alpha \beta}^{\text{HF}}$ as its elements. If we define 
$\mathcal{E} = \text{diag}(\epsilon_1,\epsilon_2,\dots,\epsilon_N)$ to be the diagonal matrix containing $\epsilon_i$ along its diagonal 
the right-hand side is the matrix-matrix product $C^\dagger \mathcal{E}$. Thus, Eq. \ref{HfEquation2} can be expressed as the 
non-linear eigenvalue problem 
\begin{equation}
 h^\text{HF}(C) C^\dagger = C^\dagger \mathcal{E}.
\end{equation}

The above equation is commonly referred to as the Roothan-Hall equations. Thus we can solve for $C$, which 
are the unknowns we are interested in, by diagonalizing $h^\text{HF}$. However, the fact that the equation is non-linear
due to the fact that $h^{\text{HF}}$ depends on $C$, means that we must solve the Roothan-Hall equations iteratively.
We will give a outline of one common approach to solve the Roothan-Hall equations.

\begin{warning}
 Please note that the eigenvalues, $\mathcal{E}$, you obtain by diagonalizing $h^\text{HF}$ is \textbf{not} 
 the Hartree-Fock energy. However they are interpreted as ionization energies through what is called Koopmans theorem.
 
 The Hartree-Fock energy on the other hand, is obtained by evaluating 
 \begin{equation}
  E[\Phi] = \sum_i \sum_{\alpha \beta} C^*_{i \alpha}C_{i \beta} \braket{\phi_\alpha|\hat{h}_0|\phi_\beta} + \frac{1}{2} \sum_{ij} \sum_{\alpha \beta \gamma \delta} C^*_{i \alpha} C^*_{j \beta} C_{i \gamma} C_{j \delta} \braket{\phi_\alpha \phi_\beta|v|\phi_\gamma \phi_\delta}_{AS}
 \end{equation}
  with the new coefficients you obtain by diagonalizing $h^\text{HF}$.
\end{warning}



\subsubsection{Outline of how to write a Hartree-Fock program}




\subsection{Many-body Perturbation Theory}

\subsection{Coupled Cluster Theory}

\end{document}          
