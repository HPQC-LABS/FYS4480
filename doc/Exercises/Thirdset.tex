\documentclass[prc]{revtex4}
\usepackage[dvips]{graphicx}
\usepackage{mathrsfs}
\usepackage{amsfonts}
\usepackage{lscape}

\usepackage{epic,eepic}
\usepackage{amsmath}
\usepackage{amssymb}
\usepackage[dvips]{epsfig}
\usepackage[T1]{fontenc}
\usepackage{hyperref}
\usepackage{bezier}
\usepackage{pstricks}
\usepackage{dcolumn}% Align table columns on decimal point
\usepackage{bm}% bold math
%\usepackage{braket}
\usepackage[dvips]{graphicx}
\usepackage{pst-plot}

\newcommand{\One}{\hat{\mathbf{1}}}
\newcommand{\eff}{\text{eff}}
\newcommand{\Heff}{\hat{H}_\text{eff}}
\newcommand{\Veff}{\hat{V}_\text{eff}}
\newcommand{\braket}[1]{\langle#1\rangle}
\newcommand{\Span}{\operatorname{sp}}
\newcommand{\tr}{\operatorname{trace}}
\newcommand{\diag}{\operatorname{diag}}
\newcommand{\bra}[1]{\left\langle #1 \right|}
\newcommand{\ket}[1]{\left| #1 \right\rangle}
\newcommand{\element}[3]
    {\bra{#1}#2\ket{#3}}

\newcommand{\normord}[1]{
    \left\{#1\right\}
}

\usepackage{amsmath}
\begin{document}

\title{Exercises FYS4480, Third set}
%\author{}
\maketitle



\subsection*{Exercise 5}
Starting with the Slater determinant 
\[
\Phi_{0}=\prod_{i=1}^{n}a_{\alpha_{i}}^{\dagger}\ket{0},
\]
use Wick's theorem to compute the normalization integral
$<\Phi_{0}|\Phi_{0}>$.



\subsection*{Exercise 6}
Write the two-particle operator
\[
G=\frac{1}{4}\sum_{\alpha\beta\gamma\delta}\bra{\alpha\beta}
g\ket{\gamma\delta}a_{\alpha}^{\dagger}a_{\beta}^{\dagger}
a_{\delta}a_{\gamma}
\]
in the quasi-particle representation for particles and holes
\[
b_{\alpha}^{\dagger}=\left\{\begin{array}{c}
a_{\alpha}^{\dagger}\\a_{\alpha}\end{array}\right.
\hspace{1cm}
b_{\alpha}=\left\{\begin{array}{cc}a_{\alpha}&\alpha>
\alpha_{F}\\
a_{\alpha}^{\dagger}&\alpha \leq \alpha_{F}
\end{array}\right.
\]
The two-body matrix elements are antisymmetric.


\subsection*{Exercise 7}
Use the results from exercise 6 and Wick's theorem to calculate 
\[
\bra{\beta_{1}\gamma_{1}^{-1}}G\ket{\beta_{2}\gamma_{2}^{-1}}
\]
You need to consider that case that
$\beta_{1}$ be equal
$\beta_{2}$ and that $\gamma_{1}$ be equal $\gamma_{2}$.


\subsection*{Exercise 8}
Show that the onebody part of the Hamiltonian
    \begin{equation*}
        \hat{H}_0 = \sum_{pq} \element{p}{\hat{h}_0}{q} a^\dagger_p a_q
    \end{equation*}
can be written, using standard annihilation and creation operators, in normal-ordered form as 
    \begin{align*}
        \hat{H}_0 &= \sum_{pq} \element{p}{\hat{h}_0}{q} a^\dagger_p a_q \nonumber \\
            &= \sum_{pq} \element{p}{\hat{h}_0}{q} \left\{a^\dagger_p a_q\right\} + 
                \delta_{pq\in i} \sum_{pq} \element{p}{\hat{h}_0}{q} \nonumber \\
            &= \sum_{pq} \element{p}{\hat{h}_0}{q} \left\{a^\dagger_p a_q\right\} +
                \sum_i \element{i}{\hat{h}_0}{i}
    \end{align*}
Explain the meaning of the various symbols. Which reference 
vacuum has been used?

\subsection*{Exercise 9}
Show that the twobody part of the Hamiltonian
    \begin{equation*}
        \hat{H}_I = \frac{1}{4} \sum_{pqrs} \element{pq}{\hat{v}}{rs} a^\dagger_p a^\dagger_q a_s  a_r
    \end{equation*}
can be written, using standard annihilation and creation operators, in normal-ordered form as 
    \begin{align*}
    \hat{H}_I &= \frac{1}{4} \sum_{pqrs} \element{pq}{\hat{v}}{rs} a^\dagger_p a^\dagger_q a_s  a_r \nonumber \\
        &= \frac{1}{4} \sum_{pqrs} \element{pq}{\hat{v}}{rs} \normord{a^\dagger_p a^\dagger_q a_s  a_r}
            + \sum_{pqi} \element{pi}{\hat{v}}{qi} \normord{a^\dagger_p a_q} 
            + \frac{1}{2} \sum_{ij} \element{ij}{\hat{v}}{ij}
    \end{align*}
Explain again the meaning of the various symbols.

     Derive the normal-ordered form of the threebody part of the Hamiltonian.
    \begin{align*}
    \hat{H}_3 &= \frac{1}{36} \sum_{\substack{
                        pqr \\
                        stu}}
                 \element{pqr}{\hat{v}_3}{stu} a^\dagger_p a^\dagger_q a^\dagger_r a_u a_t a_s\\
    \end{align*}
and specify the contributions to the twobody, onebody and the scalar part.


\end{document}
