\documentclass[a4paper, 11pt, notitlepage, english]{article}

\usepackage{babel}
\usepackage[utf8]{inputenc}
\usepackage[T1]{fontenc, url}
\usepackage{textcomp}
\usepackage{amsmath, amssymb}
\usepackage{amsbsy, amsfonts}
\usepackage{graphicx, color}
\usepackage{parskip}
\usepackage{framed}
\usepackage{amsmath}
\usepackage{xcolor}
\usepackage{multicol}
\usepackage{url}
\usepackage{flafter}


\usepackage{geometry}
\geometry{headheight=0.01mm}
\geometry{top=24mm, bottom=29mm, left=39mm, right=39mm}

\renewcommand{\arraystretch}{2}
\setlength{\tabcolsep}{10pt}
\makeatletter
\renewcommand*\env@matrix[1][*\c@MaxMatrixCols c]{%
  \hskip -\arraycolsep
  \let\@ifnextchar\new@ifnextchar
  \array{#1}}
%
% Parametere for inkludering av kode fra fil
%
\usepackage{listings}
\lstset{language=python}
\lstset{basicstyle=\ttfamily\small}
\lstset{frame=single}
\lstset{keywordstyle=\color{red}\bfseries}
\lstset{commentstyle=\itshape\color{blue}}
\lstset{showspaces=false}
\lstset{showstringspaces=false}
\lstset{showtabs=false}
\lstset{breaklines}

%
% Definering av egne kommandoer og miljøer
%
\newcommand{\dd}[1]{\ \text{d}#1}
\newcommand{\f}[2]{\frac{#1}{#2}} 
\newcommand{\beq}{\begin{equation}}
\newcommand{\eeq}{\end{equation}}
\newcommand{\bra}[1]{\langle #1|}
\newcommand{\ket}[1]{|#1 \rangle}
\newcommand{\braket}[2]{\langle #1 | #2 \rangle}
\newcommand{\braup}[1]{\langle #1 \left|\uparrow\rangle\right.}
\newcommand{\bradown}[1]{\langle #1 \left|\downarrow\rangle\right.}
\newcommand{\av}[1]{\left| #1 \right|}
\newcommand{\op}[1]{\hat{#1}}
\newcommand{\braopket}[3]{\langle #1 | {#2} | #3 \rangle}
\newcommand{\ketbra}[2]{\ket{#1}\bra{#2}}
\newcommand{\pp}[1]{\frac{\partial}{\partial #1}}
\newcommand{\ppn}[1]{\frac{\partial^2}{\partial #1^2}}
\newcommand{\up}{\left|\uparrow\rangle\right.}
\newcommand{\upup}{\left|\uparrow\uparrow\rangle\right.}
\newcommand{\down}{\left|\downarrow\rangle\right.}
\newcommand{\downdown}{\left|\downarrow\downarrow\rangle\right.}
\newcommand{\updown}{\left|\uparrow\downarrow\rangle\right.}
\newcommand{\downup}{\left|\downarrow\uparrow\rangle\right.}
\newcommand{\bupup}{\left.\langle\uparrow\uparrow\right|}
\newcommand{\bdowndown}{\left.\langle\downarrow\downarrow\right|}
\newcommand{\bupdown}{\left.\langle\uparrow\downarrow\right|}
\newcommand{\bdownup}{\left.\langle\downarrow\uparrow\right|}
\renewcommand{\d}{{\rm d}}
\newcommand{\Res}[2]{{\rm Res}(#1;#2)}
\newcommand{\To}{\quad\Rightarrow\quad}
\newcommand{\eps}{\epsilon}
\newcommand{\inner}[2]{\langle #1 , #2 \rangle}


\newcommand{\bt}[1]{\boldsymbol{#1}}
\newcommand{\mat}[1]{\textsf{\textbf{#1}}}
\newcommand{\I}{\boldsymbol{\mathcal{I}}}
\newcommand{\p}{\partial}
%
% Navn og tittel
%
\title{Exercises week 36, September 5-9}
\author{FYS4480, Fall semester 2022}
\begin{document}
\maketitle

\section*{Exercise 1}
In this exercise, we are looking at a general $N$-particle Slater determinant, which we will denote $\ket{SD}$. We also look at three other $N$-particle SDs, that have one, two and three non\-coincidence with the first SD, meaning they have certain single-particle states replaced by others. The notation we use to denote this is as follows, the state $\ket{SD}_i^j$ has had the single-particle state $\phi_i$ replaced by $\phi_j$---this of course assumes that $\phi_j$ was not a part of the original SD. Throughout these exercises, we will use the fact that these SDs form an orthonormal set, in the sense that
\begin{align*}
\braket{SD}{SD} &= 1, \\
\braket{SD}{SD_i^j} &= 0, \\
\braket{SD_i^j}{SD_i^j} &= 1,
\end{align*}
and so on.

We will look at the matrix elements of general onebody and twobody operators represented in the basis of these SDs. The operators are as usual
$$\op{F} = \sum_i \op{f}(x_i), \qquad \op{G}=\sum_{i<j}\op{g}(x_i,x_j).$$
which in second quantization can be written as
\begin{align*}
\op{F} &= \sum_{\alpha, \beta} \braopket{\alpha}{\op{f}}{\beta}a_\alpha^\dag a_\beta, \\
\op{G} &= \frac{1}{2}\sum_{\alpha, \beta, \gamma, \delta} \braopket{\alpha\beta}{\op{g}}{\gamma\delta}a_\alpha^\dag a_\beta^\dag a_\delta a_\gamma.
\end{align*}

\clearpage

\subsection*{a)}
We start by simply looking at the expectation values of $\ket{SD}$ for the onebody and twobody operators.

\subsubsection*{Onebody operator}
When we insert the onebody operator, we see that only the terms where $\alpha=\beta$ survive and we are left with $N$ contributions to the expectation value
\begin{align*}
\braopket{SD}{\op{F}}{SD} 
&= \sum_{\alpha, \beta} \braopket{\alpha}{\op{f}}{\beta}\braopket{SD}{a_\alpha^\dag a_\beta}{SD} \\&= \sum_\alpha \braopket{\alpha}{\op{f}}{\alpha}.		
\end{align*}

\subsubsection*{Twobody operator}
When inserting the twobody operator, we have
\begin{align*}
\braopket{SD}{\op{F}}{SD}
&= \frac{1}{2}\sum_{\alpha, \beta, \gamma, \delta} \braopket{\alpha\beta}{\op{g}}{\gamma\delta}\braopket{SD}{a_\alpha^\dag a_\beta^\dag a_\delta  a_\gamma}{SD}.\\
\end{align*}
We can firste note that any term where $\alpha=\beta$ or $\gamma=\delta$ vanish, because
$$\op{a}_\alpha^\dag \op{a}_\alpha^\dag \ket{SD} = 0 \quad \mbox{and} \quad \op{a}_\alpha \op{a}_\alpha \ket{SD} = 0.$$
Next we note that to get any contribution, we must add the same states back that we annihilate, meaning either $\alpha = \gamma$ and $\beta = \delta$ or vice versa. We then have
\begin{align*}
\braopket{SD}{\op{F}}{SD}
= \frac{1}{2}\sum_{\alpha, \beta}\bigg(
&\braopket{\alpha\beta}{\op{g}}{\alpha\beta}\braopket{SD}{a_\alpha^\dag a_\beta^\dag a_\delta  a_\alpha}{SD} \\[-0.33cm]
&\quad+ \braopket{\alpha\beta}{\op{g}}{\beta\alpha}\braopket{SD}{a_\alpha^\dag a_\beta^\dag a_\alpha  a_\beta}{SD}\bigg).
\end{align*}
To simplify further, we use the anticommutation relations:
$$\op{a}_\alpha^\dag \op{a}_\beta^\dag = -\op{a}_\beta^\dag \op{a}_\alpha^\dag, \quad \op{a}_\alpha \op{a}_\beta = -\op{a}_\beta \op{a}_\alpha, \qquad \op{a}_\alpha^\dag \op{a}_\beta = \delta_{\alpha\beta} - \op{a}_\beta \op{a}_\alpha^\dag,$$
and also the number operator
$$\op{a}_\alpha^\dag \op{a}_\alpha = \op{n}_\alpha,$$
we find that
\begin{align*}
\braopket{SD}{a_\alpha^\dag a_\beta^\dag a_\beta a_\alpha}{SD} &= \braopket{SD}{a_\alpha^\dag a_\alpha a_\beta^\dag a_\beta}{SD} = \braopket{SD}{\op{n}_\alpha \op{n}_\beta}{SD} = 1\\
\braopket{SD}{a_\alpha^\dag a_\beta^\dag a_\alpha  a_\beta}{SD} &= -\braopket{SD}{a_\alpha^\dag  a_\alpha  a_\beta^\dag a_\beta}{SD} = -\braopket{SD}{\op{n}_\alpha \op{n}_\beta}{SD} = -1.
\end{align*}

Combining these results gives
\begin{align*}
\braopket{SD}{\op{G}}{SD}
=  \frac{1}{2}\sum_{\alpha,\beta} \bigg(\braopket{\alpha\beta}{\op{g}}{\alpha\beta} - \braopket{\alpha\beta}{\op{g}}{\beta\alpha}\bigg).
\end{align*}
We recognize the \emph{Hartree} and \emph{Fock} terms respectively. Note that all terms where $\alpha=\beta$ will vanish, as expected.

\subsection*{b)}
We now look at the cross terms where there is one noncoincidence between the SDs, i.e., $\ket{SD}$ and $\ket{SD_i^j}$. 

\subsubsection*{Onebody operator}
When we insert the onebody operator, we get
\begin{align*}
\braopket{SD}{\op{F}}{SD_i^j} 
&= \sum_{\alpha, \beta} \braopket{\alpha}{\op{f}}{\beta}\braopket{SD}{a_\alpha^\dag a_\beta}{SD_i^j}.
\end{align*}
As $\ket{SD_i^j}$ has had its $\phi_i$ state replaced with $\phi_j$, we know that
$$\op{a}_i^\dag \op{a}_j\ket{SD_i^j} = \ket{SD}.$$
For all other combinations of $\alpha$ and $\beta$, the noncoincidence makes the integral vanish, so we are left with only one contribution
\begin{align*}
\braopket{SD}{\op{F}}{SD_i^j} 
&= \braopket{i}{\op{f}}{j}.
\end{align*}

\subsubsection*{Twobody operator}
For the twobody operator, we have the expression
\begin{align*}
\braopket{SD}{\op{G}}{SD_i^j}
&= \frac{1}{2}\sum_{\alpha, \beta, \gamma, \delta} \braopket{\alpha\beta}{\op{g}}{\gamma\delta}\braopket{SD}{a_\alpha^\dag a_\beta^\dag a_\delta a_\gamma}{SD_i^j}.
\end{align*}
For any term to contribute to the final sum the noncoincidence must be removed, this means we can reduce the sum over all four indices to a single sum as follows
\begin{align*}
\braopket{SD}{\op{G}}{SD_i^j}
= \frac{1}{2}\sum_{\alpha} \big[
&\braopket{\alpha i}{\op{g}}{\alpha j}\braopket{SD}{a_\alpha^\dag a_i^\dag a_j a_\alpha}{SD_i^j} \\[-0.3cm]
&\quad+ \braopket{\alpha i}{\op{g}}{j \alpha}\braopket{SD}{a_\alpha^\dag a_i^\dag a_\alpha a_j}{SD_i^j} \\
&\qquad+ \braopket{i \alpha}{\op{g}}{\alpha j}\braopket{SD}{a_i^\dag a_\alpha^\dag a_j a_\alpha}{SD_i^j} \\
&\qquad\quad+ \braopket{i \alpha}{\op{g}}{j \alpha}\braopket{SD}{a_i^\dag a_\alpha^\dag a_\alpha a_j}{SD_i^j}\big].
\end{align*}
Unlike the previous exercise, where there were only two terms, we now have four terms. However, we can simplify these by using the fact that generally $\braopket{\alpha\beta}{\op{q}}{\gamma\delta} = \braopket{\beta\alpha}{\op{q}}{\delta\gamma}$. Combining this with the anticommutation relations, we find that 
\begin{align*}
\braopket{SD}{\op{G}}{SD_i^j}
= \sum_{\alpha} \bigg(\braopket{\alpha i}{\op{g}}{\alpha j} - \braopket{\alpha i}{\op{g}}{j \alpha}\bigg).
\end{align*}

\clearpage

\subsection*{c)}
We now look at the cross term where there are two noncoincidences between the SDs, i.e., $\ket{SD}$ and $\ket{SD_{ij}^{kl}}$.

\subsection*{Onebody operator}
Again we start with the definition of the onebody operator
\begin{align*}
\braopket{SD}{\op{F}}{SD_{ij}^{kl}} 
&= \sum_{\alpha, \beta} \braopket{\alpha}{\op{f}}{\beta}\braopket{SD}{a_\alpha^\dag a_\beta}{SD_{ij}^{kl}}.
\end{align*}
We see that for all $\alpha$ and $\beta$, there will always be at least one noncoincidence between the SDs, and all terms vanish. We simply end up with
\begin{align*}
\braopket{SD}{\op{F}}{SD_{ij}^{kl}} 
&= 0.
\end{align*}

\subsection*{Twobody operator}
From the definition of the twobody operator we have
\begin{align*}
\braopket{SD}{\op{G}}{SD_{ij}^{kl}}
&= \frac{1}{2}\sum_{\alpha, \beta, \gamma, \delta} \braopket{\alpha\beta}{\op{g}}{\gamma\delta}\braopket{SD}{a_\alpha^\dag a_\beta^\dag a_\delta a_\gamma}{SD_{ij}^{kl}}.
\end{align*}
We see that only the terms where the $\phi_k$ and $\phi_l$ states are annhilated and the states $\phi_i \phi_j$ are constructed contribute, so we have
\begin{align*}
\braopket{SD}{\op{G}}{SD_{ij}^{kl}}
= \frac{1}{2}\big[
&\braopket{i j}{\op{g}}{k l}\braopket{SD}{a_i^\dag a_j^\dag a_k a_l}{SD_{ij}^{kl}} \\[-0.1cm]
&\quad+ \braopket{i j}{\op{g}}{l k}\braopket{SD}{a_i^\dag a_j^\dag a_k a_l}{SD_{ij}^{kl}} \\
&\qquad+ \braopket{j i}{\op{g}}{k l}\braopket{SD}{a_j^\dag a_i^\dag a_l a_k}{SD_{ij}^{kl}} \\
&\qquad\quad+ \braopket{j i}{\op{g}}{l k}\braopket{SD}{a_j^\dag a_i^\dag a_k a_l}{SD_{ij}^{kl}}\big].
\end{align*}
And we see that again we get four terms. Using the same simplifications we used in the previous exercise, we find
$$\braopket{SD}{\op{G}}{SD_{ij}^{kl}} = \braopket{ij}{\op{g}}{kl} - \braopket{ij}{\op{g}}{lk}$$


\subsubsection*{More than two noncoincidences}
We can now ask ourselves what the matrix elements would be if there were three or more noncoincidences between the SDs, e.g., $\braopket{SD}{\op{G}}{SD_{ijk}^{lmn}}$.

Let us look closer at our results so far. When there were no noncoincidences, the onebody operator had $N$ terms, one for each particle, and the twobody operator had $N^2/2$ terms\footnote{Technically, there are $(N-1)^2$ terms in the sum, as each pair of particles actually contribute twice, once with their direct term, and one for their exchange term. Yet no particle forms a pair with themselves. However, in this discussion let us lump the exchange and direct terms together as one.}, one for each pair of particles. When we introduced one noncoincidence, the onbody operator matrix elements reduced to having only one term, the one for the differing single-particle state. Similarily, the twobody operator reducing to having only $N$ terms, one for each particle-pair including the differening states as one of the particles. When we looked at two noncoincidence, the onebody operator vanished completely and the twodbody operator was left with only a single term---the interaction between the two non-matching states. 

From this development, it is apparent that the matrix elements for both operators vanish when there are three or more noncoincidences. Likewise we would expect the matrix elements of a three-body operator to consist of a single term\footnote{Again we are lumping all exchange terms that might arise into one.} for a three noncoincidence case and vanish for four or more noncoincidences.

\section*{Excercise 2}
The density of particles with coordinates $\bt{x}$ is given by
$$n(\bt x) = N \int \big|\Psi_{\rm AS}(\bt{x}, \bt{x}_2, \ldots, \bt{x}_N)\big|^2 \ \d \bt{x}_2\d \bt{x}_3\cdots\bt{x}_N.$$
Writing out the SD for $N=2$ we have
\begin{align*}
n(\bt{x}) &= 2 \int \bigg[\frac{1}{\sqrt{2}}\psi_1(\bt{x}_1)\psi_2(\bt{x}_2) - \psi_1(\bt{x}_2)\psi_2(\bt{x}_1)\bigg]^* \\
&\qquad\qquad\qquad \bigg[\frac{1}{\sqrt{2}}\psi_1(\bt{x}_1)\psi_2(\bt{x}_2) - \psi_1(\bt{x}_2)\psi_2(\bt{x}_1)\bigg] \ \d \bt{x}_2 \\
&= \psi_1^*\psi_1 +  \psi_2^*\psi_2 = |\psi_1|^2 + |\psi_2|^2.
\end{align*}
So we see that the cross terms vanish due to the orthonormality of the single-particle states. We see that when we scale to a larger $N$ this will always be the case. We then have the general result
$$n(\bt x) = N \int \big|\Psi_{\rm AS}(\bt{x}, \bt{x}_2, \ldots, \bt{x}_N)\big|^2 \ \d \bt{x}_2\d \bt{x}_3\cdots\bt{x}_N = \sum_k |\psi_k|^2.$$
Which is what we wanted to show. However, when writing this out with the antisymmetrizer operator, the result is different. See the end of this document for a better explanation of this problem.


\subsection*{b)}
We now want to calculate the matrix elements
$$\braopket{\alpha_1 \alpha_2}{\op{F}}{\alpha_1 \alpha_2} \quad \mbox{and} \quad \braopket{\alpha_1 \alpha_2}{\op{G}}{\alpha_1 \alpha_2}.$$
Where $\ket{\alpha_1\alpha_2} = a_{\alpha_1}^\dag a_{\alpha_2}^\dag\ket{0}$. We will do this in second quantization, and use our general results from previous exercises.

\subsection*{Onebody operator}
For the onebody operator, we found that
$$\braopket{SD}{\op{F}}{SD} = \sum_\alpha \braopket{\alpha}{\op{f}}{\alpha},$$
so inserting our Slater determinant gives
$$\braopket{\alpha_1 \alpha_2}{\op{F}}{\alpha_1 \alpha_2} = \braopket{\alpha_1}{\op{f}}{\alpha_1} + \braopket{\alpha_2}{\op{f}}{\alpha_2}.$$

\subsection*{Twobody operator}
For the twobody operator, we found the general result
\begin{align*}
\braopket{SD}{\op{G}}{SD}
=  \frac{1}{2}\sum_{\alpha,\beta} \bigg(\braopket{\alpha\beta}{\op{g}}{\alpha\beta} - \braopket{\alpha\beta}{\op{g}}{\beta\alpha}\bigg).
\end{align*}
So in this case, we have
\begin{align*}	
\braopket{\alpha_1 \alpha_2}{\op{G}}{\alpha_1 \alpha_2} = \frac{1}{2}\bigg(
&\braopket{\alpha_1\alpha_2}{\op{g}}{\alpha_1\alpha_2} -\braopket{\alpha_1\alpha_2}{\op{g}}{\alpha_2\alpha_1} \\
&\qquad+ \braopket{\alpha_2\alpha_1}{\op{g}}{\alpha_2\alpha_1} - \braopket{\alpha_2\alpha_1}{\op{g}}{\alpha_1\alpha_2} \bigg).
\end{align*}
Using the fact that generally $\braopket{\alpha\beta}{\op{q}}{\gamma\delta} = \braopket{\beta\alpha}{\op{q}}{\delta\gamma}$, we can rewrite this to 
$$\braopket{\alpha_1 \alpha_2}{\op{G}}{\alpha_1 \alpha_2} = 
\braopket{\alpha_1\alpha_2}{\op{g}}{\alpha_1\alpha_2} -\braopket{\alpha_1\alpha_2}{\op{g}}{\alpha_2\alpha_1} = \braopket{\alpha_1\alpha_2}{\op{g}}{\alpha_1\alpha_2}_{\rm AS}.$$


