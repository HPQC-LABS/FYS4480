\documentclass[prc]{revtex4}
\usepackage[dvips]{graphicx}
\usepackage{mathrsfs}
\usepackage{amsfonts}
\usepackage{lscape}

\usepackage{epic,eepic}
\usepackage{amsmath}
\usepackage{amssymb}
\usepackage[dvips]{epsfig}
\usepackage[T1]{fontenc}
\usepackage{hyperref}
\usepackage{bezier}
\usepackage{pstricks}
\usepackage{dcolumn}% Align table columns on decimal point
\usepackage{bm}% bold math
%\usepackage{braket}
\usepackage[dvips]{graphicx}
\usepackage{pst-plot}

\newcommand{\One}{\hat{\mathbf{1}}}
\newcommand{\eff}{\text{eff}}
\newcommand{\Heff}{\hat{H}_\text{eff}}
\newcommand{\Veff}{\hat{V}_\text{eff}}
\newcommand{\braket}[1]{\langle#1\rangle}
\newcommand{\Span}{\operatorname{sp}}
\newcommand{\tr}{\operatorname{trace}}
\newcommand{\diag}{\operatorname{diag}}
\newcommand{\bra}[1]{\left\langle #1 \right|}
\newcommand{\ket}[1]{\left| #1 \right\rangle}
\newcommand{\element}[3]
    {\bra{#1}#2\ket{#3}}

\newcommand{\normord}[1]{
    \left\{#1\right\}
}

\usepackage{amsmath}
\begin{document}
\title{Exercises FYS4480, weeks 40 and 41, October 3-14, 2022}

%\author{}
\maketitle

\subsection*{Exercise 1}

We will continue to study the schematic model (the Lipkin model, Nucl.
Phys. {\bf 62} (1965) 188) for the interaction among  $4$
fermions that can occupy two different energy levels. Each levels has degeneration $d=4$. The two levels have quantum numbers $\sigma=\pm 1$,
with the upper level having  $\sigma=+1$ and energy
$\varepsilon_{1}=
\varepsilon/2$. The lower level  has $\sigma=-1$ and energy
$\varepsilon_{2}=-\varepsilon/2$. 
In addition, the substates  of each level are characterized  
by the quantum numbers $p=1,2,3,4$.

We define the single-particle states
\[
\ket{u_{\sigma =-1,p}}=a_{-p}^{\dagger}\ket{0}
\hspace{1cm}
\ket{u_{\sigma =1,p}}=a_{+p}^{\dagger}\ket{0}.
\]
The single-particle states span an orthonormal basis.
The Hamiltonian of the system is given by
\[
\begin{array}{ll}
H=&H_{0}+H_{1}+H_{2}\\
&\\
H_{0}=&\frac{1}{2}\varepsilon\sum_{\sigma ,p}\sigma
a_{\sigma,p}^{\dagger}a_{\sigma ,p}\\
&\\
H_{1}=&\frac{1}{2}V\sum_{\sigma ,p,p'}
a_{\sigma,p}^{\dagger}a_{\sigma ,p'}^{\dagger}
a_{-\sigma ,p'}a_{-\sigma ,p}\\
&\\
H_{2}=&\frac{1}{2}W\sum_{\sigma ,p,p'}
a_{\sigma,p}^{\dagger}a_{-\sigma ,p'}^{\dagger}
a_{\sigma ,p'}a_{-\sigma ,p}\\
&\\
\end{array}
\]
where $V$ and $W$ are constants. The operator 
$H_{1}$ can move pairs of fermions as shown earlier
while $H_{2}$ is a spin-exchange term.
The
$H_{2}$ term moves a pair of fermions from a state $(p\sigma ,p' -\sigma)$ to a state
$(p-\sigma ,p'\sigma)$.
\begin{enumerate}
\item[a)] Use the quasispin operators to construct the Hamiltonian matrix 
  $H$ for the five-dimensional space that has total spin $J=2$ and spoin projections $J_z=-2,-1,0,1,2$.
Find thereafter the eigenvalues for the following parameter sets:
\[
\begin{array}{cccc}
(1)&\varepsilon=2,&V=-1/3,&W=-1/4\\
(2)&\varepsilon=2,&V=-4/3,&W=-1
\end{array}
\]
Which state is the ground state? Comment your results in terms of the coefficients of the various eigenfunctions
\item[b)]
The single-particle states for the Lipkin model
\[
\ket{u_{\sigma =-1,p}}=a_{-p}^{\dagger}\ket{0}
\hspace{1cm}
\ket{u_{\sigma =1,p}}=a_{+p}^{\dagger}\ket{0}
\]
can now be used as basis for a new single-particle state
$\ket{\phi_{\alpha ,p}}$  via a unitary  transformation
\[
\ket{\phi_{\alpha ,p}}=
\sum_{\sigma =\pm1}C_{\alpha\sigma}\ket{u_{\sigma ,p}}
\]
with $\alpha=\pm 1$ og $p=1,2,3,4$. Why is $p$ the same in 
$\ket{\phi}$
as in $\ket{u}$?  Show that the new basis is orthonormal.
\item [c)] With the new basis we can construct a new Slater determinant given by
$\ket{\Psi}$
\[
\ket{\Psi}=\prod_{p=1}^{4}b_{\alpha ,p}^{\dagger}\ket{0}
\]
with $b_{\alpha ,p}^{\dagger}\ket{0}=\ket{\phi_{\alpha ,p}}$.
h)   Use the Slater determinanten from the previous exercise to calculate
\[
E=\bra{\Psi}H\ket{\Psi},
\]
as a function of the coefficients $C_{\sigma\alpha}$. We assume the coefficients to be real.
\item[d)] Show that
\[
  \frac{\epsilon}{3} > V+W,
\]
has to be fulfilled in order to find a minimum in the energy.

Hint: calculate the functional derivative  of the energy with respect to the coefficients $C_{\sigma\alpha}$. 
\end{enumerate}



\end{document}

