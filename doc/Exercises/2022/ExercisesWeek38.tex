\documentclass[prc]{revtex4}
\usepackage[dvips]{graphicx}
\usepackage{mathrsfs}
\usepackage{amsfonts}
\usepackage{lscape}

\usepackage{epic,eepic}
\usepackage{amsmath}
\usepackage{amssymb}
\usepackage[dvips]{epsfig}
\usepackage[T1]{fontenc}
\usepackage{hyperref}
\usepackage{bezier}
\usepackage{pstricks}
\usepackage{dcolumn}% Align table columns on decimal point
\usepackage{bm}% bold math
%\usepackage{braket}
\usepackage[dvips]{graphicx}
\usepackage{pst-plot}

\newcommand{\One}{\hat{\mathbf{1}}}
\newcommand{\eff}{\text{eff}}
\newcommand{\Heff}{\hat{H}_\text{eff}}
\newcommand{\Veff}{\hat{V}_\text{eff}}
\newcommand{\braket}[1]{\langle#1\rangle}
\newcommand{\Span}{\operatorname{sp}}
\newcommand{\tr}{\operatorname{trace}}
\newcommand{\diag}{\operatorname{diag}}
\newcommand{\bra}[1]{\left\langle #1 \right|}
\newcommand{\ket}[1]{\left| #1 \right\rangle}
\newcommand{\element}[3]
    {\bra{#1}#2\ket{#3}}

\newcommand{\normord}[1]{
    \left\{#1\right\}
}

\usepackage{amsmath}
\begin{document}

\title{Exercises FYS4480, week 38, September 19-23, 2022}
%\author{}
\maketitle
Feel free to continue working on the Lipkin model from last week. See also the additional challenge to last week under exercise 3 here.

\subsection*{Exercise 1}
We define the one-particle operator
\[
\hat{T}={\displaystyle
\sum_{\alpha\beta}}\bra{\alpha}t\ket{\beta}a_{\alpha}^
{\dagger}a_{\beta},
\]
and the two-particle operator
\[
\hat{V}=
\frac{1}{2}{\displaystyle
\sum_{\alpha\beta\gamma\delta}}\bra{\alpha\beta}
v\ket{\gamma\delta}a_{\alpha}^{\dagger}a_{\beta}^{\dagger}
a_{\delta}a_{\gamma}.
\]
We have defined a single-particle basis with quantum numbers given by the set of greek letters $\alpha,\beta,\gamma,\dots$

\begin{enumerate}
\item[a)] Show that the form of these operators remain unchanged under 
a transformation  of the single-particle basis given by 
\[
\ket{i}=\sum_{\lambda}\ket{\lambda}\left\langle \lambda | i \right\rangle,
\]
with $\lambda\in \left\{\alpha,\beta,\gamma,\dots\right\}$. 
Show also that
$a_{i}^{\dagger}a_{i}$ is the number operator
for  the orbital $\ket{i}$. 
\item[b)] Find also the expressions for the operators
$T$ and $V$ when $T$ is diagonal in the representation
$i$. 
\end{enumerate}

\subsection*{Exercise 2}
Consider the Hamilton operator for a harmonic oscillator
($c=\hbar =1$)
\[
\hat{H}=\frac{1}{2m}p^{2}+\frac{1}{2}kx^{2},
\hspace{1cm}k=m\omega^{2}
\]
\begin{enumerate}
\item[a)] Define the operators
\[
a^{\dagger}=\frac{1}{\sqrt{2m\omega}}
(p+im\omega x),\hspace{1cm}a=\frac{1}{\sqrt{2m\omega}}
(p-im\omega x)
\]
and find the commutation relations for these operators by using the
corresponding relations for $p$ and $x$.
\item[b)] Show that
\[
H=\omega (a^{\dagger}a+\frac{1}{2})
\]
\item[c)] Show that if for a state $\ket{0}$ which satisfies
$\hat{H}\ket{0}=\frac{1}{2}\omega\ket{0}$, then we have
\[
\hat{H}\ket{n}=\hat{H}(a^{\dagger})^{n}\ket{0}=(n+\frac{1}{2})\omega\ket{n}
\]
\item[d)] Show that the state $\ket{0}$ from c), with the property
$a\ket{0}=0$, must exist.
\end{enumerate}

\subsection*{Exercise 3, Challenge}

In the previous exercise set from week 37 we considered a state with all fermions in the lowest
single-particle state
\[
\ket{\Phi_{J_z=-2}} =a_{1-}^{\dagger}a_{2-}^{\dagger}
a_{3-}^{\dagger}a_{4-}^{\dagger}\ket{0}.
\]

This state has $J_{z}=-2$ and belongs to the set of projections for
$J=2$.  We will use the shorthand notation $\ket{J,J_z}$ for states
with different spon $J$ and spin projection $J_z$.  The other possible
states have $J_{z}=-1$, $J_{z}=0$, $J_{z}=1$ and $J_{z}=2$.

Use the raising or lowering operators $J_{+}$ and $J_{-}$  in order to construct the 
states for spin $J_{z}=-1$ $J_{z}=0$, $J_{z}=1$
and $J_{z}=2$.
The action of these two operators on a given state with spin $J$ and projection $J_z$
is given by ($\hbar = 1$) by
$J_+\ket{J,J_z}=\sqrt{J(J+1)-J_z(J_z+1)}\ket{J,J_z+1}$ and
$J_-\ket{J,J_z}=\sqrt{J(J+1)-J_z(J_z-1)}\ket{J,J_z-1}$.


\end{document}




