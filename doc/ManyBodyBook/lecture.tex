%%%%%%%%%%%%%%%%%%%% book.tex %%%%%%%%%%%%%%%%%%%%%%%%%%%%%
%
% sample root file for the chapters of your "monograph"
%
% Use this file as a template for your own input.
%
%%%%%%%%%%%%%%%% Springer-Verlag %%%%%%%%%%%%%%%%%%%%%%%%%%


% RECOMMENDED %%%%%%%%%%%%%%%%%%%%%%%%%%%%%%%%%%%%%%%%%%%%%%%%%%%
\documentclass[graybox,envcountchap,sectrefs]{svmono}

% choose options for [] as required from the list
% in the Reference Guide

\usepackage{mathptmx}
\usepackage{helvet}
\usepackage{courier}
%
\usepackage{type1cm}         

\usepackage{makeidx}         % allows index generation
\usepackage{graphicx}        % standard LaTeX graphics tool
                             % when including figure files
\usepackage{multicol}        % used for the two-column index
\usepackage[bottom]{footmisc}% places footnotes at page bottom

% see the list of further useful packages
% in the Reference Guide
 \usepackage{chapterbib}
 \usepackage{verbatim}

\usepackage[usenames,dvipsnames,x11names]{xcolor}
\usepackage{tikz}
\usetikzlibrary{arrows,snakes,shapes}

 \usepackage{listings}
 \usepackage{epic}
 \usepackage{eepic}
 \usepackage{a4wide}
 \usepackage{color}
 \usepackage{amsmath}
 \usepackage{amssymb}
 \usepackage[dvips]{epsfig}
 \usepackage[T1]{fontenc}
 \usepackage{cite} % [2,3,4] --> [2--4]
 \usepackage{shadow}
 \usepackage{hyperref}
 \usepackage{bezier}
 \usepackage{pstricks}
 %\usepackage{refcheck}
 \setcounter{tocdepth}{2}
%\usepackage{gnuplot-lua-tikz}

\makeindex             % used for the subject index
                       % please use the style svind.ist with
                       % your makeindex program

%%%%%%%%%%%%%%%%%%%%%%%%%%%%%%%%%%%%%%%%%%%%%%%%%%%%%%%%%%%%%%%%%%%%%

\begin{document}

\author{Author name(s)}
\title{Book title}
\subtitle{-- Monograph --}
\maketitle

\frontmatter%%%%%%%%%%%%%%%%%%%%%%%%%%%%%%%%%%%%%%%%%%%%%%%%%%%%%%


%%%%%%%%%%%%%%%%%%%%%%% dedic.tex %%%%%%%%%%%%%%%%%%%%%%%%%%%%%%%%%
%
% sample dedication
%
% Use this file as a template for your own input.
%
%%%%%%%%%%%%%%%%%%%%%%%% Springer %%%%%%%%%%%%%%%%%%%%%%%%%%

\begin{dedication}
Use the template \emph{dedic.tex} together with the Springer document class SVMono for monograph-type books or SVMult for contributed volumes to style a quotation or a dedication\index{dedication} at the very beginning of your book in the Springer layout
\end{dedication}





%%%%%%%%%%%%%%%%%%%%%%foreword.tex%%%%%%%%%%%%%%%%%%%%%%%%%%%%%%%%%
% sample foreword
%
% Use this file as a template for your own input.
%
%%%%%%%%%%%%%%%%%%%%%%%% Springer %%%%%%%%%%%%%%%%%%%%%%%%%%

\foreword

%% Please have the foreword written here
Use the template \textit{foreword.tex} together with the Springer document class SVMono (monograph-type books) or SVMult (edited books) to style your foreword\index{foreword} in the Springer layout. 

The foreword covers introductory remarks preceding the text of a book that are written by a \textit{person other than the author or editor} of the book. If applicable, the foreword precedes the preface which is written by the author or editor of the book.


\vspace{\baselineskip}
\begin{flushright}\noindent
Place, month year\hfill {\it Firstname  Surname}\\
\end{flushright}




Nearly all of physics is many-body physics at the most  
microscopic level of understanding.
A theoretical understanding of the behavior of quantum mechanical systems with many interacting particles,
accompanied with experiments and simulations,   
is a  great challenge and provides fundamental insights into systems governed by quantum mechanical principles.

Many-body physics  applications range from condensed matter physics
(for example antiferromagnets, quantum liquids and solids, plasmas, metals,
superconductors), nuclear and high energy physics (for example nuclear matter, finite
nuclei, quark-gluon plasmas), dense matter astrophysics (for example  neutron
stars, white dwarfs), atoms and molecules (for example electron correlations,
reactions, quantum chemistry), and elementary particles
(for example quantum field theory, lattice gauge field models). 

Microscopic many-body methodologies have attained a high degree of
sophistication in the last decades, from both theoretical and
applicative  points of view. These developments have allowed
researchers to adapt these technologies to a much wider range of physical
systems, and to obtain quantitative descriptions of many 
observables. 

Several of these methods have been developed independently of each other.
In some  cases the same methods have been applied and studied in different
fields of research with little overlap and exchange of knowledge. 
A classic example is the development of coupled-cluster approaches,
which originated from the nuclear many-body problem in the late fifties ({\bf give refs here later}). 
These methods were adapted and refined 
to extremely high precision and predicability by quantum chemists over 
the next four decades. The developments comprised, and still do,  
both methodological innovations and the usage of advances in high-performance computing.    
As time elapsed,   
the overlap with the original nuclear many-body problem vanished gradually, 
dwindling almost to a non-vanishing exchange of knowledge and methodologies. 
These methods were however revived in nuclear physics towards the turn of the last century and offer now a viable approach
to {\em ab initio}\footnote{We would like to come with a warning here, since it is our feeling that the 
labelling {\em ab initio} attached to many titles of scientific articles nowadays, has reached an inflationary stage. 
With {\em ab initio} we will mean solving Schr\"odinger's or Dirac's many-particle equations starting with the presumably best 
inter-particle Hamiltonian, making no approximations. The particles that make up the many-body problem will be approximated 
as the constituent ones. 
In the nuclear many-body problem
these are various baryons such as protons, neutrons, isobars and hyperons and low-mass mesons. All many-body approaches discussed in this text involve
some approximations, and only few of them, if any,  can reach the glorious stage of being truely {\em ab initio}.
Truely {\em ab initio} in nuclear physics means to solve the many-body problem starting with quarks and gluons.}
nuclear structure  studies of nuclei,   spanning the whole range of stable closed-shell and several open-shell nuclei. 


Many of the developments are also not 
always readily accessible to a larger community of
researchers and especially to the young and uninitiated ones. A notorious exception is 
again the quantum chemistry community, where highly effecient codes and methods are made available.  
Numerical packages like
Gamess, Dalton and Gaussian ({\bf give refs here later}) provide a reliable computational 
platform for many  chemical applications and studies.
In other fields this is not always the case.     

Nuclear physics, where we have our basic research experience, is such a field.
This book aims therefore at giving you an overview 
of widely used quantum mechanical  many-body methods. We want also to provide you with computational tools in order to enable you
to perform studies of realistic systems  in nuclear physics. 
The methods we discuss are however not limited to applications
in nuclear physics, rather, with  minor  changes one can apply them to other fields, from quantum chemistry, via studies of atoms and
molecules to materials science and solid state devices such as quantum dots.  

Typical methods we will discuss are 
configuration interaction and large-scale diagonalization.methods (shell-model in nuclear physics), perturbative many-body methods, 
coupled cluster theory, Green's function approaches, various Monte Carlo methods, extensions to weakly  bound systems and links
from {\em ab initio} methods to  density functional and mean field theories. 
The methods we have chosen are those which enjoy a great degree of popularity in low and intermediate energy nuclear physics, 
but due to both our limited knowledge
and due to space considerations, some of these will be treated in rather cavalier way.

Focussing on methods and tools only is however not necessarily a recipe for success, as we all know a 
a bad workman quarrels with his tools. Our aim is to provide 
you with a didactical description
of some of these  powerful methods and tools with the purpose of allowing you to
gain further insights and a possible working experience in modern many-body methods.
There are several textbooks on the market which cover specific 
many-body methods and/or applications, but almost none of them
presents an updated and comparative description of various many-body methods. The reason for choosing such an approach
is that we want you, the reader, to gain the experience and form your own opinion on the suitability and applicability
of specific methods. Too often you may have met many-body practitioners
claiming 'my method is better than your method because...'.  We wish to avoid such a pitfal, although ours, as any other approach,
is tainted by our biases, preferences and obviously ignorance about all possible details. 

We also  want  to shed light, in a hopefully unbiased way, on possible interplays and
differences between the various many-body methods with a focus
on the physics content. 
We have therefore selected several physical systems,
from simple models to realistic cases,  whose properties can be
described within the theoretical frameworks presented here.
In this text we will expose you to problems which tour many nuclear physics applications, 
from the basic forces which bind nucleons together to compact objects such as neutron stars.  
Our hope is that the material we have provided   allows an eventual reader to  assess 
by himself/herself the pro and cons of the methods discussed.

Moreover,  the text can be used and read as a large project. Each chapter ends with a specific project. The code you end up developing 
for that particular project, can in turn be  reused in the next chapter and its pertaining project.  To give you an example,
in chapter xx you will end up constructing your own nucleon-nucleon interaction and solve its Schr\"odinger equation using the Lippman-Schwinger
equation.  This solution results in the so-called $T$-matrix  which in turn can be related to the experimental phase shifts.
The Lippman-Schwinger equation can easily be modified to account for a given nuclear medium, resulting in the 
so-called $G$-matrix.  In chapter xx you end up computing such an effective interaction, using similarity transformations as well.
With these codes, you are in turn in the position where you can define effective interactions for say coupled-cluster calculations
(the topic of chapter xx) or shell-model effective interactions (topics for chapter xx and xx).

Our hope is that this will enable  you to assess at a much deeper level the pros and cons of the various methods.  
We believe firmly that knowledge of the strengths and weaknesses of a given method allows you to realize where
improvements can be made.   The text has also a strong focus on computational issues, from how to build up large many-body codes 
to parallelization and high-performance computing topics.



This texts spans some four hundred pages, and with the promised wide scope we aim at, 
there have to be topics which will not be dealt with properly.
In particular we will not cover  reaction theories.  
For reaction theories we provide all the inputs needed to compute
onebody densities, spectroscopic factors and optical potentials. 
For density functional theories we discuss how one can constrain the exchange correlations based on {\em ab initio} methods.
Furthermore, when it comes to the underlying nuclear forces, we will assume that various baryons and mesons are the 
effective degrees of freedom, limiting ourselves to low and intermediate energy nuclear physics.  A discussion of recent 
advances in lattice quantumchromodynamics (QCD) and its link to effective field theories is also outside the scope
of this book, although we will refer to advances in these fields as well and link the construction of nuclear forces to
undergoing research in effective field theory and lattice QCD.  
The material on infinite matter can be extended to finite temperatures, but we do not 
adventure into the realm of heavy-ion collisions and the search of the holy grail of quark-gluon plasma. 
We will throughout pay loyalty to a physical world governed by the degrees of freedom of baryons and mesons.
We apologize for these shortcomings, which are mainly due to
our lack of detailed research knowledge in the above fields. To be a jack of all trades leads normally to mastering none.

A final warning however.
This text is not a text on many-body theories, rather it is
a text on applications and implementations of many-body theories. This applies also to many of the algoritms  we discuss.  We do not go into
details about for example Gaussian quadrature or the mathematical foundations of random walks and the Metropolis algorithm.  We will often simply state results, but add 
enough references to tutorial texts so that you can look up the missing wonders of numerical mathematics yourself.
Similarly,  handling of angular momenta and their recouplings via $3j$, $6j$ or $9j$ symbols should  not  come as a bolt out from the blue. But again, don't despair,
we'll guide you safely to the appropriate literature.

We have therefore taken the liberty to have  certain expectations about you, the potential reader.  
We assume that you are at least a graduate student who is embarking on studies in 
nuclear physics and that
you have some familiarity with basic nuclear physics, angular momentum theory and many-body theory, 
typically at the level of texts like those of Talmi, Fetter and Walecka, Blaizot and Ripka, Dickhoff and Van Neck or similar monographs (add refs later). 
We will obviously state and repeat the basic rules and theorems, but will not go into derivations.  
Some of the derivations will however be left to you via various exercises interspersed througout the text.


\subsection*{Acknowledgements}
Friends/colleagues bla bla  and sponsors bla bla. also add that errors, misunderstandings etc are mainly due to ourselves!!







 







%%%%%%%%%%%%%%%%%%%%%%acknow.tex%%%%%%%%%%%%%%%%%%%%%%%%%%%%%%%%%%%%%%%%%
% sample acknowledgement chapter
%
% Use this file as a template for your own input.
%
%%%%%%%%%%%%%%%%%%%%%%%% Springer %%%%%%%%%%%%%%%%%%%%%%%%%%

\extrachap{Acknowledgements}

Use the template \emph{acknow.tex} together with the Springer document class SVMono (monograph-type books) or SVMult (edited books) if you prefer to set your acknowledgement section as a separate chapter instead of including it as last part of your preface.



\tableofcontents

%%%%%%%%%%%%%%%%%%%%%%acronym.tex%%%%%%%%%%%%%%%%%%%%%%%%%%%%%%%%%%%%%%%%%
% sample list of acronyms
%
% Use this file as a template for your own input.
%
%%%%%%%%%%%%%%%%%%%%%%%% Springer %%%%%%%%%%%%%%%%%%%%%%%%%%

\extrachap{Acronyms}

Use the template \emph{acronym.tex} together with the Springer document class SVMono (monograph-type books) or SVMult (edited books) to style your list(s) of abbreviations or symbols in the Springer layout.

Lists of abbreviations\index{acronyms, list of}, symbols\index{symbols, list of} and the like are easily formatted with the help of the Springer-enhanced \verb|description| environment.

\begin{description}[CABR]
\item[ABC]{Spelled-out abbreviation and definition}
\item[BABI]{Spelled-out abbreviation and definition}
\item[CABR]{Spelled-out abbreviation and definition}
\end{description}


\mainmatter%%%%%%%%%%%%%%%%%%%%%%%%%%%%%%%%%%%%%%%%%%%%%%%%%%%%%%%
%%%%%%%%%%%%%%%%%%%%%part.tex%%%%%%%%%%%%%%%%%%%%%%%%%%%%%%%%%%
% 
% sample part title
%
% Use this file as a template for your own input.
%
%%%%%%%%%%%%%%%%%%%%%%%% Springer %%%%%%%%%%%%%%%%%%%%%%%%%%

\begin{partbacktext}
\part{Introduction to programming and numerical methods}
\noindent Use the template \emph{part.tex} together with the Springer document class SVMono (monograph-type books) or SVMult (edited books) to style your part title page and, if desired, a short introductory text (maximum one page) on its verso page in the Springer layout.

\end{partbacktext}

\chapter{Introduction}\label{chap:introduction}


\section{Why a text on the nuclear many-body problem?}\label{sec:introduction_why}


Nuclear physics has recently experienced several discoveries and
technological advances that address the fundamental questions of the
field.  Many of these advances have been made possible by significant
investments in frontier research facilities worldwide over the last
two decades. Some of these discoveries are the detection of perhaps
the most exotic state of matter, the quark-gluon plasma, which is
believed to have existed in the very first moments of the Universe
(refs).  Recent experiments have validated the standard solar model
and established that neutrinos have mass (refs). High-precision
measurements of the quark structure of the nucleon are challenging
existing theoretical understanding.  Nuclear physicists have started
to explore a completely unknown landscape of nuclei with extreme
proton to neutron ratios using radioactive and short-lived ions,
including rare and very neutron-rich isotopes.  These experiments push
us towards the extremes of nuclear stability.  Moreover, these rare
nuclei lie at the heart of nucleosynthesis processes in the universe
and are therefore an important component in the puzzle of matter
generation in the universe.

A firm experimental and theoretical understanding of nuclear stability
in terms of the basic constituents is a huge intellectual endavour.
Experiments indicate that developing a comprehensive description of
all nuclei and their reactions requires theoretical and experimental
investigations of rare isotopes with unusual neutron-to-proton ratios
that are very different from their stable counterparts.  These rare
nuclei are difficult to produce and study experimentally since they
can have extremely short lifetimes. Theoretical approaches to these
nuclei involve solving the nuclear quantum many-body problem.

It is within this framework the present texts finds its rationale.  We
want to give you the tools to tackle the nuclear many-body problem,
starting from a Hamiltonian which describes the interaction among the
relevant baryons and mesons, with protons and neutrons (nucleons) as
the leading dramatis personae.  The nuclear many-body problem spans
nuclei from A=2 (the deuteron) to the superheavy region, and different
theoretical techniques are pursued in different regions of the chart
of nuclei.  For many interacting nucleons, Schr\"odinger's equation is
an integro-differential equation whose complexity increases
exponentially with increasing numbers of nucleons and states that the
system can access.  Unfortunately, there are only almost no problems
which can be solved on a closed form, and virtually exact numerical
solutions are presently available only for systems with up to three or
four nucleons via the Faddeev~\cite{Faddeev} and
Faddeev-Yakubowski~\cite{FY} methods (refs here), respectively. For
problems involving more nucleons, one needs reliable numerical
many-body methods. Such methods should allow for controlled
approximations and provide a computational scheme which accounts for
successive many-body corrections in a systematic way.

This books deals with several of these many-body methods, with an
emphasis on the development of stable algorithms and codes for
handling many-nucleon systems.  Typical examples of popular many-body
methods are the family of coupled-cluster methods
\cite{bartlett2007,dean2004,helgaker,hagen2007a,hagen2007b,hagen2007c}
discussed in chapter \ref{chap:coupledcluster}, various types of Monte
Carlo methods \cite{Pudliner1997,kdl97,ceperley1995,utsuno1999}
discussed in chapters \ref{chap:vmc}, \ref{chap:dmc} and
\ref{chap:smmc}, perturbative expansions \cite{ellis1977,mhj95} and
Green's function methods \cite{barbieri2004} discussed in chapter
\ref{chap:mbpt}, and large-scale diagonalization methods
\cite{Whitehead1977,caurier2005,dean2004b,navratil2004,horoi2006} in
chapter \ref{chap:shellmodel}.


In the chapters on coupled cluster theories and shell-model
calculations we discuss also how to extract quantities like
spectroscopic factors and onebody densities. These quantities provide
in turn the many-body input to reaction theories, which are not
addressed in this text. (add refs about reaction theories).
Density-functional theory~\cite{bartlett2005,peirs2003,vanneck2006} is
discussed in chapter \ref{chap:links}, with an emphasis on how to
constrain the exchange part from the available information from
various many-body methods.  As stated in the preface, there are other
many-body methods used in nuclear physics such as the correlation
operator methods \cite{japanese,feldmeier}, the unitary-model-operator
approach of Suzuki and co-workers \cite{Fujii2004,Fujii2004b} and
Fermi hypernetted chain theory \cite{adelchi,co2007}.  Another very
interesting many-body is the density-matrix renormalization group, see
for example
Refs.~\cite{white1992,schollwock2005,dukelsky2002a,pittel2006}.  We
will briefly mention the latter and its applications to nuclear
physics.


In addition to the focus on many-body methods, this text emphasizes
various numerical algorithms and high-performance computing (HPC)
topics such as efficient parallelization of codes.  We will use a
broad range of mathematical methods, from linear algebra problems to
Monte Carlo simulations.  The text gives a survey over some of the
most used methods in computational nuclear physics and each chapter
ends with one or more applications to realistic systems, such
renormalzing the nucleon-nucleon interaction or performing shell-model
calculations for selected nuclei.  Among the algorithms we discuss,
are some of the top algorithms in computational science.  Some of the
algorithms we focus on, taken from the recent surveys by Dongarra and
Sullivan \cite{top101} and Cipra \cite{top102}, are
\begin{enumerate}
\item The Monte Carlo method or Metropolis algorithm, devised by John
  von Neumann, Stanislaw Ulam, and Nicholas Metropolis, discussed in
  chapters \ref{chap:vmc}-\ref{chap:smmc}.
\item Krylov Subspace Iteration method for large eigenvalue problems
  in particular, developed by Magnus Hestenes, Eduard Stiefel, and
  Cornelius Lanczos, discussed and used in chapter
  \ref{chap:shellmodel}.
\item The Householder matrix decomposition, developed by Alston
  Householder and used in chapter \ref{chap:shellmodel}.
\item The Fortran compiler, developed by a team lead by John Backus,
  codes used throughout this text.
\item The QR algorithm for eigenvalue calculation, developed by Joe
  Francis, used in chapter \ref{chap:shellmodel}.
\end{enumerate}
As stated in the preface, we will not use much space on the
derivations of these algorithms, our focus being the usage of these in
a specific nuclear physics context.


The next section attempts at guiding you through the plethora of
many-body methods and algorithms we have chosen.


\section{How to use this text}\label{sec:introduction_howto}

We assume that you, as a potential reader, have a background in
many-body theories and nuclear physics corresponding to a beginnning
graduate level. This would typically correspond to your fourth or
fifth year of study, undergraduate studies included. In many countries
this would match the penultimate year of a masters degree in physics
or the first year of your graduate studies.  A knowledge of angular
momentum techniques at a pedestrian level is also expected, in
addition to some programming skills. The latter is now standard in
essentially all schools and colleges in the mathematics and natural
sciences.  We assume that you have taken an introductory course in
programming and have some familiarity with high-level and modern
languages such as Java, C++, Fortran and python. Fortran\footnote{With
  Fortran we will consistently mean the latest standard, Fortran 2008.
  There are no programming examples in Fortran 77 in this text.}  and
C++ are examples of compiled high-level languages, in contrast to
interpreted ones like Maple or Matlab.  This text uses Fortran and C++
as programming languages, whilst python is used for scripting of
computational projects.


Every chapter ends with a large project, which can be reused again in
most of the following chapters.  For every chapter we have link to our
webpage with auxiliary functions and software which you may need.  To
give you an example, chapter \ref{chap:forces} ends with a project
that solves the two-body scattering equation.  From a computational
point of view you will need to discretize an integral and invert a
matrix.  Functions for setting up the grid of integration points using
Gaussian quadrature and inverting a matrix using the standard
L(ower)U(pper)-decomposition of a matrix are provided in both Fortran
and C++.  These codes are available at our webpage (add links to
chapters).  You will use this code in the subsequent chapter where you
construct a medium-dependent renormalized interaction, which can in
turn be used in the shell-model calculations of chapter
\ref{chap:shellmodel}, the coupled-cluster chapter
\ref{chap:coupledcluster} or the auxiliary field Monte Carlo approach
in chapter \ref{chap:smmc}.

In the shell-model chapter \ref{chap:shellmodel} you will end up
constructing your own shell-model code.  Furthermore, you will also
learn to parallelize the shell-model code on both a shared memory and
a distributed memory supercomputer. The shell-model code is difficult
to parallelize and we would suggest that you wait with this till you
have gained some experience from simpler cases like the variational
Monte Carlo calculations of chapter \ref{chap:vmc}.
 
The code for the nucleon-nucleon interaction you will develop in
chapter \ref{chap:forces} can also be used in the chapters on
variational and diffusion Monte Carlo calculations, see chapters
\ref{chap:vmc} and \ref{chap:dmc}.  In the Monte Carlo chapters,
parallelization is slightly simpler compared to the shell-model code.
The variational Monte Carlo calculations are particularly simple.
Parallelization is also a topic of chapter \ref{chap:coupledcluster}
and chapter \ref{chap:smmc} on the auxiliary field Monte Carlo
calculations.


If your main interest is in shell-model calculations, then we would
recommend that you go through chapters \ref{chap:forces},
\ref{chap:renorm}, \ref{chap:mbpt} and \ref{chap:shellmodel} and
possibly the chapter on auxiliary field Monte Carlo if you are
studying systems with very large model spaces.  If you are interested
in level densities and finite temperature properties of nuclei the
latter is an important chapter.  If you also wish to understand how to
use coupled-cluster theory for deriving effective interactions for the
nuclear shell, you should also read chapter \ref{chap:coupledcluster}.

If you are only interested in variational and diffusion Monte Carlo
studies, then chapters \ref{chap:forces}, \ref{chap:vmc} and
\ref{chap:dmc} are the essential ones.


However, for many of you who do not have the patience or time to go
through all the nitty gritty small details needed to end up with a
final code, (and reach the illustrious and rewarding final stages of
many-body physics), there is hope. We provide complete codes with many
additional features for most of the projects listed in this book.  You
are free to use these codes or just use them as guidelines when you
write your own. These codes contain many auxiliary functions which can
save you quite some time in your own developments.  Some of these
functions, like functions for angular momentum recoupling, are well
tested and you can safely use them as long you understand their
limitations.  There is no point in you wasting time developing your
own codes for say $9j$ symbols or Moshinsky transformation
coefficients (give ref to moshinsky).





\section{Choice of programming languages and computational issues}\label{sec:introduction_language}

As programming language we have ended up with C++ and Fortran.  The
auxiliary library functions are written in both languages.
 
{\bf notes} say more about this later, when text is more complete.
Add about scripting in high-performance computing and notes on python
later

\chapter{Observables and simple potential models}\label{chap:basics}


\section{Introduction}\label{sec:basics_intro}


\section{Masses and radii}
A basic quantity which can be measured for the ground states of nuclei is the atomic mass
$M(N, Z)$ of the neutral atom with atomic mass number $A$ and charge $Z$. The number of neutrons are $N$.

Atomic masses are
usually tabulated in terms of the mass excess defined by
\[
\Delta M(N, Z) =  M(N, Z) - uA,
\]
where $u$ is the Atomic Mass Unit 
\[
u = M(^{12}\mathrm{C})/12 = 931.49386 \hspace{0.1cm} \mathrm{MeV}/c^2.
\]
%data from the 2003 compilation of Audi, Wapstra and Thibault.

The nucleon masses are
\[
m_p = 938.27203(8)\hspace{0.1cm} \mathrm{MeV}/c^2 = 1.00727646688(13)u,
\]
and 
\[
m_n = 939.56536(8)\hspace{0.1cm} \mathrm{MeV}/c^2 = 1.0086649156(6)u.
\]
In the 2003 mass evaluation there are 2127 nuclei measured with an accuracy of 0.2
MeV or better, and 101 nuclei measured with an accuracy of greater than 0.2 MeV. For
heavy nuclei one observes several chains of nuclei with a constant $N-Z$ value whose masses
are obtained from the energy released in alpha decay.

Nuclear binding energy is defined as the energy required to break up a given nucleus
into its constituent parts of $N$ neutrons and $Z$ protons. In terms of the atomic masses
$M(N, Z)$ the binding energy is defined by:
\[
BE(N, Z) = ZM_H c^2 + Nm_n c^2 - M(N, Z)c^2 ,
\]
where $M_H$ is the mass of the hydrogen atom and $m_n$ is the mass of the neutron.
In terms
of the mass excess the binding energy is given by:
\[
BE(N, Z) = Z\Delta_H c^2 + N\Delta_n c^2 -\Delta(N, Z)c^2 ,
\]
where $\Delta_H c^2 = 7.2890$ MeV and $\Delta_n c^2 = 8.0713$ MeV.
For the liquid drop model we have
\[ BE(N,Z) = a_1A-a_2A^{2/3}-a_3\frac{Z^2}{A^{1/3}}-a_4\frac{(N-Z)^2}{A}\]
We could also add a so-called pairing term, which is a correction term that
arises from the tendency of proton pairs and neutron pairs to
occur. An even number of particles is more stable than an odd number.
\begin{itemize}
\item $a_1A$: Volume energy. When an assembly of nucleons of the same size is packed
together into the smallest volume, each interior nucleon has a certain
number of other nucleons in contact with it. This contribution is proportional to the volume.
\item $a_2A^{2/3}$:   Surface energy. A nucleon at the
surface of a nucleus interacts with fewer other nucleons than one in
the interior of the nucleus and hence its binding energy is less. This
surface energy term takes that into account and is therefore negative
and is proportional to the surface area.
\item $a_3\frac{Z^2}{A^{1/3}}$: Coulomb Energy. The electric
repulsion between each pair of protons in a nucleus yields less binding. 
\item $a_4\frac{(N-Z)^2}{A}$: Asymmetry energy, associated with the Pauli exclusion principle 
and reflecting the fact that the proton-neutron interaction is more attractive on the average than the neutron-neutron and proton-proton interactions.
\begin{itemize}
\item Red: experimental data, blue: liquid drop model
\item $a_1=15.49$ MeV
\item $a_2=17.23$ MeV
\item $a_3=0.697$ MeV
\item $a_4=22.6$ MeV
\end{itemize}
%	%\includegraphics[width=1.2\textwidth]{beexpliquid.pdf}





We consider energy conservation for nuclear transformations that include, for
example, the fusion of two nuclei $a$ and $b$ into the combined system $c$
\[
{^{N_a+Z_a}}a+ {^{N_b+Z_b}}b\rightarrow {^{N_c+Z_c}}c
\]
or the decay of nucleus $c$ into two other nuclei $a$ and $b$
\[
^{N_c+Z_c}c \rightarrow  ^{N_a+Z_a}a+ ^{N_b+Z_b}b
\]
In general we have the reactions
\[
\sum_i {^{N_i+Z_i}}i \rightarrow  \sum_f {^{N_f+Z_f}}f
\]
We require also that number of protons and neutrons are conserved in the initial stage and final stage, unless we have processes which violate baryon conservation, 
\[
\sum_iN_i = \sum_f N_f \hspace{0.2cm}\mathrm{and} \hspace{0.2cm}\sum_iZ_i = \sum_f Z_f.
\]

This process is characterized by an energy difference called the $Q$ value:
\[
Q=\sum_iM(N_i, Z_i)c^2-\sum_fM(N_f, Z_f)c^2=\sum_iBE(N_f, Z_f)-\sum_iBE(N_i, Z_i)
\]
Spontaneous decay involves a single initial nuclear state and is allowed if $Q > 0$. In the
decay, energy is released in the form of the kinetic energy of the final products. Reactions
involving two initial nuclei and are endothermic (a net loss of energy) if $Q < 0$; the reactions
are exothermic (a net release of energy) if $Q > 0$.

We can consider the Q values associated with the removal of one or two nucleons from
a nucleus. These are conventionally defined in terms of the one-nucleon and two-nucleon
separation energies
\[
S_n= -Q_n= BE(N,Z)-BE(N-1,Z),
\]
\[
S_p= -Q_p= BE(N,Z)-BE(N,Z-1),
\]
\[
S_{2n}= -Q_{2n}= BE(N,Z)-BE(N-2,Z),
\]
and
\[
S_{2p}= -Q_{2p}= BE(N,Z)-BE(N,Z-2),
\]

Using say the neutron separation energies (alternatively the proton separation energies)
\[
S_n= -Q_n= BE(N,Z)-BE(N-1,Z),
\]
we can define the so-called energy gap for neutrons (or protons) as 
\[
\Delta S_n= BE(N,Z)-BE(N-1,Z)-\left(BE(N+1,Z)-BE(N,Z)\right),
\]
or 
\[
\Delta S_n= 2BE(N,Z)-BE(N-1,Z)-BE(N+1,Z).
\]
This quantity can in turn be used to determine which nuclei are magic or not. 
For protons we would have 
\[
\Delta S_p= 2BE(N,Z)-BE(N,Z-1)-BE(N,Z+1).
\]
We can also define the two-neutron or two-proton gap as well. 


\section{Definitions  and single-particle basis functions}

In this text we define an operator as $\hat{O}$. Unless otherwise
specified the number of particles is always $A$ (representing the total number of nucleons) 
and $d$ is the dimension of the 
system. 
In nuclear physics we normally define the total number of particles to be $A=N+Z$,
where $N$ is total number of neutrons and $Z$ the total number of protons. In case of other baryons like isobars $\Delta$ or
various hyperons, one needs to add their definitions.  

The quantum numbers of a single-particle state in coordinate space are
defined by the variable $x=({\bf r},\sigma)$, where ${\bf r}\in {\mathbb{R}}^{d}$with $d=1,2,3$ represents 
the spatial coordinates and $\sigma$ is the eigenspin of the particle. For fermions with eigenspin $1/2$ this means that
\[
 x\in {\mathbb{R}}^{d}\oplus (\frac{1}{2}),
\]
and the integral
\[
\int dx = \sum_{\sigma}\int d^dr = \sum_{\sigma}\int d{\bf r},
\]
and for $A$ nucleons we have
\[
\int d^Ax= \int dx_1\int dx_2\dots\int dx_A.
\]

The quantum mechanical wave function of a 
given state with quantum numbers $\lambda$ (encompassing all quantum numbers needed to specify the system), but ignoring time, is
\[
\Psi_{\lambda}=\Psi_{\lambda}(x_1,x_2,\dots,x_A),
\]
with $x_i=({\bf r}_i,\sigma_i)$ and the projection of $\sigma_i$ takes the values
$\{-1/2,+1/2\}$ for particles with spin $1/2$. 
We will hereafter always refer to $\Psi_{\lambda}$ as the exact wave function, and if the ground state is not degenerate we label it as 
\[
\Psi_0=\Psi_0(x_1,x_2,\dots,x_A).
\]
We will throughout this text use 
upper-case letters are used for many-particle state functions, while lower-case letters will be used for single-particle
state functions.
Since the solution $\Psi_{\lambda}$ seldomly can be found in closed
form, approximations are sought. In this text we define an
approximate wave function or an ansatz to the exact wave function as
\[
\Phi_{\lambda}=\Phi_{\lambda}(x_1,x_2,\dots,x_N),
\]
with 
\[
\Phi_0=\Phi_0(x_1,x_2,\dots,x_N),
\]
being the ansatz to the ground state.  

The state function $\Psi_{\lambda}$ is sought in the Hilbert space of either symmetric or anti-symmetric $A$-body functions, namely
\[
\Psi_{\lambda}\in {\cal H}_N:= {\cal H}_1\oplus{\cal H}_1\oplus\dots\oplus{\cal H}_1,
\]
where the single-particle Hilbert space ${\cal H}_1$ is the space of square integrable functions over
$\in {\mathbb{R}}^{d}\oplus (\sigma)$
resulting in
\[
{\cal H}_1:= L^2(\mathbb{R}^{d}\oplus (\sigma)).
\]

%      \frametitle{Do we understand the physics of dripline systems?}
\begin{itemize}
\item The oxygen isotopes are the heaviest isotopes for
which the drip line is well established.
\item Two out of four
stable even-even isotopes exhibit a doubly magic nature,
namely $^{22}$O ($Z=8$, $N=14$) and $^{24}$O ($Z=8$, $N=16$).
\item 
The structure of $^{22}$O and $^{24}$O is assumed to be governed
by the evolution of the $1s_{1/2}$ and $0d_{5/2}$  one-quasiparticle states.
\item The isotopes
$^{25}$O
$^{26}$O, $^{27}$O and $^{28}$O are outside the drip line, since the $0d_{3/2}$ orbit is not bound.
\end{itemize}

\begin{figure}
	%\includegraphics[width=1.2\textwidth]{snoxygen.pdf}
\end{figure}
\begin{figure}
	%\includegraphics[width=1.2\textwidth]{gapoxygen.pdf}
\end{figure}

\begin{itemize}
\item The Ca  isotopes exhibit several possible closed-shell nuclei $^{40}$Ca, $^{48}$Ca, $^{52}$Ca, $^{54}$Ca,
and  $^{60}$Ca. 
\item  Magic neutron numbers are then $N=20, 28, 32, 34, 40$. 
\item Masses available up to $^{54}$Ca, Gallant {\em et al.},Phys.~Rev.~Lett.~{\bf 109}, 032506 (2012) and K.~Baum {\em et al}, Nature {\bf 498}, 346 (2013).
\item Heaviest observed $^{57,58}$Ca. NSCL experiment,  O.~B.~Tarasov {\it et al.}, Phys.~Rev.~Lett.~{\bf 102}, 142501 (2009). Cross sections for $^{59,60}$Ca assumed small ($< 10^{-12}$mb).
\item Which degrees of freedom prevail close to $^{60}$Ca and beyond?
\end{itemize}

\begin{itemize}
\item {\bf Mass models and mean field models predict the dripline at $A\sim 70$!} Important consequences for modeling of nucleosynthesis related processes.
\item Can we predict reliably which is the last stable calcium isotope? 
\item And how
does this compare with popular mass models on the market? See Nature 486, 509 (2012). 
\item And which parts of the underlying forces
are driving the physics towards the dripline?
\end{itemize}


\begin{figure}
\begin{center}
\setlength{\unitlength}{0.4cm}
\begin{picture}(16,20)
\thicklines
   \put(1,0.5){\makebox(0,0)[bl]{
              \put(0,1){\line(0,1){8}}
              \put(0,9){\line(1,0){8}}
              \put(8,9){\line(0,1){9}}
              \put(8,18){\line(1,0){9}}
              \put(17,1){\line(0,1){17}}
\thinlines
              \put(0.5,2){\line(1,0){7}}
              \put(0.5,4){\line(1,0){7}}
              \put(0.5,6){\line(1,0){7}}
              \put(9.5,6){\line(1,0){7}}
              \put(0.5,11){\line(1,0){7}}
              \put(9.5,4){\line(1,0){7}}
              \put(9.5,2){\line(1,0){7}}
\color{green}
\put(-2,11){\makebox(0,0){$1p0f$}}
\put(-2,6){\makebox(0,0){$1s0d$}}
\put(-2,4){\makebox(0,0){$0p$}}
\put(-2,2){\makebox(0,0){$0s$}}

\color{red}
\put(3,19){\makebox(0,0){$\pi$--protons}}
\put(3,2){\circle*{0.3}}
\put(5,2){\circle*{0.3}}
\put(1.5,4){\circle*{0.3}}
%\put(4,11){\circle*{0.3}}
\put(2.5,4){\circle*{0.3}}
\put(3.5,4){\circle*{0.3}}
\put(4.5,4){\circle*{0.3}}
\put(5.5,4){\circle*{0.3}}
\put(6.5,4){\circle*{0.3}}
\put(1,6){\circle*{0.3}}
\put(1.5,6){\circle*{0.3}}
\put(2,6){\circle*{0.3}}
\put(2.5,6){\circle*{0.3}}
\put(3,6){\circle*{0.3}}
\put(3.5,6){\circle*{0.3}}
\put(4,6){\circle*{0.3}}
\put(4.5,6){\circle*{0.3}}
\put(5,6){\circle*{0.3}}
\put(5.5,6){\circle*{0.3}}
\put(6,6){\circle*{0.3}}
\put(6.5,6){\circle*{0.3}}


\color{blue}
\put(9.5,10){\line(1,0){7}}
\put(12,2){\circle*{0.3}}
\put(14,2){\circle*{0.3}}
\put(10.5,4){\circle*{0.3}}
\put(11.5,4){\circle*{0.3}}
\put(12.5,4){\circle*{0.3}}
\put(13.5,4){\circle*{0.3}}
\put(14.5,4){\circle*{0.3}}
\put(15.5,4){\circle*{0.3}}
\put(10,6){\circle*{0.3}}
\put(10.5,6){\circle*{0.3}}
\put(11,6){\circle*{0.3}}
\put(11.5,6){\circle*{0.3}}
\put(12,6){\circle*{0.3}}
\put(12.5,6){\circle*{0.3}}
\put(13,6){\circle*{0.3}}
\put(13.5,6){\circle*{0.3}}
\put(14,6){\circle*{0.3}}
\put(14.5,6){\circle*{0.3}}
\put(15,6){\circle*{0.3}}
\put(15.5,6){\circle*{0.3}}

\put(11,10){\circle*{0.3}}
\put(11.5,10){\circle*{0.3}}
\put(12,10){\circle*{0.3}}
\put(12.5,10){\circle*{0.3}}
\put(13,10){\circle*{0.3}}
\put(13.5,10){\circle*{0.3}}
\put(14,10){\circle*{0.3}}
\put(14.5,10){\circle*{0.3}}



\put(13,11){\makebox(0,0){$\nu 0f_{7/2}$}}




\put(12,19){\makebox(0,0){$\nu$--neutrons}}
%\put(-1.0,14){\makebox(0,0){\alert{$\Delta \epsilon^{\pi}_{j_a}\propto\sum_{j_i\le F}\overline{v}_{j_aj_i}\normord{a^\dagger_a a_i}$}}}
\pause
              \put(9.5,12){\line(1,0){7}}
\put(13,13){\makebox(0,0){$\nu 1p_{3/2}$}}
\put(10.5,12){\circle*{0.3}}
\put(12.25,12){\circle*{0.3}}
\put(14,12){\circle*{0.3}}
\put(15.75,12){\circle*{0.3}}

\put(13,15){\makebox(0,0){$\nu 1p_{1/2}$}}
              \put(9.5,14){\line(1,0){7}}
\put(12,14){\circle*{0.3}}
\put(14,14){\circle*{0.3}}

              \put(9.5,16){\line(1,0){7}}
\put(13,17){\makebox(0,0){$\nu 0f_{5/2}$}}
\put(10.5,16){\circle*{0.3}}
\put(11.5,16){\circle*{0.3}}
\put(12.5,16){\circle*{0.3}}
\put(13.5,16){\circle*{0.3}}
\put(14.5,16){\circle*{0.3}}
\put(15.5,16){\circle*{0.3}}
         }}
\end{picture}
\end{center}
\end{figure}


	%\includegraphics[width=1.2\textwidth]{sncalcium.pdf}
	%\includegraphics[width=1.2\textwidth]{gapcalcium.pdf}
\begin{itemize}
\item This chain of isotopes exhibits four possible closed-shell nuclei $^{48}$Ni, $^{56}$Ni, $^{68}$Ni
and  $^{78}$Ni.  {\bf FRIB plans systematic studies from $^{48}$Ni to $^{88}$Ni.}
\item  Neutron skin possible for $^{84}$Ni at FRIB.
\item Which is the best closed-shell nucleus?
And again, which part of the nuclear forces drives it?  Is it the strong spin-orbit force, the tensor force, or ..?
\end{itemize}
\begin{figure}
      \begin{center}
\setlength{\unitlength}{0.3cm}
\begin{picture}(16,20)
\thicklines
   \put(1,0.5){\makebox(0,0)[bl]{
              \put(0,1){\line(0,1){8}}
              \put(0,9){\line(1,0){8}}
              \put(8,9){\line(0,1){9}}
              \put(8,18){\line(1,0){9}}
              \put(17,1){\line(0,1){17}}
\thinlines
              \put(0.5,2){\line(1,0){7}}
              \put(0.5,4){\line(1,0){7}}
              \put(0.5,6){\line(1,0){7}}
              \put(9.5,6){\line(1,0){7}}
              \put(9.5,8){\line(1,0){7}}
              \put(0.5,12){\line(1,0){7}}
              \put(0.5,8){\line(1,0){7}}
              \put(9.5,4){\line(1,0){7}}
              \put(9.5,2){\line(1,0){7}}
\color{green}
\put(-0.5,11){\makebox(0,0){$1p0f_{5/2}0g_{9/2}$}}
\put(-2,8){\makebox(0,0){$0f_{7/2}$}}
\put(-2,6){\makebox(0,0){$1s0d$}}
\put(-2,4){\makebox(0,0){$0p$}}
\put(-2,2){\makebox(0,0){$0s$}}

\color{red}
\put(3,19){\makebox(0,0){$\pi$--protons}}
\put(3,2){\circle*{0.3}}
\put(5,2){\circle*{0.3}}
\put(1.5,4){\circle*{0.3}}
\put(2.5,4){\circle*{0.3}}
\put(3.5,4){\circle*{0.3}}
\put(4.5,4){\circle*{0.3}}
\put(5.5,4){\circle*{0.3}}
%\put(4,12){\circle*{0.3}}
\put(6.5,4){\circle*{0.3}}
\put(1,6){\circle*{0.3}}
\put(1.5,6){\circle*{0.3}}
\put(2,6){\circle*{0.3}}
\put(2.5,6){\circle*{0.3}}
\put(3,6){\circle*{0.3}}
\put(3.5,6){\circle*{0.3}}
\put(4,6){\circle*{0.3}}
\put(4.5,6){\circle*{0.3}}
\put(5,6){\circle*{0.3}}
\put(5.5,6){\circle*{0.3}}
\put(6,6){\circle*{0.3}}
\put(6.5,6){\circle*{0.3}}
\put(2,8){\circle*{0.3}}
\put(2.5,8){\circle*{0.3}}
\put(3,8){\circle*{0.3}}
\put(3.5,8){\circle*{0.3}}
\put(4,8){\circle*{0.3}}
\put(4.5,8){\circle*{0.3}}
\put(5,8){\circle*{0.3}}
\put(5.5,8){\circle*{0.3}}


\color{blue}
\put(9.5,10){\line(1,0){7}}
\put(12,2){\circle*{0.3}}
\put(14,2){\circle*{0.3}}
\put(10.5,4){\circle*{0.3}}
\put(11.5,4){\circle*{0.3}}
\put(12.5,4){\circle*{0.3}}
\put(13.5,4){\circle*{0.3}}
\put(14.5,4){\circle*{0.3}}
\put(15.5,4){\circle*{0.3}}
\put(10,6){\circle*{0.3}}
\put(10.5,6){\circle*{0.3}}
\put(11,6){\circle*{0.3}}
\put(11.5,6){\circle*{0.3}}
\put(12,6){\circle*{0.3}}
\put(12.5,6){\circle*{0.3}}
\put(13,6){\circle*{0.3}}
\put(13.5,6){\circle*{0.3}}
\put(14,6){\circle*{0.3}}
\put(14.5,6){\circle*{0.3}}
\put(15,6){\circle*{0.3}}
\put(15.5,6){\circle*{0.3}}

\put(11,8){\circle*{0.3}}
\put(11.5,8){\circle*{0.3}}
\put(12,8){\circle*{0.3}}
\put(12.5,8){\circle*{0.3}}
\put(13,8){\circle*{0.3}}
\put(13.5,8){\circle*{0.3}}
\put(14,8){\circle*{0.3}}
\put(14.5,8){\circle*{0.3}}
\put(12,19){\makebox(0,0){$\nu$--neutrons}}
              \put(9.5,10){\line(1,0){7}}
\put(13,11){\makebox(0,0){$\nu 1p_{3/2}$}}
\put(10.5,10){\circle*{0.3}}
\put(12.25,10){\circle*{0.3}}
\put(14,10){\circle*{0.3}}
\put(15.75,10){\circle*{0.3}}


\put(13,13){\makebox(0,0){$\nu 1p_{1/2}$}}
              \put(9.5,12){\line(1,0){7}}
\put(12,12){\circle*{0.3}}
\put(14,12){\circle*{0.3}}

              \put(9.5,14){\line(1,0){7}}
\put(13,15){\makebox(0,0){$\nu 0f_{5/2}$}}
\put(10.5,14){\circle*{0.3}}
\put(11.5,14){\circle*{0.3}}
\put(12.5,14){\circle*{0.3}}
\put(13.5,14){\circle*{0.3}}
\put(14.5,14){\circle*{0.3}}
\put(15.5,14){\circle*{0.3}}
\pause
\put(13,17){\makebox(0,0){$\nu 0g_{9/2}$}}
              \put(9.5,16){\line(1,0){7}}
\put(10.5,16){\circle*{0.3}}
\put(11,16){\circle*{0.3}}
\put(11.5,16){\circle*{0.3}}
\put(12,16){\circle*{0.3}}
\put(12.5,16){\circle*{0.3}}
\put(13,16){\circle*{0.3}}
\put(13.5,16){\circle*{0.3}}
\put(14,16){\circle*{0.3}}
\put(14.5,16){\circle*{0.3}}
\put(15,16){\circle*{0.3}}
         }}
\end{picture}
      \end{center}
\end{figure}
	%\includegraphics[width=1.2\textwidth]{snnickel.pdf}
	%\includegraphics[width=1.2\textwidth]{gapnickel.pdf}
\begin{enumerate}
\item $^{137}$Sn is the last reported neutron-rich isotope (with half-life).
\item To understand which parts of the nuclear Hamiltonian that drives the
properties of such nuclei will be crucial for our understanding of the stability of matter.
\item Zr isotopes form also long chains of neutron-rich isotopes. {\bf FRIB plans from $^{80}$Zr to
$^{120}$Zr.}
\end{enumerate}
%	%\includegraphics[width=1.2\textwidth]{sntin.pdf}
%	%\includegraphics[width=1.2\textwidth]{gaptin.pdf}
%	%\includegraphics[width=1.2\textwidth]{snlead.pdf}
%	%\includegraphics[width=1.2\textwidth]{gaplead.pdf}
Since we will focus in the beginning on single-particle degrees of freedom and mean-field approaches before we
start with nuclear forces and many-body approaches like the nuclear shell-model, there are some features to be noted
\begin{enumerate}
\item  The total binding energy is not that different from the sum of the individual neutron and proton masses. 
One may thus infer that intrinsic properties of nucleons in a nucleus are close to those of free nucleons.
\item We note clearly a staggering effect between odd and even isotopes with the even ones being more bound (larger separation energies). We will later link this to strong pairing correlations in nuclei.
\item We note also that there are large shell-gaps for some nuclei, meaning that more energy is needed to remove one nucleon. These gaps are used to define so-called magic numbers. 
\end{enumerate}

The root-mean-square (rms) charge radius has been measured for the ground states of many
nuclei. For a spherical charge density, $\rho({\bf r})$, the mean-square radius is defined by:
\[
\langle r^2\rangle = \frac{ \int  d {\bf r} \rho({\bf r}) r^2}{ \int  d {\bf r} \rho({\bf r})},
\]
and the rms radius is the square root of this quantity denoted by
\[
R =\sqrt{ \langle r^2\rangle}.
\]
Radii for most stable
nuclei have been deduced from electron scattering form
factors and/or from the x-ray transition energies of muonic atoms. 
The relative radii for a
series of isotopes can be extracted from the isotope shifts of atomic x-ray transitions.
The rms radius for the nuclear point-proton density, $R_p$ is obtained from the rms charge radius by:
\[
R_p = \sqrt{R^2_{\mathrm{ch}}- R^2_{\mathrm{corr}}},
\]
where
\[
R^2_{\mathrm{corr}}= R^2_{\mathrm{op}}+(N/Z)R^2_{\mathrm{on}}+R^2_{\mathrm{rel}},
\]
where $ R_{\mathrm{op}}= 0.875(7)$ fm  is the rms radius of the proton, $R^2_{\mathrm{on}} = 0.116(2)$ fm$^2$ is the
mean-square radius of the neutron and $R^2_{\mathrm{rel}} = 0.033$ fm$^2$ is the relativistic Darwin-Foldy correction. There are also smaller nucleus-dependent relativistic spin-orbit and
mesonic-exchange corrections that should be included.





\subsection{Many-body Schr\"odinger equation}



The non-relativistic Schr\"odinger equation reads 
\begin{equation}
\hat{H}(x_1, x_2, \hdots , x_A) \Psi_{\lambda}(x_1, x_2, \dots , x_A) = 
E_\lambda  \Psi_\lambda(x_1, x_2, \hdots , x_A), 
\label{eq:basicSE1}
\end{equation}
where the vector $x_i$ represents the coordinates (spatial and spin) of particle $i$, $\lambda$ stands  for all the quantum
numbers needed to classify a given $A$-particle state and $\Psi_{\lambda}$ is the pertaining eigenfunction.  

We write the Hamilton operator, or Hamiltonian,  in a generic way 
\[
	\hat{H} = \hat{T} + \hat{V} 
\]
where $\hat{T}$  represents the kinetic energy of the system
\[
	\hat{T} = \sum_{i=1}^A \frac{\mathbf{p}_i^2}{2m_i} = \sum_{i=1}^A \left( -\frac{\hbar^2}{2m_i} \mathbf{\nabla_i}^2 \right) =
		\sum_{i=1}^A \hat{t}(x_i)
\]
while the operator $\hat{V}$ for the potential energy is given by
\begin{equation}
	\hat{V} = \sum_{i=1}^A \hat{u}_{\mathrm{ext}}(x_i) + \sum_{ji=1}^A \hat{v}(x_i,x_j)+\sum_{ijk=1}^A\hat{v}(x_i,x_j,x_k)+\dots
\label{eq:firstv}
\end{equation}
In the last equation we have singled out an external one-body
potential term $\hat{u}_{\mathrm{ext}}$ which is meant to represent an
effective onebody field in which our particles move. We have therefore
assumed that a picture consisting of individual fermions is a viable
starting point for wave function approximations.  We will specify this
potential later when we need to introduce a calculational basis for
the single-particle states. We have also hinted at the possibility
that our interaction can have three-body terms or even more
complicated many terms.  In atomic, molecular and solid state physics
we normally assume that two-body interactions are sufficient to
describe the system. This is not the case in nuclear physics.

Much of the many-body formalism we will develop, can, with appropriate
modifications, be applied to other fields and disciplines in physics.
If one does quantum chemistry, after having introduced the
Born-Oppenheimer approximation which effectively freezes out the
nucleonic degrees of freedom, the Hamiltonian for $N=n_e$ electrons
takes the following form
\[
  \hat{H} = \sum_{i=1}^{n_e} t(x_i) 
  - \sum_{i=1}^{n_e} k\frac{Z}{r_i} + \sum_{i<j}^{n_e} \frac{k}{r_{ij}},
\]
with $k=1.44$ eVnm\footnote{For nuclei we will replace the units eVnm with MeVfm.} and $n_e$ being the number of electrons. We can rewrite this as
as
\begin{equation}
    \hat{H} = \hat{H_0} + \hat{H_I} 
    = \sum_{i=1}^{n_e}\hat{h_i} + \sum_{i<j=1}^{n_e}\frac{1}{r_{ij}},
\label{H1H2}
\end{equation}
where  we have defined $r_{ij}=| x_i-x_j|$ and
\begin{equation}
  \hat{h_i} =  t(x_i) - \frac{Z}{r_i}.
\label{hi}
\end{equation}
The first term of eq.~(\ref{H1H2}), $H_0$, is the sum of the $N$ or $n_e$
\emph{one-body} Hamiltonians $\hat{h_i}$. Each individual
Hamiltonian $\hat{h_i}$ contains the kinetic energy operator of an
electron and its potential energy due to the attraction of the
nucleus. The second term, $H_I$, is the sum of the $n_e(n_e-1)/2$
two-body interactions between each pair of electrons. Note that the double sum carries a restriction $i<j$.

The potential energy term due to the attraction of the nucleus defines
the onebody field $\hat{u}_i=\hat{u}_{\mathrm{ext}}(x_i)$ of
Eq.~(\ref{eq:firstv}).  We have moved this term into the $\hat{H}_0$
part of the Hamiltonian, instead of keeping it in $\hat{V}$ as in
Eq.~(\ref{eq:firstv}).  The reason is that we will hereafter treat
$\hat{H}_0$ as our non-interacting Hamiltonian. For a many-body
wavefunction $\Phi_{\lambda}$ defined by an appropriate
single-particle basis, we may solve exactly the non-interacting
eigenvalue problem
\[
\hat{H}_0\Phi_{\lambda}= e_{\lambda}\Phi_{\lambda},
\]
with $e_{\lambda}$ being the non-interacting energy. This energy is defined by the sum over single-particle energies to be defined below.
For atoms the single-particle energies could be the hydrogen-like single-particle energies corrected for the charge $Z$. For nuclei and quantum
dots, these energies could be given by the harmonic oscillator in three and two dimensions, respectively.

If we switch to nuclei we no longer have well-defined potential energy  terms (in the strict sense of a potential) and we have to revert
to interaction models.   In this chapter, and in many others as well, we will assume that the interacting part of the Hamiltonian
can be approximated by a two-body interaction model, normally based  on field-thereotical models with selected baryons and mesons.  
This means that our Hamiltonian is written as 
\begin{equation}
    \hat{H} = \hat{H_0} + \hat{H_1} 
    = \sum_{i=1}^A h_i + \sum_{i<j=1}^A V(r_{ij}),
\label{Hnuclei}
\end{equation}
with 
\begin{equation}
  H_0=\sum_{i=1}^A h_i =  \sum_{i=1}^A\left(t(x_i) + u(x_i)\right).
\label{hinuclei}
\end{equation}
The onebody part $u(x_i)$ is normally approximated by a harmonic oscillator potential. However, other potentials are fully possible, such as 
one derived from the self-consistent solution of the Hartree-Fock equations.

For quantum dots, the onebody part is also approximated with a harmonic oscillator potential, either two-dimensional 
or three-dimensional, while the 
interaction term is the standard Coulomb interaction.

Irrespective of these approximations, there is a wealth of experimental evidence that these interactions have to obey specific symmetries. 
The total Hamiltonian should be translationally invariant. If angular momentum is conserved, 
the Hamiltonian is invariant under rotations. Furthermore,
it is invariant under the interchange of two particles and invariant under time reversal and space reflections.  
This means that our Hamiltonian commutes with the respective operators


\subsection{Single-particle basis and computational aspects}

The one-body part $u_{\mathrm{ext}}(x_i)$ is normally approximated by a harmonic oscillator potential or the Coulomb interaction an electron feels from the nucleus. However, other potentials are fully possible, such as 
one derived from the self-consistent solution of the Hartree-Fock equations or so-called Woods-Saxon potentials.

We have defined
\[
    \hat{H} = \hat{H_0} + \hat{H_I} 
    = \sum_{i=1}^A \hat{h}_0(x_i) + \sum_{i<j=1}^A \hat{v}(x_{ij}),
\]
with 
\[
  H_0=\sum_{i=1}^A \hat{h}_0(x_i) =  \sum_{i=1}^A\left(\hat{t}(x_i) + \hat{u}_{\mathrm{ext}}(x_i)\right).
\]

In nuclear physics the one-body part $u_{\mathrm{ext}}(x_i)$ is often approximated by a harmonic oscillator potential or a
Woods-Saxon potential. However, this is not fully correct, because as we have discussed, nuclei are self-bound systems and there is no external confining potential. {\bf The Hamiltonian $H_0$ cannot be used to compute the binding energy of a nucleus since it is not based on a model for the nuclear forces.}. That is, the binding energy is not the sum of the individual single-particle energies. 

The Woods-Saxon potential is a mean field potential for the nucleons (protons and neutrons) 
inside an atomic nucleus. It represent an average potential that a given nucleon feels from  the forces applied on each nucleon. 
The parametrization is
\[
\hat{u}_{\mathrm{ext}}(r)=-\frac{V_0}{1+\exp{(r-R)/a}},
\]
with $V_0\approx 50$ MeV representing the potential well depth, $a\approx 0.5$ fm 
length representing the "surface thickness" of the nucleus and $R=r_0A^{1/3}$, with $r_0=1.25$ fm and $A$ the number of nucleons.
The value for $r_0$ can be extracted from a fit to data, see for example M.~Kirson, Nucl.~Phys.~A {\bf 781}, 350 (2007).
\begin{itemize}
\item It rapidly approaches zero as $r$ goes to infinity, reflecting the short-distance nature of the strong nuclear force.
\item For large $A$, it is approximately flat in the center.
\item Nucleons near the surface of the nucleus experience a large force towards the center.
\end{itemize}
%	%\includegraphics[width=1.25\textwidth]{woodsaxon.pdf}
We have defined
\[
    \hat{H} = \hat{H_0} + \hat{H_I} 
    = \sum_{i=1}^A \hat{h}_0(x_i) + \sum_{i<j=1}^A \hat{v}(x_{ij}),
\]
with 
\[
  H_0=\sum_{i=1}^A \hat{h}_0(x_i) =  \sum_{i=1}^A\left(\hat{t}(x_i) + \hat{u}_{\mathrm{ext}}(x_i)\right).
\]
As stated in previous slides, 
in nuclear physics the one-body part $u_{\mathrm{ext}}(x_i)$ is often 
approximated by a harmonic oscillator potential. However,  as we also noted with the Woods-Saxon potential there is no 
external confining potential in nuclei. 
What many people do then, is to add and subtract a harmonic oscillator potential,
with 
\[
\hat{u}_{\mathrm{ext}}(x_i)=\hat{u}_{\mathrm{ho}}(x_i)= \frac{1}{2}m\omega^2 r_i^2,
\]
where $\omega$ is the oscillator frequency. This leads to 
\[
    \hat{H} = \hat{H_0} + \hat{H_I} 
    = \sum_{i=1}^A \hat{h}_0(x_i) + \sum_{i<j=1}^A \hat{v}(x_{ij})-\sum_{i=1}^A\hat{u}_{\mathrm{ho}}(x_i),
\]
with 
\[
  H_0=\sum_{i=1}^A \hat{h}_0(x_i) =  \sum_{i=1}^A\left(\hat{t}(x_i) + \hat{u}_{\mathrm{ho}}(x_i)\right).
\]











We have introduced a single-particle Hamiltonian
\[
  H_0=\sum_{i=1}^A \hat{h}_0(x_i) =  \sum_{i=1}^A\left(\hat{t}(x_i) + \hat{u}_{\mathrm{ext}}(x_i)\right),
\]
with an external and central symmetric potential $u_{\mathrm{ext}}(x_i)$, which is often 
approximated by a harmonic oscillator potential or a Woods-Saxon potential. Being central symmetric leads to a degeneracy 
in energy which is not observed experimentally. We see this from for example our discussion of separation energies and magic numbers. There are, in addition to the assumed magic numbers from a harmonic oscillator basis of $2,8,20,40,70\dots$ magic numbers like $28$, $50$, $82$ and $126$. 

To produce these additional numbers, we need to add a phenomenological spin-orbit force which lifts the degeneracy, that is
\[
\hat{h}(x_i) =  \hat{t}(x_i) + \hat{u}_{\mathrm{ext}}(x_i) +\xi({\bf r}){\bf ls}=\hat{h}_0(x_i)+\xi({\bf r}){\bf ls}. 
\]
%            %\includegraphics[scale=0.35]{graphics/singleparticle.png}
We have introduced a modified single-particle Hamiltonian
\[
\hat{h}(x_i) =  \hat{t}(x_i) + \hat{u}_{\mathrm{ext}}(x_i) +\xi({\bf r}){\bf ls}=\hat{h}_0(x_i)+\xi({\bf r}){\bf ls}. 
\]
We can calculate the expectation value of the latter using the fact that
\[
\xi({\bf r}){\bf ls}=\frac{1}{2}\xi({\bf r})\left({\bf j}^2-{\bf l}^2-{\bf s}^2\right).
\]
For a single-particle state with quantum numbers $nlj$ (we suppress $s$ and $m_j$), with $s=1/2$ and ${\bf j=l}\pm {\bf s}$, we obtain
the single-particle energies
\[
\varepsilon_{nlj} = \varepsilon_{nlj}^{(0)}+\Delta\varepsilon_{nlj}, 
\]
with $\varepsilon_{nlj}^{(0)}$ being the single-particle energy obtained with $\hat{h}_0(x)$ and 
\[
\Delta\varepsilon_{nlj}=\frac{C}{2}\left(j(j+1)-l(l+1)-\frac{3}{4}\right).
\]
The spin-orbit force gives thus an additional contribution to the energy
\[
\Delta\varepsilon_{nlj}=\frac{C}{2}\left(j(j+1)-l(l+1)-\frac{3}{4}\right),
\]
which lifts the degeneracy we have seen before in the harmonic oscillator or Woods-Saxon potentials. The value $C$ is the radial
integral involving $\xi({\bf r})$. Depending on the value of $j=l\pm 1/2$, we obtain 
\[
\Delta\varepsilon_{nlj=l-1/2}=\frac{C}{2}l,
\]
or
\[
\Delta\varepsilon_{nlj=l+1/2}=-\frac{C}{2}(l+1),
\]
clearly lifting the degeneracy. Note well that till now we have simply postulated the spin-orbit force in {\em ad hoc} way.
Later, we will see how this term arises from the two-nucleon force in a natural way. 
With the spin-orbit force, we can modify our Woods-Saxon potential to 
\[
\hat{u}_{\mathrm{ext}}(r)=-\frac{V_0}{1+\exp{(r-R)/a}}+V_{so}(r){\bf ls},
\]
with 
\[
V_{so}(r) = V_{so}\frac{1}{r}\frac{d f_{so}(r)}{dr},
\]
where we have 
\[
f_{so}(r) = \frac{1}{1+\exp{(r-R_{so})/a_{so}}}.
\]
We can also add, in case of proton, a Coulomb potential. The
Woods-Saxon potential has been widely used in parametrizations of
effective single-particle potentials. {\bf However, as was the case
  with the harmonic oscillator, none of these potentials are linked
  directly to the nuclear forces}. Our next step is to build a mean
field based on the nucleon-nucleon interaction.  This will lead us to
our first and simplest many-body theory, Hartree-Fock theory.



\section{Exercises}
\begin{Exercise}
\begin{enumerate}
\item[a)] 
\end{enumerate}
\end{Exercise}
\begin{Exercise}

\begin{enumerate}
\item[a)] 
\end{enumerate}
\end{Exercise}

%%%%%%%%%%%%%%%%%%%%% appendix.tex %%%%%%%%%%%%%%%%%%%%%%%%%%%%%%%%%
%
% sample appendix
%
% Use this file as a template for your own input.
%
%%%%%%%%%%%%%%%%%%%%%%%% Springer-Verlag %%%%%%%%%%%%%%%%%%%%%%%%%%

\appendix
\motto{All's well that ends well}
\chapter{Chapter Heading}
\label{introA} % Always give a unique label
% use \chaptermark{}
% to alter or adjust the chapter heading in the running head

Use the template \emph{appendix.tex} together with the Springer document class SVMono (monograph-type books) or SVMult (edited books) to style appendix of your book in the Springer layout.


\section{Section Heading}
\label{sec:A1}
% Always give a unique label
% and use \ref{<label>} for cross-references
% and \cite{<label>} for bibliographic references
% use \sectionmark{}
% to alter or adjust the section heading in the running head
Instead of simply listing headings of different levels we recommend to let every heading be followed by at least a short passage of text. Furtheron please use the \LaTeX\ automatism for all your cross-references and citations.


\subsection{Subsection Heading}
\label{sec:A2}
Instead of simply listing headings of different levels we recommend to let every heading be followed by at least a short passage of text. Furtheron please use the \LaTeX\ automatism for all your cross-references and citations as has already been described in Sect.~\ref{sec:A1}.

For multiline equations we recommend to use the \verb|eqnarray| environment.
\begin{eqnarray}
\vec{a}\times\vec{b}=\vec{c} \nonumber\\
\vec{a}\times\vec{b}=\vec{c}
\label{eq:A01}
\end{eqnarray}

\subsubsection{Subsubsection Heading}
Instead of simply listing headings of different levels we recommend to let every heading be followed by at least a short passage of text. Furtheron please use the \LaTeX\ automatism for all your cross-references and citations as has already been described in Sect.~\ref{sec:A2}.

Please note that the first line of text that follows a heading is not indented, whereas the first lines of all subsequent paragraphs are.

% For figures use
%
\begin{figure}[t]
\sidecaption[t]
%\centering
% Use the relevant command for your figure-insertion program
% to insert the figure file.
% For example, with the option graphics use
\includegraphics[scale=.65]{figure}
%
% If not, use
%\picplace{5cm}{2cm} % Give the correct figure height and width in cm
%
\caption{Please write your figure caption here}
\label{fig:A1}       % Give a unique label
\end{figure}

% For tables use
%
\begin{table}
\caption{Please write your table caption here}
\label{tab:A1}       % Give a unique label
%
% For LaTeX tables use
%
\begin{tabular}{p{2cm}p{2.4cm}p{2cm}p{4.9cm}}
\hline\noalign{\smallskip}
Classes & Subclass & Length & Action Mechanism  \\
\noalign{\smallskip}\hline\noalign{\smallskip}
Translation & mRNA$^a$  & 22 (19--25) & Translation repression, mRNA cleavage\\
Translation & mRNA cleavage & 21 & mRNA cleavage\\
Translation & mRNA  & 21--22 & mRNA cleavage\\
Translation & mRNA  & 24--26 & Histone and DNA Modification\\
\noalign{\smallskip}\hline\noalign{\smallskip}
\end{tabular}
$^a$ Table foot note (with superscript)
\end{table}
%


\backmatter%%%%%%%%%%%%%%%%%%%%%%%%%%%%%%%%%%%%%%%%%%%%%%%%%%%%%%%
%%%%%%%%%%%%%%%%%%%%%%acronym.tex%%%%%%%%%%%%%%%%%%%%%%%%%%%%%%%%%%%%%%%%%
% sample list of acronyms
%
% Use this file as a template for your own input.
%
%%%%%%%%%%%%%%%%%%%%%%%% Springer %%%%%%%%%%%%%%%%%%%%%%%%%%

\Extrachap{Glossary}


Use the template \emph{glossary.tex} together with the Springer document class SVMono (monograph-type books) or SVMult (edited books) to style your glossary\index{glossary} in the Springer layout.


\runinhead{glossary term} Write here the description of the glossary term. Write here the description of the glossary term. Write here the description of the glossary term.

\runinhead{glossary term} Write here the description of the glossary term. Write here the description of the glossary term. Write here the description of the glossary term.

\runinhead{glossary term} Write here the description of the glossary term. Write here the description of the glossary term. Write here the description of the glossary term.

\runinhead{glossary term} Write here the description of the glossary term. Write here the description of the glossary term. Write here the description of the glossary term.

\runinhead{glossary term} Write here the description of the glossary term. Write here the description of the glossary term. Write here the description of the glossary term.

\Extrachap{Solutions}

\section*{Problems of Chapter~\ref{intro}}

\begin{sol}{prob1}
The solution\index{problems}\index{solutions} is revealed here.
\end{sol}


\begin{sol}{prob2}
\textbf{Problem Heading}\\
(a) The solution of first part is revealed here.\\
(b) The solution of second part is revealed here.
\end{sol}


\printindex

%%%%%%%%%%%%%%%%%%%%%%%%%%%%%%%%%%%%%%%%%%%%%%%%%%%%%%%%%%%%%%%%%%%%%%

\end{document}





 \documentclass[10pt,english,a4wide,psfig,twoside]{book}


\usepackage{textcomp,type1ec,pdfpages}
\usepackage{bera}

\definecolor{dkgreen}{rgb}{0,0.6,0}
\definecolor{gray}{rgb}{0.5,0.5,0.5}
\definecolor{mauve}{rgb}{0.58,0,0.82}

 \lstset{language=c++}
 \lstset{alsolanguage=[90]Fortran}
 \lstset{alsolanguage=python}
% \lstset{basicstyle=\small}
 \lstset{backgroundcolor=\color{white}}
 \lstset{frame=single}
 \lstset{stringstyle=\ttfamily}
 \lstset{keywordstyle=\color{red}\bfseries}
 \lstset{commentstyle=\itshape\color{blue}}
 \lstset{showspaces=false}
 \lstset{showstringspaces=false}
 \lstset{showtabs=false}
 \lstset{breaklines}
 

% Default settings for code listings
% \lstnewenvironment{Python}[1]{
\lstset{%frame=tb,
  language=c++,
  alsolanguage=python,
  %aboveskip=3mm,
 % belowskip=3mm,
  showstringspaces=false,
  columns=flexible,
  basicstyle={\footnotesize\ttfamily},
  numbers=none,
  numberstyle=\tiny\color{gray},
  commentstyle=\color{dkgreen},
  stringstyle=\color{mauve},
  frame=single,  
  breaklines=true,
  %%%% FOR PYTHON 
  otherkeywords={\ , \}, \{},
  keywordstyle=\color{blue},
  emph={void, ||, &&, break, class,continue, delete, else,
  for, if, include, return,try,while},
  emphstyle=\color{black}\bfseries,
  emph={[2]True, False, None, self},
  emphstyle=[2]\color{dkgreen},
  emphstyle=[2]\color{red},
  emph={[3]from, import, as},
  emphstyle=[3]\color{blue},
  upquote=true,
  morecomment=[s]{"""}{"""},
  commentstyle=\color{green}\slshape, %%% cambie gray por green
  emph={[4]1, 2, 3, 4, 5, 6, 7, 8, 9, 0},
  emphstyle=[4]\color{blue},
  breakatwhitespace=true,
  tabsize=2
}

\renewcommand{\lstlistlistingname}{Code Listings}
\renewcommand{\lstlistingname}{Code Listing}
\definecolor{gray}{gray}{0.5}
\definecolor{green}{rgb}{0,0.5,0}

\lstnewenvironment{Python}[1]{
\lstset{
language=python,
basicstyle=\footnotesize\setstretch{1},
stringstyle=\color{red},
showstringspaces=false,
alsoletter={1234567890},
otherkeywords={\ , \}, \{},
keywordstyle=\color{blue},
emph={access,and,break,class,continue,def,del,elif ,else,%
except,exec,finally,for,from,global,if,import,in,is,%
lambda,not,or,pass,print,raise,return,try,while},
emphstyle=\color{black}\bfseries,
emph={[2]True, False, None, self},
emphstyle=[2]\color{red},
emph={[3]from, import, as},
emphstyle=[3]\color{blue},
upquote=true,
morecomment=[s]{"""}{"""},
commentstyle=\color{dkgreen}\slshape, % el color era gray pero lo cambie a verde
emph={[4]1, 2, 3, 4, 5, 6, 7, 8, 9, 0},
emphstyle=[4]\color{blue},
framexleftmargin=1mm, framextopmargin=1mm, rulesepcolor=\color{blue},
breakatwhitespace=true,
tabsize=2
}}{}


\lstnewenvironment{C++}[1]{
\lstset{
language=c++,
% basicstyle=\ttfamily\small\setstretch{1},
basicstyle=\footnotesize\setstretch{1},
stringstyle=\color{red},
showstringspaces=false,
alsoletter={1234567890},
otherkeywords={\ , \}, \{},
keywordstyle=\color{blue},
emph={access,and,break,class,continue,def,del,elif ,else,%
except,exec,finally,for,from,global,if,import,in,is,%
lambda,not,or,pass,print,raise,return,try,while},
emphstyle=\color{black}\bfseries,
emph={[2]True, False, None, self},
emphstyle=[2]\color{red},
emph={[3]from, import, as},
emphstyle=[3]\color{blue},
upquote=true,
morecomment=[s]{"""}{"""},
commentstyle=\color{dkgreen}\slshape, % el color era gray pero lo cambie a verde
emph={[4]1, 2, 3, 4, 5, 6, 7, 8, 9, 0},
emphstyle=[4]\color{blue},
% literate=*{:}{{\textcolor{blue}:}}{1}%
% {=}{{\textcolor{blue}=}}{1}%
% {-}{{\textcolor{blue}-}}{1}%
% {+}{{\textcolor{blue}+}}{1}%
% {*}{{\textcolor{blue}*}}{1}%
% {!}{{\textcolor{blue}!}}{1}%
% {(}{{\textcolor{blue}(}}{1}%
% {)}{{\textcolor{blue})}}{1}%
% {[}{{\textcolor{blue}[}}{1}%
% {]}{{\textcolor{blue}]}}{1}%
% {<}{{\textcolor{blue}<}}{1}%
% {>}{{\textcolor{blue}>}}{1},%
framexleftmargin=1mm, framextopmargin=1mm, rulesepcolor=\color{blue},
breakatwhitespace=true,
tabsize=2
}}{}




\usepackage{tikz}
\usetikzlibrary{shapes,arrows}

% Define block styles
\tikzstyle{decision} = [diamond, draw, fill=blue!20,
    text width=3.5em, text badly centered, node distance=2.5cm, inner sep=0pt]
\tikzstyle{block} = [rectangle, draw, fill=blue!20,
    text width=8em, text centered, rounded corners, minimum height=4em]
\tikzstyle{line} = [draw, very thick, color=black!50, -latex']
\tikzstyle{cloud} = [draw, ellipse,fill=red!20, node distance=2.5cm,
    minimum height=2em]

\def\radius{.7mm} 
\tikzstyle{branch}=[fill,shape=circle,minimum size=3pt,inner sep=0pt]


\newcommand{\bfv}[1]{\boldsymbol{#1}} 
\newcommand{\Div}[1]{\nabla \bullet \vbf{#1}}           % define divergence
\newcommand{\Grad}[1]{\boldsymbol{\nabla}{#1}}
 \newcommand{\OP}[1]{{\bf\widehat{#1}}}
 \newcommand{\be}{\begin{equation}}
 \newcommand{\ee}{\end{equation}}
\newcommand{\beN}{\begin{equation*}}
\newcommand{\bea}{\begin{eqnarray}}
\newcommand{\beaN}{\begin{eqnarray*}}
\newcommand{\eeN}{\end{equation*}}
\newcommand{\eea}{\end{eqnarray}}
\newcommand{\eeaN}{\end{eqnarray*}}
\newcommand{\bdm}{\begin{displaymath}}
\newcommand{\edm}{\end{displaymath}}
\newcommand{\bsubeqs}{\begin{subequations}}
\newcommand{\esubeqs}{\end{subequations}}
\newcommand{\Obs}[1]{\langle{\Op{#1}\rangle}}             % define observable
\newcommand{\PsiT}{\bfv{\Psi_T}(\bfv{R})}                       % symbol for trial wave function
%\newcommand{\braket}[2]{\langle{#1}|\Op{#2}|{#1}\rangle}
\newcommand{\Det}[1]{{|\bfv{#1}|}}
\newcommand{\uvec}[1]{\mbox{\boldmath$\hat{#1}$\unboldmath}}
\newcommand{\Op}[1]{{\bf\widehat{#1}}}    
\newcommand{\eqbrace}[4]{\left\{
\begin{array}{ll}
#1 & #2 \\[0.5cm]
#3 & #4
\end{array}\right.}
\newcommand{\eqbraced}[4]{\left\{
\begin{array}{ll}
#1 & #2 \\[0.5cm]
#3 & #4
\end{array}\right\}}
\newcommand{\eqbracetriple}[6]{\left\{
\begin{array}{ll}
#1 & #2 \\
#3 & #4 \\
#5 & #6
\end{array}\right.}
\newcommand{\eqbracedtriple}[6]{\left\{
\begin{array}{ll}
#1 & #2 \\
#3 & #4 \\
#5 & #6
\end{array}\right\}}

\newcommand{\mybox}[3]{\mbox{\makebox[#1][#2]{$#3$}}}
\newcommand{\myframedbox}[3]{\mbox{\framebox[#1][#2]{$#3$}}}

%% Infinitesimal (and double infinitesimal), useful at end of integrals
%\newcommand{\ud}[1]{\mathrm d#1}
\newcommand{\ud}[1]{d#1}
\newcommand{\udd}[1]{d^2\!#1}

%% Operators, algebraic matrices, algebraic vectors

%% Operator (hat, bold or bold symbol, whichever you like best):
\newcommand{\op}[1]{\widehat{#1}}
%\newcommand{\op}[1]{\mathbf{#1}}
%\newcommand{\op}[1]{\boldsymbol{#1}}

%% Vector:
\renewcommand{\vec}[1]{\boldsymbol{#1}}

%% Matrix symbol:
\newcommand{\matr}[1]{\boldsymbol{#1}}
%\newcommand{\bb}[1]{\mathbb{#1}}

%% Determinant symbol:
\renewcommand{\det}[1]{|#1|}

%% Means (expectation values) of varius sizes
\newcommand{\mean}[1]{\langle #1 \rangle}
\newcommand{\meanb}[1]{\big\langle #1 \big\rangle}
\newcommand{\meanbb}[1]{\Big\langle #1 \Big\rangle}
\newcommand{\meanbbb}[1]{\bigg\langle #1 \bigg\rangle}
\newcommand{\meanbbbb}[1]{\Bigg\langle #1 \Bigg\rangle}

%% Shorthands for text set in roman font
\newcommand{\prob}[0]{\mathrm{Prob}} %probability
\newcommand{\cov}[0]{\mathrm{Cov}}   %covariance
\newcommand{\var}[0]{\mathrm{Var}}   %variancd

%% Big-O (typically for specifying the speed scaling of an algorithm)
\newcommand{\bigO}{\mathcal{O}}

%% Real value of a complex number
\newcommand{\real}[1]{\mathrm{Re}\!\left\{#1\right\}}

%% Quantum mechanical state vectors and matrix elements (of different sizes)
\newcommand{\brab}[1]{\big\langle #1 \big|}
\newcommand{\brabb}[1]{\Big\langle #1 \Big|}
\newcommand{\brabbb}[1]{\bigg\langle #1 \bigg|}
\newcommand{\brabbbb}[1]{\Bigg\langle #1 \Bigg|}
\newcommand{\ketb}[1]{\big| #1 \big\rangle}
\newcommand{\ketbb}[1]{\Big| #1 \Big\rangle}
\newcommand{\ketbbb}[1]{\bigg| #1 \bigg\rangle}
\newcommand{\ketbbbb}[1]{\Bigg| #1 \Bigg\rangle}
\newcommand{\overlap}[2]{\langle #1 | #2 \rangle}
\newcommand{\overlapb}[2]{\big\langle #1 \big| #2 \big\rangle}
\newcommand{\overlapbb}[2]{\Big\langle #1 \Big| #2 \Big\rangle}
\newcommand{\overlapbbb}[2]{\bigg\langle #1 \bigg| #2 \bigg\rangle}
\newcommand{\overlapbbbb}[2]{\Bigg\langle #1 \Bigg| #2 \Bigg\rangle}
\newcommand{\bracket}[3]{\langle #1 | #2 | #3 \rangle}
\newcommand{\bracketb}[3]{\big\langle #1 \big| #2 \big| #3 \big\rangle}
\newcommand{\bracketbb}[3]{\Big\langle #1 \Big| #2 \Big| #3 \Big\rangle}
\newcommand{\bracketbbb}[3]{\bigg\langle #1 \bigg| #2 \bigg| #3 \bigg\rangle}
\newcommand{\bracketbbbb}[3]{\Bigg\langle #1 \Bigg| #2 \Bigg| #3 \Bigg\rangle}
\newcommand{\projection}[2]
{| #1 \rangle \langle  #2 |}
\newcommand{\projectionb}[2]
{\big| #1 \big\rangle \big\langle #2 \big|}
\newcommand{\projectionbb}[2]
{ \Big| #1 \Big\rangle \Big\langle #2 \Big|}
\newcommand{\projectionbbb}[2]
{ \bigg| #1 \bigg\rangle \bigg\langle #2 \bigg|}
\newcommand{\projectionbbbb}[2]
{ \Bigg| #1 \Bigg\rangle \Bigg\langle #2 \Bigg|}


%\proton{xposition,yposition}
\newcommand{\proton}[1]{%
    \shade[ball color=red] (#1) circle (.25);\draw (#1) node{$+$};
}

%\neutron{xposition,yposition}
\newcommand{\neutron}[1]{%
    \shade[ball color=green] (#1) circle (.25);
}

%\electron{xwidth,ywidth,rotation angle}
\newcommand{\electron}[3]{%
    \draw[rotate = #3](0,0) ellipse (#1 and #2)[color=blue];
    \shade[ball color=Gold2] (0,#2)[rotate=#3] circle (.1);
}

\newcommand{\nucleus}{%
    \neutron{0.1,0.3}
    \proton{0,0}
    \neutron{0.3,0.2}
    \proton{-0.2,0.1}
    \neutron{-0.1,0.3}
    \proton{0.2,-0.15}
    \neutron{-0.05,-0.12}
    \proton{0.17,0.21}
}

%\photoelectron{xwidth,ywidth,rotation angle}
\newcommand{\photoelectron}[3]{%
    \draw[rotate = #3](0,0) ellipse (#1 and #2)[color=blue];%
    \draw[snake=coil,%
        line after snake=0pt, segment aspect=0,%
        segment length=20pt,color=red!50!blue](#3:#1)-- +(-6,0)%
        node[fill=white!70!Gold2,draw=red!80!white, above=0.2cm,pos=0.5]%
            {Incoming $\gamma$-photon};%
    \draw[-stealth,Gold2](#3:#1) -- ++ (5,0.625);%
    \shade[ball color=Gold2](#3:#1)  --  ++(4,0.5)%
        node[fill=white!70!Gold2,draw=red!80!white,%
        text width=3cm, below right=0.2cm]%
            {Photoelectron from an inner shell} circle(0.1);%
    \fill  (#1,0)[rotate=#3,color=white,opaque] circle (.1);%
    \draw  (#1,0)[rotate=#3,color=Gold2] circle (.1) ;%
}

%\comptonelectron{xwidth,ywidth,rotation angle}
\newcommand{\comptonelectron}[3]{%
    \draw[rotate = #3](0,0) ellipse (#1 and #2)[color=blue];%
    \draw[snake=coil, line after snake=0pt,%
        segment aspect=0, segment length=10pt,color=red!50!blue]%
        (#3:#1)-- +(-6,0)%
        node[fill=white!70!Gold2,draw=red!80!white, above=0.2cm,pos=0.5]%
            {Incoming $\gamma$-photon};%
    \draw[-stealth,Gold2](#3:#1) -- ++ (5,2.5);%
    \shade[ball color=Gold2](#3:#1)  --  ++(4,2.0)%
        node[fill=white!70!Gold2,draw=red!80!white, text width=3cm,%
        below right=0.2cm]{Scattered electron from an outer shell} circle(0.1);%
    \fill  (#1,0)[rotate=#3,color=white,opaque] circle (.1);%
    \draw  (#1,0)[rotate=#3,color=Gold2] circle (.1) ;%
    \draw[snake=coil, line after snake=1mm, segment aspect=0,%
        segment length=15pt,color=red!50!blue,-stealth] (#3:#1)-- ++(6,-3)%
        node[fill=white!70!Gold2,draw=red!80!white, right=1cm,pos=0.5]%
            {Scattered $\gamma$-photon};%
}

%\paircreation{impact parameter}
\newcommand{\paircreation}[1]{%
    \draw[snake=coil, line after snake=0pt, segment aspect=0,%
        segment length=5pt,color=red!50!blue] (0,#1)-- +(-6,0)%
        node[fill=white!70!Gold2,draw=red!80!white, above=0.2cm,pos=0.5]%
            {Incoming $\gamma$-photon};%
    \draw[-stealth,Gold2](0,#1) -- ++ (5,2.5);%
    \shade[ball color=Gold2](0,#1)  --  ++(4,2.0)%
        node[fill=white!70!Gold2,draw=red!80!white, below right=0.2cm]%
            {Positron} circle(0.1);%
    \draw[-stealth,Gold2](0,#1) -- ++ (4,-2.0);%
    \shade[ball color=Gold2](0,#1)  --  ++(3.2,-1.6)%
        node[fill=white!70!Gold2!,draw=red!80!white, above right=0.2cm]%
            {Electron} circle(0.1);%
}





 \renewcommand{\rmdefault}{ptm} % Times
 \newcommand{\clearemptydoublepage}{\newpage{\pagestyle{empty}\cleardoublepage}}
 \newcommand{\mymarkright}[1]{\markright{\thesection\ -- #1}}
 \newcommand{\mymarkboth}[2]{\markboth{#1}{#2}}
 %
 % redefine page style
 %
 \usepackage{fancyhdr}
 % choose pagestyle fancy and define it
 \pagestyle{fancy}
 \fancyhead{} %clear everything
 \fancyfoot{} %clear everything
 \addtolength{\headheight}{1.5pt}
 \addtolength{\headwidth}{2.5\marginparsep}
 \renewcommand{\chaptermark}[1]{\mymarkboth{#1}{}}
 \renewcommand{\sectionmark}[1]{\mymarkright{#1}}
 \fancyfoot[EL,OR]{\bfseries\itshape\thepage}
 \fancyhead[EL]{\nouppercase{\bfseries\itshape\leftmark}}
 \fancyhead[OR]{\nouppercase{\bfseries\itshape\rightmark}}
 \renewcommand{\headrulewidth}{1pt}
 % redefine plain page style. used on first pages of chapters et.c.
 \fancypagestyle{plain}{\fancyhf{}\fancyfoot[EL,OR]{\bfseries\itshape\thepage}%
 \renewcommand{\headrulewidth}{0pt}
 }
 %
 % redefine bullet. looks better with an ndash.
 %
 \renewcommand{\labelitemi}{\bfseries--}
 %
 % define styles for section and subsection headings
 %
 %\font\tenhv  = phvb at 12pt
 %\newcommand{\hvfont}{\fontfamily{phv}\selectfont}
 %\newcommand{\hvfont}{\fontfamily{phv}\fontseries{bc}\selectfont}
 \newcommand{\secstyle}{\normalfont\itshape\bfseries\large}
 %\newcommand{\secstyle}{\hvfont\itshape\large}
 \newcommand{\subsecstyle}{\normalfont\itshape\large}
 \newcommand{\paragraphstyle}{\normalfont\itshape}
 %
 % redefine look of sections, subsections and paragraphs.
 %
 \makeatletter
 \renewcommand{\section}{%
 \@startsection
   {section}%   the name
   {1}%         the level
   {0pt} %{-\Myindent}%       the indent
   {-2\baselineskip}%   beforeskip
   {\baselineskip}%    afterskip
   {\secstyle}}%   style
 \renewcommand{\subsection}{%
 \@startsection
   {subsection}%   the name
   {2}%         the level
   {0pt}%{-\Myindent}%       the indent
   {-\baselineskip}%   beforeskip
   {0.5\baselineskip}%    afterskip
   {\subsecstyle}}%   style
 \renewcommand{\paragraph}{%
 \@startsection
   {paragraph}%   the name
   {4}%         the level
   {0pt}%{-\Myindent}%       the indent
   {-\baselineskip}%   beforeskip
   {-1em}%    afterskip
   {\paragraphstyle}}%   style
 \makeatother
 %%
 %% modify margins...
 %%
 \addtolength{\oddsidemargin}{-0.5cm}
 %\addtolength{\evensidemargin}{-0.5cm}
 \addtolength{\textwidth}{0.5cm}


 \begin{document}


 \thispagestyle{empty}
 %    \vspace*{\stretch{1}}
     \rule{\linewidth}{1mm}
     \begin{flushright}
           \Huge COMPUTATIONAL PHYSICS\\[5mm]
            Morten Hjorth-Jensen
     \end{flushright}
     \rule{\linewidth}{1mm}
     \vspace*{\stretch{2}}
 \begin{center}
 \begin{figure}[hb]
 \includegraphics[scale=1.0]{Nebbdyr2.ps}
 \end{figure}
 \end{center}
       \begin{center}
 %        \Large{www.computationalphysics.net}
         \Large{University of Oslo, Fall 2010}
       \end{center}

 \pagenumbering{roman}

 \clearemptydoublepage

 \pagestyle{fancy}

 %  the preface

 \section*{Preface}

 
Nearly all of physics is many-body physics at the most  
microscopic level of understanding.
A theoretical understanding of the behavior of quantum mechanical systems with many interacting particles,
accompanied with experiments and simulations,   
is a  great challenge and provides fundamental insights into systems governed by quantum mechanical principles.

Many-body physics  applications range from condensed matter physics
(for example antiferromagnets, quantum liquids and solids, plasmas, metals,
superconductors), nuclear and high energy physics (for example nuclear matter, finite
nuclei, quark-gluon plasmas), dense matter astrophysics (for example  neutron
stars, white dwarfs), atoms and molecules (for example electron correlations,
reactions, quantum chemistry), and elementary particles
(for example quantum field theory, lattice gauge field models). 

Microscopic many-body methodologies have attained a high degree of
sophistication in the last decades, from both theoretical and
applicative  points of view. These developments have allowed
researchers to adapt these technologies to a much wider range of physical
systems, and to obtain quantitative descriptions of many 
observables. 

Several of these methods have been developed independently of each other.
In some  cases the same methods have been applied and studied in different
fields of research with little overlap and exchange of knowledge. 
A classic example is the development of coupled-cluster approaches,
which originated from the nuclear many-body problem in the late fifties ({\bf give refs here later}). 
These methods were adapted and refined 
to extremely high precision and predicability by quantum chemists over 
the next four decades. The developments comprised, and still do,  
both methodological innovations and the usage of advances in high-performance computing.    
As time elapsed,   
the overlap with the original nuclear many-body problem vanished gradually, 
dwindling almost to a non-vanishing exchange of knowledge and methodologies. 
These methods were however revived in nuclear physics towards the turn of the last century and offer now a viable approach
to {\em ab initio}\footnote{We would like to come with a warning here, since it is our feeling that the 
labelling {\em ab initio} attached to many titles of scientific articles nowadays, has reached an inflationary stage. 
With {\em ab initio} we will mean solving Schr\"odinger's or Dirac's many-particle equations starting with the presumably best 
inter-particle Hamiltonian, making no approximations. The particles that make up the many-body problem will be approximated 
as the constituent ones. 
In the nuclear many-body problem
these are various baryons such as protons, neutrons, isobars and hyperons and low-mass mesons. All many-body approaches discussed in this text involve
some approximations, and only few of them, if any,  can reach the glorious stage of being truely {\em ab initio}.
Truely {\em ab initio} in nuclear physics means to solve the many-body problem starting with quarks and gluons.}
nuclear structure  studies of nuclei,   spanning the whole range of stable closed-shell and several open-shell nuclei. 


Many of the developments are also not 
always readily accessible to a larger community of
researchers and especially to the young and uninitiated ones. A notorious exception is 
again the quantum chemistry community, where highly effecient codes and methods are made available.  
Numerical packages like
Gamess, Dalton and Gaussian ({\bf give refs here later}) provide a reliable computational 
platform for many  chemical applications and studies.
In other fields this is not always the case.     

Nuclear physics, where we have our basic research experience, is such a field.
This book aims therefore at giving you an overview 
of widely used quantum mechanical  many-body methods. We want also to provide you with computational tools in order to enable you
to perform studies of realistic systems  in nuclear physics. 
The methods we discuss are however not limited to applications
in nuclear physics, rather, with  minor  changes one can apply them to other fields, from quantum chemistry, via studies of atoms and
molecules to materials science and solid state devices such as quantum dots.  

Typical methods we will discuss are 
configuration interaction and large-scale diagonalization.methods (shell-model in nuclear physics), perturbative many-body methods, 
coupled cluster theory, Green's function approaches, various Monte Carlo methods, extensions to weakly  bound systems and links
from {\em ab initio} methods to  density functional and mean field theories. 
The methods we have chosen are those which enjoy a great degree of popularity in low and intermediate energy nuclear physics, 
but due to both our limited knowledge
and due to space considerations, some of these will be treated in rather cavalier way.

Focussing on methods and tools only is however not necessarily a recipe for success, as we all know a 
a bad workman quarrels with his tools. Our aim is to provide 
you with a didactical description
of some of these  powerful methods and tools with the purpose of allowing you to
gain further insights and a possible working experience in modern many-body methods.
There are several textbooks on the market which cover specific 
many-body methods and/or applications, but almost none of them
presents an updated and comparative description of various many-body methods. The reason for choosing such an approach
is that we want you, the reader, to gain the experience and form your own opinion on the suitability and applicability
of specific methods. Too often you may have met many-body practitioners
claiming 'my method is better than your method because...'.  We wish to avoid such a pitfal, although ours, as any other approach,
is tainted by our biases, preferences and obviously ignorance about all possible details. 

We also  want  to shed light, in a hopefully unbiased way, on possible interplays and
differences between the various many-body methods with a focus
on the physics content. 
We have therefore selected several physical systems,
from simple models to realistic cases,  whose properties can be
described within the theoretical frameworks presented here.
In this text we will expose you to problems which tour many nuclear physics applications, 
from the basic forces which bind nucleons together to compact objects such as neutron stars.  
Our hope is that the material we have provided   allows an eventual reader to  assess 
by himself/herself the pro and cons of the methods discussed.

Moreover,  the text can be used and read as a large project. Each chapter ends with a specific project. The code you end up developing 
for that particular project, can in turn be  reused in the next chapter and its pertaining project.  To give you an example,
in chapter xx you will end up constructing your own nucleon-nucleon interaction and solve its Schr\"odinger equation using the Lippman-Schwinger
equation.  This solution results in the so-called $T$-matrix  which in turn can be related to the experimental phase shifts.
The Lippman-Schwinger equation can easily be modified to account for a given nuclear medium, resulting in the 
so-called $G$-matrix.  In chapter xx you end up computing such an effective interaction, using similarity transformations as well.
With these codes, you are in turn in the position where you can define effective interactions for say coupled-cluster calculations
(the topic of chapter xx) or shell-model effective interactions (topics for chapter xx and xx).

Our hope is that this will enable  you to assess at a much deeper level the pros and cons of the various methods.  
We believe firmly that knowledge of the strengths and weaknesses of a given method allows you to realize where
improvements can be made.   The text has also a strong focus on computational issues, from how to build up large many-body codes 
to parallelization and high-performance computing topics.



This texts spans some four hundred pages, and with the promised wide scope we aim at, 
there have to be topics which will not be dealt with properly.
In particular we will not cover  reaction theories.  
For reaction theories we provide all the inputs needed to compute
onebody densities, spectroscopic factors and optical potentials. 
For density functional theories we discuss how one can constrain the exchange correlations based on {\em ab initio} methods.
Furthermore, when it comes to the underlying nuclear forces, we will assume that various baryons and mesons are the 
effective degrees of freedom, limiting ourselves to low and intermediate energy nuclear physics.  A discussion of recent 
advances in lattice quantumchromodynamics (QCD) and its link to effective field theories is also outside the scope
of this book, although we will refer to advances in these fields as well and link the construction of nuclear forces to
undergoing research in effective field theory and lattice QCD.  
The material on infinite matter can be extended to finite temperatures, but we do not 
adventure into the realm of heavy-ion collisions and the search of the holy grail of quark-gluon plasma. 
We will throughout pay loyalty to a physical world governed by the degrees of freedom of baryons and mesons.
We apologize for these shortcomings, which are mainly due to
our lack of detailed research knowledge in the above fields. To be a jack of all trades leads normally to mastering none.

A final warning however.
This text is not a text on many-body theories, rather it is
a text on applications and implementations of many-body theories. This applies also to many of the algoritms  we discuss.  We do not go into
details about for example Gaussian quadrature or the mathematical foundations of random walks and the Metropolis algorithm.  We will often simply state results, but add 
enough references to tutorial texts so that you can look up the missing wonders of numerical mathematics yourself.
Similarly,  handling of angular momenta and their recouplings via $3j$, $6j$ or $9j$ symbols should  not  come as a bolt out from the blue. But again, don't despair,
we'll guide you safely to the appropriate literature.

We have therefore taken the liberty to have  certain expectations about you, the potential reader.  
We assume that you are at least a graduate student who is embarking on studies in 
nuclear physics and that
you have some familiarity with basic nuclear physics, angular momentum theory and many-body theory, 
typically at the level of texts like those of Talmi, Fetter and Walecka, Blaizot and Ripka, Dickhoff and Van Neck or similar monographs (add refs later). 
We will obviously state and repeat the basic rules and theorems, but will not go into derivations.  
Some of the derivations will however be left to you via various exercises interspersed througout the text.


\subsection*{Acknowledgements}
Friends/colleagues bla bla  and sponsors bla bla. also add that errors, misunderstandings etc are mainly due to ourselves!!







 








 \clearemptydoublepage

 \tableofcontents

 \clearemptydoublepage

 \pagenumbering{arabic}
 %  Introductory chapters
         \part{Introduction to Programming and Numerical Methods}
 %%  Introduction
         \input{introduction}
 \clearemptydoublepage
 %%  introduction to C++ and F90 programming
         \input{basicCF90.tex}
 \clearemptydoublepage
 %% numerical derivation
      \input{differentiate.tex}
 \clearemptydoublepage
 %% Classes.
%\input{classes}
% \clearemptydoublepage
 %% numerical integration
      \input{integrate.tex}
 \clearemptydoublepage
 % this section should be included at an earlier stage
      \input{nonlinear.tex}
 \clearemptydoublepage
% \clearemptydoublepage
         \part{Linear Algebra and Eigenvalue problems}
 %% Linear algebra.
 \input{linalgebra}
% Put diagonalization chapter here
 \clearemptydoublepage
 %% interpolation  this chapter should come earlier
%      \input{interpolate.tex}
% \clearemptydoublepage
 %% eigenvalue systems
     \input{eigenvalue.tex}
 \clearemptydoublepage
        \part{Ordinary and Partial Differential Equations}
 %%  Differential equations 
         \input{diffeq.tex}
 \clearemptydoublepage
 %% Two point boundary value problems. 
         \input{twopboundary.tex}
 \clearemptydoublepage
 %% Partial differential equations, finite difference
 \input{partdiff.tex}
 \clearemptydoublepage
        \part{Monte Carlo Methods}
 %% Monte Carlo methods
      \input{montecarlo_intro.tex}
 \clearemptydoublepage
 %% random walks and the diffusion equation
      \input{randowalks.tex}
 \clearemptydoublepage
 %% Monte carlo applications, stat phys
      \input{stat_phys.tex}
 \clearemptydoublepage
% \chapter{Modelling Phase Transitions in Statistical Physics}\label{chap:advancedstatphys}
% \input{advancedsm.tex}
% \clearemptydoublepage
 %% Monte carlo applications, quantum mechanics
      \input{vmc.tex}
 \clearemptydoublepage




 %  Advanced topics
 \part{Advanced topics}   

 \chapter{Many-body approaches to studies of electronic systems: Hartree-Fock theory}\label{chap:advancedatoms}

 \input{advancedatoms}

 \clearemptydoublepage
 \chapter{Bose-Einstein condensation and Diffusion Monte Carlo}\label{chap:advancedqmc}

 \input{advancedqm.tex}


% \clearemptydoublepage
%\chapter{Density functional theory}\label{chap:dft}
%\input{dft}



% \clearemptydoublepage




%\chapter{Quantum Information Theory and Quantum Algorithms}\label{chap:quantinfo}


%\input{quantuminformation}





 \clearemptydoublepage



 \bibliographystyle{unsrt}

 \bibliography{../../../Library_of_Topics/mylib}

 \end{document}









