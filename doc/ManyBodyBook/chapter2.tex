\chapter{Observables and simple potential models}\label{chap:basics}


\section{Introduction}\label{sec:basics_intro}


\section{Masses and radii}
A basic quantity which can be measured for the ground states of nuclei is the atomic mass
$M(N, Z)$ of the neutral atom with atomic mass number $A$ and charge $Z$. The number of neutrons are $N$.

Atomic masses are
usually tabulated in terms of the mass excess defined by
\[
\Delta M(N, Z) =  M(N, Z) - uA,
\]
where $u$ is the Atomic Mass Unit 
\[
u = M(^{12}\mathrm{C})/12 = 931.49386 \hspace{0.1cm} \mathrm{MeV}/c^2.
\]
%data from the 2003 compilation of Audi, Wapstra and Thibault.

The nucleon masses are
\[
m_p = 938.27203(8)\hspace{0.1cm} \mathrm{MeV}/c^2 = 1.00727646688(13)u,
\]
and 
\[
m_n = 939.56536(8)\hspace{0.1cm} \mathrm{MeV}/c^2 = 1.0086649156(6)u.
\]
In the 2003 mass evaluation there are 2127 nuclei measured with an accuracy of 0.2
MeV or better, and 101 nuclei measured with an accuracy of greater than 0.2 MeV. For
heavy nuclei one observes several chains of nuclei with a constant $N-Z$ value whose masses
are obtained from the energy released in alpha decay.

Nuclear binding energy is defined as the energy required to break up a given nucleus
into its constituent parts of $N$ neutrons and $Z$ protons. In terms of the atomic masses
$M(N, Z)$ the binding energy is defined by:
\[
BE(N, Z) = ZM_H c^2 + Nm_n c^2 - M(N, Z)c^2 ,
\]
where $M_H$ is the mass of the hydrogen atom and $m_n$ is the mass of the neutron.
In terms
of the mass excess the binding energy is given by:
\[
BE(N, Z) = Z\Delta_H c^2 + N\Delta_n c^2 -\Delta(N, Z)c^2 ,
\]
where $\Delta_H c^2 = 7.2890$ MeV and $\Delta_n c^2 = 8.0713$ MeV.
For the liquid drop model we have
\[ BE(N,Z) = a_1A-a_2A^{2/3}-a_3\frac{Z^2}{A^{1/3}}-a_4\frac{(N-Z)^2}{A}\]
We could also add a so-called pairing term, which is a correction term that
arises from the tendency of proton pairs and neutron pairs to
occur. An even number of particles is more stable than an odd number.
\begin{itemize}
\item $a_1A$: Volume energy. When an assembly of nucleons of the same size is packed
together into the smallest volume, each interior nucleon has a certain
number of other nucleons in contact with it. This contribution is proportional to the volume.
\item $a_2A^{2/3}$:   Surface energy. A nucleon at the
surface of a nucleus interacts with fewer other nucleons than one in
the interior of the nucleus and hence its binding energy is less. This
surface energy term takes that into account and is therefore negative
and is proportional to the surface area.
\item $a_3\frac{Z^2}{A^{1/3}}$: Coulomb Energy. The electric
repulsion between each pair of protons in a nucleus yields less binding. 
\item $a_4\frac{(N-Z)^2}{A}$: Asymmetry energy, associated with the Pauli exclusion principle 
and reflecting the fact that the proton-neutron interaction is more attractive on the average than the neutron-neutron and proton-proton interactions.
\begin{itemize}
\item Red: experimental data, blue: liquid drop model
\item $a_1=15.49$ MeV
\item $a_2=17.23$ MeV
\item $a_3=0.697$ MeV
\item $a_4=22.6$ MeV
\end{itemize}
%	%\includegraphics[width=1.2\textwidth]{beexpliquid.pdf}





We consider energy conservation for nuclear transformations that include, for
example, the fusion of two nuclei $a$ and $b$ into the combined system $c$
\[
{^{N_a+Z_a}}a+ {^{N_b+Z_b}}b\rightarrow {^{N_c+Z_c}}c
\]
or the decay of nucleus $c$ into two other nuclei $a$ and $b$
\[
^{N_c+Z_c}c \rightarrow  ^{N_a+Z_a}a+ ^{N_b+Z_b}b
\]
In general we have the reactions
\[
\sum_i {^{N_i+Z_i}}i \rightarrow  \sum_f {^{N_f+Z_f}}f
\]
We require also that number of protons and neutrons are conserved in the initial stage and final stage, unless we have processes which violate baryon conservation, 
\[
\sum_iN_i = \sum_f N_f \hspace{0.2cm}\mathrm{and} \hspace{0.2cm}\sum_iZ_i = \sum_f Z_f.
\]

This process is characterized by an energy difference called the $Q$ value:
\[
Q=\sum_iM(N_i, Z_i)c^2-\sum_fM(N_f, Z_f)c^2=\sum_iBE(N_f, Z_f)-\sum_iBE(N_i, Z_i)
\]
Spontaneous decay involves a single initial nuclear state and is allowed if $Q > 0$. In the
decay, energy is released in the form of the kinetic energy of the final products. Reactions
involving two initial nuclei and are endothermic (a net loss of energy) if $Q < 0$; the reactions
are exothermic (a net release of energy) if $Q > 0$.

We can consider the Q values associated with the removal of one or two nucleons from
a nucleus. These are conventionally defined in terms of the one-nucleon and two-nucleon
separation energies
\[
S_n= -Q_n= BE(N,Z)-BE(N-1,Z),
\]
\[
S_p= -Q_p= BE(N,Z)-BE(N,Z-1),
\]
\[
S_{2n}= -Q_{2n}= BE(N,Z)-BE(N-2,Z),
\]
and
\[
S_{2p}= -Q_{2p}= BE(N,Z)-BE(N,Z-2),
\]

Using say the neutron separation energies (alternatively the proton separation energies)
\[
S_n= -Q_n= BE(N,Z)-BE(N-1,Z),
\]
we can define the so-called energy gap for neutrons (or protons) as 
\[
\Delta S_n= BE(N,Z)-BE(N-1,Z)-\left(BE(N+1,Z)-BE(N,Z)\right),
\]
or 
\[
\Delta S_n= 2BE(N,Z)-BE(N-1,Z)-BE(N+1,Z).
\]
This quantity can in turn be used to determine which nuclei are magic or not. 
For protons we would have 
\[
\Delta S_p= 2BE(N,Z)-BE(N,Z-1)-BE(N,Z+1).
\]
We can also define the two-neutron or two-proton gap as well. 


\section{Definitions  and single-particle basis functions}

In this text we define an operator as $\hat{O}$. Unless otherwise
specified the number of particles is always $A$ (representing the total number of nucleons) 
and $d$ is the dimension of the 
system. 
In nuclear physics we normally define the total number of particles to be $A=N+Z$,
where $N$ is total number of neutrons and $Z$ the total number of protons. In case of other baryons like isobars $\Delta$ or
various hyperons, one needs to add their definitions.  

The quantum numbers of a single-particle state in coordinate space are
defined by the variable $x=({\bf r},\sigma)$, where ${\bf r}\in {\mathbb{R}}^{d}$with $d=1,2,3$ represents 
the spatial coordinates and $\sigma$ is the eigenspin of the particle. For fermions with eigenspin $1/2$ this means that
\[
 x\in {\mathbb{R}}^{d}\oplus (\frac{1}{2}),
\]
and the integral
\[
\int dx = \sum_{\sigma}\int d^dr = \sum_{\sigma}\int d{\bf r},
\]
and for $A$ nucleons we have
\[
\int d^Ax= \int dx_1\int dx_2\dots\int dx_A.
\]

The quantum mechanical wave function of a 
given state with quantum numbers $\lambda$ (encompassing all quantum numbers needed to specify the system), but ignoring time, is
\[
\Psi_{\lambda}=\Psi_{\lambda}(x_1,x_2,\dots,x_A),
\]
with $x_i=({\bf r}_i,\sigma_i)$ and the projection of $\sigma_i$ takes the values
$\{-1/2,+1/2\}$ for particles with spin $1/2$. 
We will hereafter always refer to $\Psi_{\lambda}$ as the exact wave function, and if the ground state is not degenerate we label it as 
\[
\Psi_0=\Psi_0(x_1,x_2,\dots,x_A).
\]
We will throughout this text use 
upper-case letters are used for many-particle state functions, while lower-case letters will be used for single-particle
state functions.
Since the solution $\Psi_{\lambda}$ seldomly can be found in closed
form, approximations are sought. In this text we define an
approximate wave function or an ansatz to the exact wave function as
\[
\Phi_{\lambda}=\Phi_{\lambda}(x_1,x_2,\dots,x_N),
\]
with 
\[
\Phi_0=\Phi_0(x_1,x_2,\dots,x_N),
\]
being the ansatz to the ground state.  

The state function $\Psi_{\lambda}$ is sought in the Hilbert space of either symmetric or anti-symmetric $A$-body functions, namely
\[
\Psi_{\lambda}\in {\cal H}_N:= {\cal H}_1\oplus{\cal H}_1\oplus\dots\oplus{\cal H}_1,
\]
where the single-particle Hilbert space ${\cal H}_1$ is the space of square integrable functions over
$\in {\mathbb{R}}^{d}\oplus (\sigma)$
resulting in
\[
{\cal H}_1:= L^2(\mathbb{R}^{d}\oplus (\sigma)).
\]

%      \frametitle{Do we understand the physics of dripline systems?}
\begin{itemize}
\item The oxygen isotopes are the heaviest isotopes for
which the drip line is well established.
\item Two out of four
stable even-even isotopes exhibit a doubly magic nature,
namely $^{22}$O ($Z=8$, $N=14$) and $^{24}$O ($Z=8$, $N=16$).
\item 
The structure of $^{22}$O and $^{24}$O is assumed to be governed
by the evolution of the $1s_{1/2}$ and $0d_{5/2}$  one-quasiparticle states.
\item The isotopes
$^{25}$O
$^{26}$O, $^{27}$O and $^{28}$O are outside the drip line, since the $0d_{3/2}$ orbit is not bound.
\end{itemize}

\begin{figure}
	%\includegraphics[width=1.2\textwidth]{snoxygen.pdf}
\end{figure}
\begin{figure}
	%\includegraphics[width=1.2\textwidth]{gapoxygen.pdf}
\end{figure}

\begin{itemize}
\item The Ca  isotopes exhibit several possible closed-shell nuclei $^{40}$Ca, $^{48}$Ca, $^{52}$Ca, $^{54}$Ca,
and  $^{60}$Ca. 
\item  Magic neutron numbers are then $N=20, 28, 32, 34, 40$. 
\item Masses available up to $^{54}$Ca, Gallant {\em et al.},Phys.~Rev.~Lett.~{\bf 109}, 032506 (2012) and K.~Baum {\em et al}, Nature {\bf 498}, 346 (2013).
\item Heaviest observed $^{57,58}$Ca. NSCL experiment,  O.~B.~Tarasov {\it et al.}, Phys.~Rev.~Lett.~{\bf 102}, 142501 (2009). Cross sections for $^{59,60}$Ca assumed small ($< 10^{-12}$mb).
\item Which degrees of freedom prevail close to $^{60}$Ca and beyond?
\end{itemize}

\begin{itemize}
\item {\bf Mass models and mean field models predict the dripline at $A\sim 70$!} Important consequences for modeling of nucleosynthesis related processes.
\item Can we predict reliably which is the last stable calcium isotope? 
\item And how
does this compare with popular mass models on the market? See Nature 486, 509 (2012). 
\item And which parts of the underlying forces
are driving the physics towards the dripline?
\end{itemize}


\begin{figure}
\begin{center}
\setlength{\unitlength}{0.4cm}
\begin{picture}(16,20)
\thicklines
   \put(1,0.5){\makebox(0,0)[bl]{
              \put(0,1){\line(0,1){8}}
              \put(0,9){\line(1,0){8}}
              \put(8,9){\line(0,1){9}}
              \put(8,18){\line(1,0){9}}
              \put(17,1){\line(0,1){17}}
\thinlines
              \put(0.5,2){\line(1,0){7}}
              \put(0.5,4){\line(1,0){7}}
              \put(0.5,6){\line(1,0){7}}
              \put(9.5,6){\line(1,0){7}}
              \put(0.5,11){\line(1,0){7}}
              \put(9.5,4){\line(1,0){7}}
              \put(9.5,2){\line(1,0){7}}
\color{green}
\put(-2,11){\makebox(0,0){$1p0f$}}
\put(-2,6){\makebox(0,0){$1s0d$}}
\put(-2,4){\makebox(0,0){$0p$}}
\put(-2,2){\makebox(0,0){$0s$}}

\color{red}
\put(3,19){\makebox(0,0){$\pi$--protons}}
\put(3,2){\circle*{0.3}}
\put(5,2){\circle*{0.3}}
\put(1.5,4){\circle*{0.3}}
%\put(4,11){\circle*{0.3}}
\put(2.5,4){\circle*{0.3}}
\put(3.5,4){\circle*{0.3}}
\put(4.5,4){\circle*{0.3}}
\put(5.5,4){\circle*{0.3}}
\put(6.5,4){\circle*{0.3}}
\put(1,6){\circle*{0.3}}
\put(1.5,6){\circle*{0.3}}
\put(2,6){\circle*{0.3}}
\put(2.5,6){\circle*{0.3}}
\put(3,6){\circle*{0.3}}
\put(3.5,6){\circle*{0.3}}
\put(4,6){\circle*{0.3}}
\put(4.5,6){\circle*{0.3}}
\put(5,6){\circle*{0.3}}
\put(5.5,6){\circle*{0.3}}
\put(6,6){\circle*{0.3}}
\put(6.5,6){\circle*{0.3}}


\color{blue}
\put(9.5,10){\line(1,0){7}}
\put(12,2){\circle*{0.3}}
\put(14,2){\circle*{0.3}}
\put(10.5,4){\circle*{0.3}}
\put(11.5,4){\circle*{0.3}}
\put(12.5,4){\circle*{0.3}}
\put(13.5,4){\circle*{0.3}}
\put(14.5,4){\circle*{0.3}}
\put(15.5,4){\circle*{0.3}}
\put(10,6){\circle*{0.3}}
\put(10.5,6){\circle*{0.3}}
\put(11,6){\circle*{0.3}}
\put(11.5,6){\circle*{0.3}}
\put(12,6){\circle*{0.3}}
\put(12.5,6){\circle*{0.3}}
\put(13,6){\circle*{0.3}}
\put(13.5,6){\circle*{0.3}}
\put(14,6){\circle*{0.3}}
\put(14.5,6){\circle*{0.3}}
\put(15,6){\circle*{0.3}}
\put(15.5,6){\circle*{0.3}}

\put(11,10){\circle*{0.3}}
\put(11.5,10){\circle*{0.3}}
\put(12,10){\circle*{0.3}}
\put(12.5,10){\circle*{0.3}}
\put(13,10){\circle*{0.3}}
\put(13.5,10){\circle*{0.3}}
\put(14,10){\circle*{0.3}}
\put(14.5,10){\circle*{0.3}}



\put(13,11){\makebox(0,0){$\nu 0f_{7/2}$}}




\put(12,19){\makebox(0,0){$\nu$--neutrons}}
%\put(-1.0,14){\makebox(0,0){\alert{$\Delta \epsilon^{\pi}_{j_a}\propto\sum_{j_i\le F}\overline{v}_{j_aj_i}\normord{a^\dagger_a a_i}$}}}
\pause
              \put(9.5,12){\line(1,0){7}}
\put(13,13){\makebox(0,0){$\nu 1p_{3/2}$}}
\put(10.5,12){\circle*{0.3}}
\put(12.25,12){\circle*{0.3}}
\put(14,12){\circle*{0.3}}
\put(15.75,12){\circle*{0.3}}

\put(13,15){\makebox(0,0){$\nu 1p_{1/2}$}}
              \put(9.5,14){\line(1,0){7}}
\put(12,14){\circle*{0.3}}
\put(14,14){\circle*{0.3}}

              \put(9.5,16){\line(1,0){7}}
\put(13,17){\makebox(0,0){$\nu 0f_{5/2}$}}
\put(10.5,16){\circle*{0.3}}
\put(11.5,16){\circle*{0.3}}
\put(12.5,16){\circle*{0.3}}
\put(13.5,16){\circle*{0.3}}
\put(14.5,16){\circle*{0.3}}
\put(15.5,16){\circle*{0.3}}
         }}
\end{picture}
\end{center}
\end{figure}


	%\includegraphics[width=1.2\textwidth]{sncalcium.pdf}
	%\includegraphics[width=1.2\textwidth]{gapcalcium.pdf}
\begin{itemize}
\item This chain of isotopes exhibits four possible closed-shell nuclei $^{48}$Ni, $^{56}$Ni, $^{68}$Ni
and  $^{78}$Ni.  {\bf FRIB plans systematic studies from $^{48}$Ni to $^{88}$Ni.}
\item  Neutron skin possible for $^{84}$Ni at FRIB.
\item Which is the best closed-shell nucleus?
And again, which part of the nuclear forces drives it?  Is it the strong spin-orbit force, the tensor force, or ..?
\end{itemize}
\begin{figure}
      \begin{center}
\setlength{\unitlength}{0.3cm}
\begin{picture}(16,20)
\thicklines
   \put(1,0.5){\makebox(0,0)[bl]{
              \put(0,1){\line(0,1){8}}
              \put(0,9){\line(1,0){8}}
              \put(8,9){\line(0,1){9}}
              \put(8,18){\line(1,0){9}}
              \put(17,1){\line(0,1){17}}
\thinlines
              \put(0.5,2){\line(1,0){7}}
              \put(0.5,4){\line(1,0){7}}
              \put(0.5,6){\line(1,0){7}}
              \put(9.5,6){\line(1,0){7}}
              \put(9.5,8){\line(1,0){7}}
              \put(0.5,12){\line(1,0){7}}
              \put(0.5,8){\line(1,0){7}}
              \put(9.5,4){\line(1,0){7}}
              \put(9.5,2){\line(1,0){7}}
\color{green}
\put(-0.5,11){\makebox(0,0){$1p0f_{5/2}0g_{9/2}$}}
\put(-2,8){\makebox(0,0){$0f_{7/2}$}}
\put(-2,6){\makebox(0,0){$1s0d$}}
\put(-2,4){\makebox(0,0){$0p$}}
\put(-2,2){\makebox(0,0){$0s$}}

\color{red}
\put(3,19){\makebox(0,0){$\pi$--protons}}
\put(3,2){\circle*{0.3}}
\put(5,2){\circle*{0.3}}
\put(1.5,4){\circle*{0.3}}
\put(2.5,4){\circle*{0.3}}
\put(3.5,4){\circle*{0.3}}
\put(4.5,4){\circle*{0.3}}
\put(5.5,4){\circle*{0.3}}
%\put(4,12){\circle*{0.3}}
\put(6.5,4){\circle*{0.3}}
\put(1,6){\circle*{0.3}}
\put(1.5,6){\circle*{0.3}}
\put(2,6){\circle*{0.3}}
\put(2.5,6){\circle*{0.3}}
\put(3,6){\circle*{0.3}}
\put(3.5,6){\circle*{0.3}}
\put(4,6){\circle*{0.3}}
\put(4.5,6){\circle*{0.3}}
\put(5,6){\circle*{0.3}}
\put(5.5,6){\circle*{0.3}}
\put(6,6){\circle*{0.3}}
\put(6.5,6){\circle*{0.3}}
\put(2,8){\circle*{0.3}}
\put(2.5,8){\circle*{0.3}}
\put(3,8){\circle*{0.3}}
\put(3.5,8){\circle*{0.3}}
\put(4,8){\circle*{0.3}}
\put(4.5,8){\circle*{0.3}}
\put(5,8){\circle*{0.3}}
\put(5.5,8){\circle*{0.3}}


\color{blue}
\put(9.5,10){\line(1,0){7}}
\put(12,2){\circle*{0.3}}
\put(14,2){\circle*{0.3}}
\put(10.5,4){\circle*{0.3}}
\put(11.5,4){\circle*{0.3}}
\put(12.5,4){\circle*{0.3}}
\put(13.5,4){\circle*{0.3}}
\put(14.5,4){\circle*{0.3}}
\put(15.5,4){\circle*{0.3}}
\put(10,6){\circle*{0.3}}
\put(10.5,6){\circle*{0.3}}
\put(11,6){\circle*{0.3}}
\put(11.5,6){\circle*{0.3}}
\put(12,6){\circle*{0.3}}
\put(12.5,6){\circle*{0.3}}
\put(13,6){\circle*{0.3}}
\put(13.5,6){\circle*{0.3}}
\put(14,6){\circle*{0.3}}
\put(14.5,6){\circle*{0.3}}
\put(15,6){\circle*{0.3}}
\put(15.5,6){\circle*{0.3}}

\put(11,8){\circle*{0.3}}
\put(11.5,8){\circle*{0.3}}
\put(12,8){\circle*{0.3}}
\put(12.5,8){\circle*{0.3}}
\put(13,8){\circle*{0.3}}
\put(13.5,8){\circle*{0.3}}
\put(14,8){\circle*{0.3}}
\put(14.5,8){\circle*{0.3}}
\put(12,19){\makebox(0,0){$\nu$--neutrons}}
              \put(9.5,10){\line(1,0){7}}
\put(13,11){\makebox(0,0){$\nu 1p_{3/2}$}}
\put(10.5,10){\circle*{0.3}}
\put(12.25,10){\circle*{0.3}}
\put(14,10){\circle*{0.3}}
\put(15.75,10){\circle*{0.3}}


\put(13,13){\makebox(0,0){$\nu 1p_{1/2}$}}
              \put(9.5,12){\line(1,0){7}}
\put(12,12){\circle*{0.3}}
\put(14,12){\circle*{0.3}}

              \put(9.5,14){\line(1,0){7}}
\put(13,15){\makebox(0,0){$\nu 0f_{5/2}$}}
\put(10.5,14){\circle*{0.3}}
\put(11.5,14){\circle*{0.3}}
\put(12.5,14){\circle*{0.3}}
\put(13.5,14){\circle*{0.3}}
\put(14.5,14){\circle*{0.3}}
\put(15.5,14){\circle*{0.3}}
\pause
\put(13,17){\makebox(0,0){$\nu 0g_{9/2}$}}
              \put(9.5,16){\line(1,0){7}}
\put(10.5,16){\circle*{0.3}}
\put(11,16){\circle*{0.3}}
\put(11.5,16){\circle*{0.3}}
\put(12,16){\circle*{0.3}}
\put(12.5,16){\circle*{0.3}}
\put(13,16){\circle*{0.3}}
\put(13.5,16){\circle*{0.3}}
\put(14,16){\circle*{0.3}}
\put(14.5,16){\circle*{0.3}}
\put(15,16){\circle*{0.3}}
         }}
\end{picture}
      \end{center}
\end{figure}
	%\includegraphics[width=1.2\textwidth]{snnickel.pdf}
	%\includegraphics[width=1.2\textwidth]{gapnickel.pdf}
\begin{enumerate}
\item $^{137}$Sn is the last reported neutron-rich isotope (with half-life).
\item To understand which parts of the nuclear Hamiltonian that drives the
properties of such nuclei will be crucial for our understanding of the stability of matter.
\item Zr isotopes form also long chains of neutron-rich isotopes. {\bf FRIB plans from $^{80}$Zr to
$^{120}$Zr.}
\end{enumerate}
%	%\includegraphics[width=1.2\textwidth]{sntin.pdf}
%	%\includegraphics[width=1.2\textwidth]{gaptin.pdf}
%	%\includegraphics[width=1.2\textwidth]{snlead.pdf}
%	%\includegraphics[width=1.2\textwidth]{gaplead.pdf}
Since we will focus in the beginning on single-particle degrees of freedom and mean-field approaches before we
start with nuclear forces and many-body approaches like the nuclear shell-model, there are some features to be noted
\begin{enumerate}
\item  The total binding energy is not that different from the sum of the individual neutron and proton masses. 
One may thus infer that intrinsic properties of nucleons in a nucleus are close to those of free nucleons.
\item We note clearly a staggering effect between odd and even isotopes with the even ones being more bound (larger separation energies). We will later link this to strong pairing correlations in nuclei.
\item We note also that there are large shell-gaps for some nuclei, meaning that more energy is needed to remove one nucleon. These gaps are used to define so-called magic numbers. 
\end{enumerate}

The root-mean-square (rms) charge radius has been measured for the ground states of many
nuclei. For a spherical charge density, $\rho({\bf r})$, the mean-square radius is defined by:
\[
\langle r^2\rangle = \frac{ \int  d {\bf r} \rho({\bf r}) r^2}{ \int  d {\bf r} \rho({\bf r})},
\]
and the rms radius is the square root of this quantity denoted by
\[
R =\sqrt{ \langle r^2\rangle}.
\]
Radii for most stable
nuclei have been deduced from electron scattering form
factors and/or from the x-ray transition energies of muonic atoms. 
The relative radii for a
series of isotopes can be extracted from the isotope shifts of atomic x-ray transitions.
The rms radius for the nuclear point-proton density, $R_p$ is obtained from the rms charge radius by:
\[
R_p = \sqrt{R^2_{\mathrm{ch}}- R^2_{\mathrm{corr}}},
\]
where
\[
R^2_{\mathrm{corr}}= R^2_{\mathrm{op}}+(N/Z)R^2_{\mathrm{on}}+R^2_{\mathrm{rel}},
\]
where $ R_{\mathrm{op}}= 0.875(7)$ fm  is the rms radius of the proton, $R^2_{\mathrm{on}} = 0.116(2)$ fm$^2$ is the
mean-square radius of the neutron and $R^2_{\mathrm{rel}} = 0.033$ fm$^2$ is the relativistic Darwin-Foldy correction. There are also smaller nucleus-dependent relativistic spin-orbit and
mesonic-exchange corrections that should be included.





\subsection{Many-body Schr\"odinger equation}



The non-relativistic Schr\"odinger equation reads 
\begin{equation}
\hat{H}(x_1, x_2, \hdots , x_A) \Psi_{\lambda}(x_1, x_2, \dots , x_A) = 
E_\lambda  \Psi_\lambda(x_1, x_2, \hdots , x_A), 
\label{eq:basicSE1}
\end{equation}
where the vector $x_i$ represents the coordinates (spatial and spin) of particle $i$, $\lambda$ stands  for all the quantum
numbers needed to classify a given $A$-particle state and $\Psi_{\lambda}$ is the pertaining eigenfunction.  

We write the Hamilton operator, or Hamiltonian,  in a generic way 
\[
	\hat{H} = \hat{T} + \hat{V} 
\]
where $\hat{T}$  represents the kinetic energy of the system
\[
	\hat{T} = \sum_{i=1}^A \frac{\mathbf{p}_i^2}{2m_i} = \sum_{i=1}^A \left( -\frac{\hbar^2}{2m_i} \mathbf{\nabla_i}^2 \right) =
		\sum_{i=1}^A \hat{t}(x_i)
\]
while the operator $\hat{V}$ for the potential energy is given by
\begin{equation}
	\hat{V} = \sum_{i=1}^A \hat{u}_{\mathrm{ext}}(x_i) + \sum_{ji=1}^A \hat{v}(x_i,x_j)+\sum_{ijk=1}^A\hat{v}(x_i,x_j,x_k)+\dots
\label{eq:firstv}
\end{equation}
In the last equation we have singled out an external one-body
potential term $\hat{u}_{\mathrm{ext}}$ which is meant to represent an
effective onebody field in which our particles move. We have therefore
assumed that a picture consisting of individual fermions is a viable
starting point for wave function approximations.  We will specify this
potential later when we need to introduce a calculational basis for
the single-particle states. We have also hinted at the possibility
that our interaction can have three-body terms or even more
complicated many terms.  In atomic, molecular and solid state physics
we normally assume that two-body interactions are sufficient to
describe the system. This is not the case in nuclear physics.

Much of the many-body formalism we will develop, can, with appropriate
modifications, be applied to other fields and disciplines in physics.
If one does quantum chemistry, after having introduced the
Born-Oppenheimer approximation which effectively freezes out the
nucleonic degrees of freedom, the Hamiltonian for $N=n_e$ electrons
takes the following form
\[
  \hat{H} = \sum_{i=1}^{n_e} t(x_i) 
  - \sum_{i=1}^{n_e} k\frac{Z}{r_i} + \sum_{i<j}^{n_e} \frac{k}{r_{ij}},
\]
with $k=1.44$ eVnm\footnote{For nuclei we will replace the units eVnm with MeVfm.} and $n_e$ being the number of electrons. We can rewrite this as
as
\begin{equation}
    \hat{H} = \hat{H_0} + \hat{H_I} 
    = \sum_{i=1}^{n_e}\hat{h_i} + \sum_{i<j=1}^{n_e}\frac{1}{r_{ij}},
\label{H1H2}
\end{equation}
where  we have defined $r_{ij}=| x_i-x_j|$ and
\begin{equation}
  \hat{h_i} =  t(x_i) - \frac{Z}{r_i}.
\label{hi}
\end{equation}
The first term of eq.~(\ref{H1H2}), $H_0$, is the sum of the $N$ or $n_e$
\emph{one-body} Hamiltonians $\hat{h_i}$. Each individual
Hamiltonian $\hat{h_i}$ contains the kinetic energy operator of an
electron and its potential energy due to the attraction of the
nucleus. The second term, $H_I$, is the sum of the $n_e(n_e-1)/2$
two-body interactions between each pair of electrons. Note that the double sum carries a restriction $i<j$.

The potential energy term due to the attraction of the nucleus defines
the onebody field $\hat{u}_i=\hat{u}_{\mathrm{ext}}(x_i)$ of
Eq.~(\ref{eq:firstv}).  We have moved this term into the $\hat{H}_0$
part of the Hamiltonian, instead of keeping it in $\hat{V}$ as in
Eq.~(\ref{eq:firstv}).  The reason is that we will hereafter treat
$\hat{H}_0$ as our non-interacting Hamiltonian. For a many-body
wavefunction $\Phi_{\lambda}$ defined by an appropriate
single-particle basis, we may solve exactly the non-interacting
eigenvalue problem
\[
\hat{H}_0\Phi_{\lambda}= e_{\lambda}\Phi_{\lambda},
\]
with $e_{\lambda}$ being the non-interacting energy. This energy is defined by the sum over single-particle energies to be defined below.
For atoms the single-particle energies could be the hydrogen-like single-particle energies corrected for the charge $Z$. For nuclei and quantum
dots, these energies could be given by the harmonic oscillator in three and two dimensions, respectively.

If we switch to nuclei we no longer have well-defined potential energy  terms (in the strict sense of a potential) and we have to revert
to interaction models.   In this chapter, and in many others as well, we will assume that the interacting part of the Hamiltonian
can be approximated by a two-body interaction model, normally based  on field-thereotical models with selected baryons and mesons.  
This means that our Hamiltonian is written as 
\begin{equation}
    \hat{H} = \hat{H_0} + \hat{H_1} 
    = \sum_{i=1}^A h_i + \sum_{i<j=1}^A V(r_{ij}),
\label{Hnuclei}
\end{equation}
with 
\begin{equation}
  H_0=\sum_{i=1}^A h_i =  \sum_{i=1}^A\left(t(x_i) + u(x_i)\right).
\label{hinuclei}
\end{equation}
The onebody part $u(x_i)$ is normally approximated by a harmonic oscillator potential. However, other potentials are fully possible, such as 
one derived from the self-consistent solution of the Hartree-Fock equations.

For quantum dots, the onebody part is also approximated with a harmonic oscillator potential, either two-dimensional 
or three-dimensional, while the 
interaction term is the standard Coulomb interaction.

Irrespective of these approximations, there is a wealth of experimental evidence that these interactions have to obey specific symmetries. 
The total Hamiltonian should be translationally invariant. If angular momentum is conserved, 
the Hamiltonian is invariant under rotations. Furthermore,
it is invariant under the interchange of two particles and invariant under time reversal and space reflections.  
This means that our Hamiltonian commutes with the respective operators


\subsection{Single-particle basis and computational aspects}

The one-body part $u_{\mathrm{ext}}(x_i)$ is normally approximated by a harmonic oscillator potential or the Coulomb interaction an electron feels from the nucleus. However, other potentials are fully possible, such as 
one derived from the self-consistent solution of the Hartree-Fock equations or so-called Woods-Saxon potentials.

We have defined
\[
    \hat{H} = \hat{H_0} + \hat{H_I} 
    = \sum_{i=1}^A \hat{h}_0(x_i) + \sum_{i<j=1}^A \hat{v}(x_{ij}),
\]
with 
\[
  H_0=\sum_{i=1}^A \hat{h}_0(x_i) =  \sum_{i=1}^A\left(\hat{t}(x_i) + \hat{u}_{\mathrm{ext}}(x_i)\right).
\]

In nuclear physics the one-body part $u_{\mathrm{ext}}(x_i)$ is often approximated by a harmonic oscillator potential or a
Woods-Saxon potential. However, this is not fully correct, because as we have discussed, nuclei are self-bound systems and there is no external confining potential. {\bf The Hamiltonian $H_0$ cannot be used to compute the binding energy of a nucleus since it is not based on a model for the nuclear forces.}. That is, the binding energy is not the sum of the individual single-particle energies. 

The Woods-Saxon potential is a mean field potential for the nucleons (protons and neutrons) 
inside an atomic nucleus. It represent an average potential that a given nucleon feels from  the forces applied on each nucleon. 
The parametrization is
\[
\hat{u}_{\mathrm{ext}}(r)=-\frac{V_0}{1+\exp{(r-R)/a}},
\]
with $V_0\approx 50$ MeV representing the potential well depth, $a\approx 0.5$ fm 
length representing the "surface thickness" of the nucleus and $R=r_0A^{1/3}$, with $r_0=1.25$ fm and $A$ the number of nucleons.
The value for $r_0$ can be extracted from a fit to data, see for example M.~Kirson, Nucl.~Phys.~A {\bf 781}, 350 (2007).
\begin{itemize}
\item It rapidly approaches zero as $r$ goes to infinity, reflecting the short-distance nature of the strong nuclear force.
\item For large $A$, it is approximately flat in the center.
\item Nucleons near the surface of the nucleus experience a large force towards the center.
\end{itemize}
%	%\includegraphics[width=1.25\textwidth]{woodsaxon.pdf}
We have defined
\[
    \hat{H} = \hat{H_0} + \hat{H_I} 
    = \sum_{i=1}^A \hat{h}_0(x_i) + \sum_{i<j=1}^A \hat{v}(x_{ij}),
\]
with 
\[
  H_0=\sum_{i=1}^A \hat{h}_0(x_i) =  \sum_{i=1}^A\left(\hat{t}(x_i) + \hat{u}_{\mathrm{ext}}(x_i)\right).
\]
As stated in previous slides, 
in nuclear physics the one-body part $u_{\mathrm{ext}}(x_i)$ is often 
approximated by a harmonic oscillator potential. However,  as we also noted with the Woods-Saxon potential there is no 
external confining potential in nuclei. 
What many people do then, is to add and subtract a harmonic oscillator potential,
with 
\[
\hat{u}_{\mathrm{ext}}(x_i)=\hat{u}_{\mathrm{ho}}(x_i)= \frac{1}{2}m\omega^2 r_i^2,
\]
where $\omega$ is the oscillator frequency. This leads to 
\[
    \hat{H} = \hat{H_0} + \hat{H_I} 
    = \sum_{i=1}^A \hat{h}_0(x_i) + \sum_{i<j=1}^A \hat{v}(x_{ij})-\sum_{i=1}^A\hat{u}_{\mathrm{ho}}(x_i),
\]
with 
\[
  H_0=\sum_{i=1}^A \hat{h}_0(x_i) =  \sum_{i=1}^A\left(\hat{t}(x_i) + \hat{u}_{\mathrm{ho}}(x_i)\right).
\]











We have introduced a single-particle Hamiltonian
\[
  H_0=\sum_{i=1}^A \hat{h}_0(x_i) =  \sum_{i=1}^A\left(\hat{t}(x_i) + \hat{u}_{\mathrm{ext}}(x_i)\right),
\]
with an external and central symmetric potential $u_{\mathrm{ext}}(x_i)$, which is often 
approximated by a harmonic oscillator potential or a Woods-Saxon potential. Being central symmetric leads to a degeneracy 
in energy which is not observed experimentally. We see this from for example our discussion of separation energies and magic numbers. There are, in addition to the assumed magic numbers from a harmonic oscillator basis of $2,8,20,40,70\dots$ magic numbers like $28$, $50$, $82$ and $126$. 

To produce these additional numbers, we need to add a phenomenological spin-orbit force which lifts the degeneracy, that is
\[
\hat{h}(x_i) =  \hat{t}(x_i) + \hat{u}_{\mathrm{ext}}(x_i) +\xi({\bf r}){\bf ls}=\hat{h}_0(x_i)+\xi({\bf r}){\bf ls}. 
\]
%            %\includegraphics[scale=0.35]{graphics/singleparticle.png}
We have introduced a modified single-particle Hamiltonian
\[
\hat{h}(x_i) =  \hat{t}(x_i) + \hat{u}_{\mathrm{ext}}(x_i) +\xi({\bf r}){\bf ls}=\hat{h}_0(x_i)+\xi({\bf r}){\bf ls}. 
\]
We can calculate the expectation value of the latter using the fact that
\[
\xi({\bf r}){\bf ls}=\frac{1}{2}\xi({\bf r})\left({\bf j}^2-{\bf l}^2-{\bf s}^2\right).
\]
For a single-particle state with quantum numbers $nlj$ (we suppress $s$ and $m_j$), with $s=1/2$ and ${\bf j=l}\pm {\bf s}$, we obtain
the single-particle energies
\[
\varepsilon_{nlj} = \varepsilon_{nlj}^{(0)}+\Delta\varepsilon_{nlj}, 
\]
with $\varepsilon_{nlj}^{(0)}$ being the single-particle energy obtained with $\hat{h}_0(x)$ and 
\[
\Delta\varepsilon_{nlj}=\frac{C}{2}\left(j(j+1)-l(l+1)-\frac{3}{4}\right).
\]
The spin-orbit force gives thus an additional contribution to the energy
\[
\Delta\varepsilon_{nlj}=\frac{C}{2}\left(j(j+1)-l(l+1)-\frac{3}{4}\right),
\]
which lifts the degeneracy we have seen before in the harmonic oscillator or Woods-Saxon potentials. The value $C$ is the radial
integral involving $\xi({\bf r})$. Depending on the value of $j=l\pm 1/2$, we obtain 
\[
\Delta\varepsilon_{nlj=l-1/2}=\frac{C}{2}l,
\]
or
\[
\Delta\varepsilon_{nlj=l+1/2}=-\frac{C}{2}(l+1),
\]
clearly lifting the degeneracy. Note well that till now we have simply postulated the spin-orbit force in {\em ad hoc} way.
Later, we will see how this term arises from the two-nucleon force in a natural way. 
With the spin-orbit force, we can modify our Woods-Saxon potential to 
\[
\hat{u}_{\mathrm{ext}}(r)=-\frac{V_0}{1+\exp{(r-R)/a}}+V_{so}(r){\bf ls},
\]
with 
\[
V_{so}(r) = V_{so}\frac{1}{r}\frac{d f_{so}(r)}{dr},
\]
where we have 
\[
f_{so}(r) = \frac{1}{1+\exp{(r-R_{so})/a_{so}}}.
\]
We can also add, in case of proton, a Coulomb potential. The
Woods-Saxon potential has been widely used in parametrizations of
effective single-particle potentials. {\bf However, as was the case
  with the harmonic oscillator, none of these potentials are linked
  directly to the nuclear forces}. Our next step is to build a mean
field based on the nucleon-nucleon interaction.  This will lead us to
our first and simplest many-body theory, Hartree-Fock theory.



\section{Exercises}
\begin{Exercise}
\begin{enumerate}
\item[a)] 
\end{enumerate}
\end{Exercise}
\begin{Exercise}

\begin{enumerate}
\item[a)] 
\end{enumerate}
\end{Exercise}
