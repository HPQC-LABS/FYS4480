\chapter{Definitions and examples of physical systems}\label{chap:basics}


\section{Introduction}\label{sec:basics_intro}

Throughout this text we will mainly limit ourselves to systems of interacting fermions whose properties
can be extracted from studies of the non-relativistic Schr\"odinger  equation. In our discussion 
of nuclear interactions we will however need some details about relativistic quantum mechanics as well.

We will mainly stay within a non-relativistic regime, meaning that Schr\"odinger's equation for interacting fermions 
will be our most frequent starting point. Our particles are identical and indistiguishable.
  


\section{Definitions}


An operator is defined as $\hat{O}$ throughout this text. Unless otherwise
specified the number of particles is always $N$ and $d$ is the dimension of the 
system. 
In nuclear physics we normally define the total number of particles to be $A=N+Z$,
where $N$ is total number of neutrons and $Z$ the total number of protons. In case of other baryons such isobars $\Delta$ or
various hyperons such as $\Lambda$ or $\Sigma$, one needs to add their definitions.  
%
Hereafter, $N$ is reserved for the total number of particles, unless otherwise specificied.

The quantum numbers of a single-particle state in coordinate space are
defined by the variable $x=({\bf r},\sigma)$, where ${\bf r}\in {\mathbb{R}}^{d}$with $d=1,2,3$ represents 
the spatial coordinates and $\sigma$ is the eigenspin of the particle. For fermions with eigenspin $1/2$ this means that
\[
 x\in {\mathbb{R}}^{d}\oplus (\frac{1}{2}),
\]
and the integral
\[
\int dx = \sum_{\sigma}\int d^dr = \sum_{\sigma}\int d{\bf r},
\]
and
\[
\int d^Nx= \int dx_1\int dx_2\dots\int dx_N.
\]

The quantum mechanical wave function of a given state with quantum numbers $\lambda$ (encompassing all quantum numbers needed to specify the system), ignoring time, is
\[
\Psi_{\lambda}=\Psi_{\lambda}(x_1,x_2,\dots,x_N),
\]
with $x_i=({\bf r}_i,\sigma_i)$ and the projection of $\sigma_i$ takes the values
$\{-1/2,+1/2\}$ for particles with spin $1/2$. 
We will hereafter always refer to $\Psi_{\lambda}$ as the exact wave function, and if the ground state is not degenerate we label it as 
\[
\Psi_0=\Psi_0(x_1,x_2,\dots,x_N).
\]

Since the solution $\Psi_{\lambda}$ seldomly can be found in closed form, approximations are sought. In this text we define an approximative wave function or an ansatz to the exact wave function as 
\[
\Phi_{\lambda}=\Phi_{\lambda}(x_1,x_2,\dots,x_N),
\]
with 
\[
\Phi_0=\Phi_0(x_1,x_2,\dots,x_N),
\]
being the ansatz to the ground state.  

The wave function $\Psi_{\lambda}$ is sought in the Hilbert space of either symmetric or anti-symmetric $N$-body functions, namely
\[
\Psi_{\lambda}\in {\cal H}_N:= {\cal H}_1\oplus{\cal H}_1\oplus\dots\oplus{\cal H}_1,
\]
where the single-particle Hilbert space ${\cal H}_1$ is the space of square integrable functions over
$\in {\mathbb{R}}^{d}\oplus (\sigma)$
resulting in
\[
{\cal H}_1:= L^2(\mathbb{R}^{d}\oplus (\sigma)).
\]


\subsection{Non-iteracting systems and wave functions}

We consider an isolated system of $N$ identical particles that we assume can be treated non-relativistic. The properties of the system are given by the fundamental Schr�dinger equation,
\begin{align}
\label{eq:Many-body timedependent Schr�dinger equation}
i\hbar \frac{\partial}{\partial t}\ket{\Psi} = \OP{H}\ket{\Psi},
\end{align}
where $\ket{\Psi}$ is the $N$-particle wave function and $\OP{H}$ is the hamiltonian of the system defined as
\begin{align}
\label{exp:general hamilton operator}
\OP{H} = \OP{T} + \OP{V}.
\end{align}
The operators $\OP{T}$ and $\OP{V}$ denotes the kinetic energy and potential energy of the system, respectively. The time dependent Schr�dinger equation (\ref{eq:Many-body timedependent Schr�dinger equation}) provides all the properties of the system including the time evolution. It is appropriate to introduce the time elvolution operator $\OP{\mathcal{U}}$, which acting on a system state $\ket{\Psi(t_0)}$ yields the time evolution of the system at all times $t>t_0$,
\begin{align}
\label{exp:time evolution operator}
\ket{\Psi(t)} = \OP{\mathcal{U}}(t,t_0)\ket{\Psi(t_0)}.
\end{align}
Inserting (\ref{exp:time evolution operator}) into Eq.(\ref{eq:Many-body timedependent Schr�dinger equation}) yields the equation for the time-evolution operator,
\begin{align}
\label{eq:time evolution operator equation}
i\hbar\frac{\partial}{\partial t}\OP{\mathcal{U}}(t,t_0) = \OP{H}\OP{\mathcal{U}}(t,t_0).
\end{align}
When the hamiltonian is time independent, the time evolution operator reads
\begin{align}
\label{exp:time evolution operator 2}
\OP{\mathcal{U}}(t,t_0) = e^{-\frac{i}{\hbar}\OP{H}(t-t_0)}.
\end{align}
If we were to prepare a system in state $\ket{\Psi(t_0)}$ at time $t_0$, which for convenience often is chosen as $t_0=0$, the system state $\ket{\Psi(t)}$ for all $t>t_0$ is determined by simply letting the time evolution operator act on the initial state. It is already clear that the hamiltonian of the system determine the time evolution of the system, which in itself is a fundamental improtance, but it also provide the energy spectrum and eigenstates which are often our main interest. For example, an analytical expression of Eq. (\ref{exp:time evolution operator 2}) is only possible if the reference state is written as a linear combination of energy eigenstates. 

The general non-interacting (unperturbed) system consists of $N$ non-interacting particles subject to an external potential $u$. The hamiltonian reads
\begin{align}
\label{exp:Non-interacting hamiltonian}
\OP{H}_0 = \sum_{i=1}^N \OP{h}_0 = -\frac{\hbar^2}{2m}\sum_{i=1}^N \nabla_i^2 + \sum_{i=1}^N u(\vek{r}_i),
\end{align}
where the potential is assumed to be a function of spatial coordinates $\vek{r}_i$ including spin variable. The non-interacting Schr�dinger equation
\begin{align}
\label{eq:Non-interacting schr�dinger equation}
\OP{H}_0\Phi_a(\vek{r}_1,\vek{r}_2,..,\vek{r}_N) = E_a^0 \Phi_a(\vek{r}_1,\vek{r}_2,..,\vek{r}_N),
\end{align}
can be solved exactly by seperation of variables since the hamiltonian only consists of one-body operators. Inserting
\begin{align}
\label{exp:Non-interacting energy eigenvectors}
\Phi_a(\vek{r}_1,\vek{r}_2,..,\vek{r}_N) = \phi_\alpha(\vek{r}_1)\phi_\beta(\vek{r}_2)..\phi_\gamma(\vek{r}_N)
\end{align}
into Eq. (\ref{eq:Non-interacting schr�dinger equation}) yields $N$ identical single-particle equations, 
\begin{align}
\label{eq:Single-particle Schr�dinger equation}
\OP{h}_0 \phi(\vek{r}) = \epsilon \phi(\vek{r}).
\end{align}
We identify Eq. (\ref{eq:Single-particle Schr�dinger equation}) as the Schr�dinger equation for the system when only one particle is present. Solving this equation yields the set of single-particle energy eigenvalues $\kpr{\epsilon}$  and eigenfunctions $\kpr{\phi}$, and hence all the $N$-particle eigenfunctions in Eq. (\ref{exp:Non-interacting energy eigenvectors}) are determined. The total non-interacting energy eigenvalues are found by summing over all the single-particle energies,
\begin{align}
E_a^0 = \sum_{\alpha=1}^N \epsilon_{\alpha}.
\end{align}
A general state in the non-interacting system can be written as the linear combination of energy eigenstates,
\begin{align}
\Phi(\vek{r}_1,\vek{r}_2,..,\vek{r}_N) = \sum_{\kpr{\alpha\beta..\gamma}} c_{\alpha\beta..\gamma} \phi_\alpha(\vek{r}_1)\phi_\beta(\vek{r}_2)..\phi_\gamma(\vek{r}_N),
\end{align}
where $c_{\alpha\beta..\gamma}$ are the expansion coefficients satisfying the normalization condition
\begin{align}
\sum_{\kpr{\alpha\beta..\gamma}} \abs{c_{\alpha\beta..\gamma}}^2 = 1. 
\end{align}


The basis functions for the symmetric $N$-particle Hilbert space and the antisymmetric $N$-particle Hilbert space can be constructed from an orthonormal and colomplete single-particle basis set. As was the case for distinguishable particles, the direct product states in Eq. (\ref{exp:Direct product states}) do not constitute a basis for the symmetric and antisymmetric subspaces. As can be done for all wave functions, the bosonic and fermionic states can be written as a linear combination of a (at the moment unknown) basis set,
\begin{align}
\ket{\Psi_B} &= \sum_i c_i \ket{i}\\
\ket{\Psi_F} &= \sum_j c_j \ket{j}.
\end{align}
To have $\ket{\Psi_S}$ and $\ket{\Psi_A}$ symmetric and antisymmetric, respectively, require the basis functions $\ket{i}$ to be symmetric and $\ket{j}$ to be antisymmetric. Since the product states are not all symmetric and antisymmetric, they cannot be used as basis functions. In section \ref{sec: Identical particles}, symmetric and antisymmetric states were constructed from the product states, see Eq. (\ref{exp:Symmetrized energy eigenfunctions}) and Eq. (\ref{exp:Antisymetrized energy eigenfunctions}). $\textit{These}$ functions have the correct symmetry in addition to inherit the orthonormality from the product states. Hence, 
\begin{align}
\ket{i} &=  \frac{1}{\sqrt{n_\alpha!n_\beta!..n_\gamma!}}\OP{\mathcal{S}} \ket{\alpha\beta..\gamma}\\
\ket{j} &= \OP{\mathcal{A}} \phi_\alpha(\vek{r}_1) \ket{\alpha\beta..\gamma},
\end{align}
where the symmetrizer operators are defined in Eq. (\ref{exp:Symmetrizer operator}) and Eq. (\ref{exp:Antisymmetrizer operator}). $\textit{Any}$ orthonormal and complete single-particle basis can be used. Nevertheless, as was the case for distinguishable particles, the $\textit{good}$ single-particle basis is often the non-interacting set. Written in the coordinate representation, the bosonic and fermionic wave function reads
\begin{align}
\Psi_B(\vek{r}_1,\vek{r}_2,..,\vek{r}_N) &=  \sum_{\kpr{\alpha\beta..\gamma}} \frac{c_{\alpha\beta..\gamma}}{\sqrt{n_\alpha!n_\beta!..n_\gamma!}}\OP{\mathcal{S}} \phi_\alpha(\vek{r}_1) \phi_\beta(\vek{r}_2)..\phi_\gamma(\vek{r}_N)\\
\label{exp:Fermionic wave function}
\Psi_{F}(\vek{r}_1,\vek{r}_2,..,\vek{r}_N) &= \sum_{\kpr{\alpha\beta..\gamma}} d_{\alpha\beta..\gamma}\OP{\mathcal{A}} \phi_\alpha(\vek{r}_1) \phi_\beta(\vek{r}_2)..\phi_\gamma(\vek{r}_N).
\end{align}
The fermionic basis functions in Eq. (\ref{exp:Fermionic wave function}) can be written as a determinant, called $\textit{Slater determinant}$,
\begin{align}
\label{exp: Slater determinant}
\Phi_{\alpha\beta..\gamma}(\vek{r}_1,\vek{r}_2,..,\vek{r}_N) = \frac{1}{\sqrt{N!}}\left|\begin{array}{cccc}\phi_\alpha(\vek{r}_1) & \phi_\beta(\vek{r}_1) & \cdots & \phi_\gamma(\vek{r}_1)\\
\phi_\alpha(\vek{r}_2) & \phi_\beta(\vek{r}_2) & \cdots & \phi_\gamma(\vek{r}_2) \\ \vdots & \vdots & \vdots & \vdots\\ 
\phi_\alpha(\vek{r}_N) & \phi_\beta(\vek{r}_N) & \cdots & \phi_\gamma(\vek{r}_N)\end{array} \right|,
\end{align}
where $\Phi$ and $\phi$ are used to indicate that we are working with the good basis, i.e. the non-interacting set. But as pointed out before, every set of single-particle functions satisfying basis requirements, can be used. A general fermionic wave function can thus be expanded as an infinite series of Slater determinants,
\begin{align}
\Psi_{F}(\vek{r}_1,\vek{r}_2,..,\vek{r}_N) &= \sum_{\kpr{\alpha\beta..\gamma}} d_{\alpha\beta..\gamma} \Phi_{\alpha\beta..\gamma}(\vek{r}_1,\vek{r}_2,..,\vek{r}_N).
\end{align}



\subsection{Many-body Schr\"odinger equation}



The Schr\"odinger equation reads 
\begin{equation}
\hat{H}(x_1, x_2, \hdots , x_N) \Psi_{\lambda}(x_1, x_2, \dots , x_N) = 
E_\lambda  \Psi_\lambda(x_1, x_2, \hdots , x_N), 
\label{eq:basicSE1}
\end{equation}
where the vector $x_i$ represents the coordinates (spatial and spin) of particle $i$, $\lambda$ stands  for all the quantum
numbers needed to classify a given $N$-particle state and $\Psi_{\lambda}$ is the pertaining eigenfunction.  

We write the Hamilton operator, or Hamiltonian,  in a generical way 
\[
	\hat{H} = \hat{T} + \hat{V} 
\]
where $\hat{T}$  represents the kinetic energy of the system
\[
	\hat{T} = \sum_{i=1}^N \frac{\mathbf{p}_i^2}{2m_i} = \sum_{i=1}^N \left( -\frac{\hbar^2}{2m_i} \mathbf{\nabla_i}^2 \right) =
		\sum_{i=1}^N \hat{t}(x_i)
\]
while the operator $\hat{V}$ for the potential energy is given by
\begin{equation}
	\hat{V} = \sum_{i=1}^N \hat{u}_{\mathrm{ext}}(x_i) + \sum_{ji=1}^N \hat{v}(x_i,x_j)+\sum_{ijk=1}^N\hat{v}(x_i,x_j,x_k)+\dots
\label{eq:firstv}
\end{equation}
Hereafter we use natural units, viz.~$\hbar=c=e=1$, with $e$ the elementary chargeand $c$ the speed of light. This means that momenta and masses
have dimension energy. Our codes use MeV as units for energy, but they could obviously be rewritten in dimensionless units. 
In the last equation 
we have singled out an external one-body potential term $\hat{u}_{\mathrm{ext}}$ which is meant to represent an effective
onebody field in which our particles move. We have therefore assumed that a picture consisting of individual fermions 
is a viable starting point for wave function approximations.  We will specify this potential later when we need to introduce a calculational
basis for the single-particle states. We have also hinted at the possibility that our
interaction can have three-body terms or even more complicated many terms.  In atomic, molecular and solid state physics we normally assume that two-body interactions are sufficient to describe the system. This is not the case in nuclear physics.

Much of the many-body formalism we will develop,  can, with appropriate modifications, be applied to other fields and disciplines in physics.
If one does quantum chemistry, after having introduced the  Born-Oppenheimer approximation which effectively freezes out the nucleonic degrees
of freedom, the Hamiltonian for $N=n_e$ electrons takes the following form 
\[
  \hat{H} = \sum_{i=1}^{n_e} t(x_i) 
  - \sum_{i=1}^{n_e} k\frac{Z}{r_i} + \sum_{i<j}^{n_e} \frac{k}{r_{ij}},
\]
with $k=1.44$ eVnm\footnote{For nuclei we will replace the units eVnm with MeVfm.} and $n_e$ being the number of electrons. We can rewrite this as
as
\begin{equation}
    \hat{H} = \hat{H_0} + \hat{H_I} 
    = \sum_{i=1}^{n_e}\hat{h_i} + \sum_{i<j=1}^{n_e}\frac{1}{r_{ij}},
\label{H1H2}
\end{equation}
where  we have defined $r_{ij}=| x_i-x_j|$ and
\begin{equation}
  \hat{h_i} =  t(x_i) - \frac{Z}{r_i}.
\label{hi}
\end{equation}
The first term of eq.~(\ref{H1H2}), $H_0$, is the sum of the $N$ or $n_e$
\emph{one-body} Hamiltonians $\hat{h_i}$. Each individual
Hamiltonian $\hat{h_i}$ contains the kinetic energy operator of an
electron and its potential energy due to the attraction of the
nucleus. The second term, $H_I$, is the sum of the $n_e(n_e-1)/2$
two-body interactions between each pair of electrons. Note that the double sum carries a restriction $i<j$.

The potential energy term due to the attraction of the nucleus defines the onebody field $\hat{u}_i=\hat{u}_{\mathrm{ext}}(x_i)$ of Eq.~(\ref{eq:firstv}).
We have moved this term into the $\hat{H}_0$ part of the Hamiltonian, instead of keeping  it in $\hat{V}$ as in  Eq.~(\ref{eq:firstv}).
The reason is that we will hereafter treat $\hat{H}_0$ as our non-interacting  Hamiltonian. For a many-body wavefunction $\Phi_{\lambda}$ defined by an  
appropriate single-particle basis, we may solve exactly the non-interacting eigenvalue problem 
\[
\hat{H}_0\Phi_{\lambda}= e_{\lambda}\Phi_{\lambda},
\]
with $e_{\lambda}$ being the non-interacting energy. This energy is defined by the sum over single-particle energies to be defined below.
For atoms the single-particle energies could be the hydrogen-like single-particle energies corrected for the charge $Z$. For nuclei and quantum
dots, these energies could be given by the harmonic oscillator in three and two dimensions, respectively.

If we switch to nuclei we no longer have well-defined potential energy  terms (in the strict sense of a potential) and we have to revert
to interaction models.   In this chapter, and in many others as well, we will assume that the interacting part of the Hamiltonian
can be approximated by a two-body interaction model, normally based  on field-thereotical models with selected baryons and mesons.  
This means that our Hamiltonian is written as 
\begin{equation}
    \hat{H} = \hat{H_0} + \hat{H_1} 
    = \sum_{i=1}^A h_i + \sum_{i<j=1}^A V(r_{ij}),
\label{Hnuclei}
\end{equation}
with 
\begin{equation}
  H_0=\sum_{i=1}^A h_i =  \sum_{i=1}^A\left(t(x_i) + u(x_i)\right).
\label{hinuclei}
\end{equation}
The onebody part $u(x_i)$ is normally approximated by a harmonic oscillator potential. However, other potentials are fully possible, such as 
one derived from the self-consistent solution of the Hartree-Fock equations.

For quantum dots, the onebody part is also approximated with a harmonic oscillator potential, either two-dimensional 
or three-dimensional, while the 
interaction term is the standard Coulomb interaction.

Irrespective of these approximations, there is a wealth of experimental evidence that these interactions have to obey specific symmetries. 
The total Hamiltonian should be translationally invariant. If angular momentum is conserved, 
the Hamiltonian is invariant under rotations. Furthermore,
it is invariant under the interchange of two particles and invariant under time reversal and space reflections.  
This means that our Hamiltonian commutes with the respective operators





\subsection{Slater determinant and expectation values of the Hamiltonian}

Our Hamiltonian is invariant under the permutation (interchange) of two particles. % (exercise here, prove it)
Since we deal with fermions however, the total wave function is antisymmetric.
Let $\hat{P}$ be an operator which interchanges two particles.
Due to the symmetries we have ascribed to our Hamiltonian, this operator commutes with the total Hamiltonian,
\[
[\hat{H},\hat{P}] = 0,
\]
meaning that $\Psi_{\lambda}(x_1, x_2, \dots , x_N)$ is an eigenfunction of 
$\hat{P}$ as well, that is
\[
\hat{P}_{ij}\Psi_{\lambda}(x_1, x_2, \dots,x_i,\dots,x_j,\dots,x_N)=
\beta\Psi_{\lambda}(x_1, x_2, \dots,x_j,\dots,x_i,\dots,x_N),
\]
where $\beta$ is the eigenvalue of $\hat{P}$. We have introduced the suffix $ij$ in order to indicate that we permute particles $i$ and $j$.
The Pauli principle tells us that the total wave function for a system of fermions
has to be antisymmetric, resulting in the eigenvalue $\beta = -1$.   


The assumption we make now is crucial for the rest of this text. 
There is a wealth of experimental and theoretical data which lend support to the
assumption that nucleons do not loose their individuality in dense
matter.  This is reflected in for example spectroscopic factors for states near the Fermi surface ({\bf add more refs later and deepen
the argument}). Nucleons do exibit much stronger correlations than atoms, but we can still attempt to approximate our 
many-body wave function with the product of single-particle wave functions. Since we assume that our 
Hamiltonian is time-independent, these single-particle wave functions are normally the eigenfunctions of a selected 
onebody Hamiltonian $h_i$ acting on particle $i$.  

In our case we assume that  we can approximate the exact eigenfunction with a Slater determinant\footnote{Recall that we have reserved 
$\Psi$ as labelling for our exact wave function (eigen function, since there is no time-dependence). The Slater determinant is 
only an approximation to the exact solution.}
\be
   \Phi(x_1, x_2,\dots ,x_N,\alpha,\beta,\dots, \sigma)=\frac{1}{\sqrt{A!}}
\left| \begin{array}{ccccc} \psi_{\alpha}(x_1)& \psi_{\alpha}(x_2)& \dots & \dots & \psi_{\alpha}(x_N)\\
                            \psi_{\beta}(x_1)&\psi_{\beta}(x_2)& \dots & \dots & \psi_{\beta}(x_N)\\  
                            \dots & \dots & \dots & \dots & \dots \\
                            \dots & \dots & \dots & \dots & \dots \\
                     \psi_{\sigma}(x_1)&\psi_{\sigma}(x_2)& \dots & \dots & \psi_{\gamma}(x_N)\end{array} \right|, 
\label{HartreeFockDet}
\ee 
where  $x_i$  stand for the coordinates and spin values of a particle $i$ and $\alpha,\beta,\dots, \gamma$ 
are quantum numbers needed to describe remaining quantum numbers.  
The single-particle function $\psi_{\alpha}(x_i)$  are eigenfunctions of the onebody
Hamiltonian $h_i$, that is
\[
h_i=h(x_i)=t(x_i) + u(x_i),
\]
with eigenvalues 
\[
 h_i\psi_{\alpha}(x_i)=t(x_i) + u(x_i)\psi_{\alpha}(x_i)=\varepsilon_{\alpha}\psi_{\alpha}(x_i).
\]
For most practical purposes we will equate $h_i$ with the single-particle Hamiltonian of the harmonic oscillator.
The energies $\varepsilon_{\alpha}$ are the so-called non-interacting single-particle energies, or unperturbed energies. 
The total energy is in this case the sum over all  single-particle energies, if no two-body or more complicated
many-body interactions are present.

As an example, consider the following three-nucleon Slater determinant
For three nucleons we have  the general expression
\be
   \Phi(x_1,x_2,x_3,\alpha,\beta,\gamma)=\frac{1}{\sqrt{3!}}
\left| \begin{array}{ccc} \psi_{\alpha}(x_1)& \psi_{\alpha}(x_2)& \psi_{\alpha}(x_3)\\\psi_{\beta}(x_1)&\psi_{\beta}(x_2)&\psi_{\beta}(x_3)\\\psi_{\gamma}(x_1)&\psi_{\gamma}(x_2)&\psi_{\gamma}(x_3)\end{array} \right|.
\ee 
Computing the determinant gives 
\begin{eqnarray}
\Phi(x_1,x_2,x_3,\alpha,\beta,\gamma)&=
\frac{1}{\sqrt{3!}}\left[
\psi_{\alpha}(x_1)\psi_{\beta}(x_2)\psi_{\gamma}(x_3)+
\psi_{\beta}(x_1)\psi_{\gamma}(x_2)\psi_{\alpha}(x_3)+
\psi_{\gamma}(x_1)\psi_{\alpha}(x_2)\psi_{\beta}(x_3)-\right. \nonumber \\
&\left.\psi_{\gamma}(x_1)\psi_{\beta}(x_2)\psi_{\alpha}(x_3)-
\psi_{\beta}(x_1)\psi_{\alpha}(x_2)\psi_{\gamma}(x_3)-
\psi_{\alpha}(x_1)\psi_{\gamma}(x_2)\psi_{\beta}(x_3)
\right].
\end{eqnarray}
We note again that 
the wave-function is antisymmetric with respect to an
interchange of any two particles, as required by the Pauli principle. For an $A$-body Slater determinant we have thus
(omitting the quantum numbers $\alpha,\beta,\dots,\nu$)
\[
  \Phi(\mathbf{r}_1, \mathbf{r}_2, \dots, \mathbf{r}_i, \dots,
  \mathbf{r}_j, \dots \mathbf{r}_N) = -
  \Phi(\mathbf{r}_1, \mathbf{r}_2, \dots, \mathbf{r}_j, \dots,
  \mathbf{r}_i, \dots \mathbf{r}_N).
\]

%  Exercise, write out the Slater determinant for helium.

Before we proceed with setting up the expectation value of the Hamiltonian with a Slater determinant, 
we feel that a small digression is needed. 
Although we do not know  the exact solution to Schr\"odinger's equation, there are several  constraints from the Hamiltonian
which pose specific  requirements on the trial wave function. 
The single-particle basis we use in setting up the Slater determinant, which is one of many possible trial wave functions,
have to obey the so-called cusp conditions, that is, the trial single-particle wave functions should be constructed to that
eventual divergencies which appear in the one-body part of the Hamiltonian  cancel out.  
In quantum chemistry  this leads to the typical shape $\psi\propto \exp{(-\alpha r)}$ for the single-particle wave functions, with $\alpha$
normally interpreted as  an effective charge. 
This functional form arises due to the requirement that the divergency in the one-particle kinetic energy should cancel the divergency 
coming from the one-body Coulomb term $-Z/r$, with $r$ the absolute value of the distance from the nucleus to a give electron.

When we include the interaction, which typically has a functional form of the type $\pm 1/r_{ij}$,one needs a
wave function proportional to $\exp{(\pm\beta r_{ij})}$, with $\beta$ a parameter, in order to satisfy the cusp condition. 
The total trial wave function $\Psi_T$ ($T$ for trial) takes then the following form
\begin{equation}
\Psi_T(x_1, x_2,\dots ,x_N,\alpha,\beta,\dots, \sigma)=
\Phi(x_1, x_2,\dots ,x_N,\alpha,\beta,\dots, \sigma)\prod_{i<j}^A f(r_{ij}),
\label{eq:trialwf}
\end{equation}
where $f(r_{ij}$ includes terms proportional to $\exp{(\pm\beta r_{ij})}$.  This term is normally called the correlation function (mention
about Jastrow) and carries the information about the interactions.
How the interaction is turned on and how we approximate the total wave 
function is the topic of many-body physics and this text. Throughout this text we will come back in essentially all chapters to
ways of obtained as good as possible trial wave functions.



Let us denote the ground state energy by $E_0$. According to the
variational principle we have
\begin{equation*}
  E_0 \le E[\Phi] = \int \Phi^*\hat{H}\Phi d\mathbf{\tau}
\end{equation*}
where $\Phi$ is a trial function which we assume to be normalized
\begin{equation*}
  \int \Phi^*\Phi d\mathbf{\tau} = 1,
\end{equation*}
where we have used the shorthand $d\mathbf{\tau}=d\mathbf{r}_1d\mathbf{r}_2\dots d\mathbf{r}_N$.
In the Hartree-Fock method the trial function is the Slater
determinant of Eq.~(\ref{HartreeFockDet}) which can be rewritten as 
\begin{equation}
  \Psi(\mathbf{r}_1,\mathbf{r}_2,\dots,\mathbf{r}_N,\alpha,\beta,\dots,\nu) = \frac{1}{\sqrt{A!}}\sum_{P} (-)^PP\psi_{\alpha}(\mathbf{r}_1)
    \psi_{\beta}(\mathbf{r}_2)\dots\psi_{\nu}(\mathbf{r}_N)=\sqrt{A!}{\cal A}\Phi_H,
\label{HartreeFockPermutation}
\end{equation}
where we have introduced the antisymmetrization operator ${\cal A}$ defined by the 
summation over all possible permutations of two nucleons.
It is defined as
\begin{equation}
  {\cal A} = \frac{1}{A!}\sum_{p} (-)^p\hat{P},
\label{antiSymmetryOperator}
\end{equation}
with $p$ standing for the number of permutations. We have introduced for later use the so-called
Hartree-function, defined by the simple product of all possible single-particle functions
\begin{equation*}
  \Phi_H(\mathbf{r}_1,\mathbf{r}_2,\dots,\mathbf{r}_N,\alpha,\beta,\dots,\nu) =
  \psi_{\alpha}(\mathbf{r}_1)
    \psi_{\beta}(\mathbf{r}_2)\dots\psi_{\nu}(\mathbf{r}_N).
\end{equation*}

Both $\hat{H_0}$ and $\hat{H_1}$ are invariant under all possible permutations of any two nucleons 
and hence commute with ${\cal A}$
\begin{equation}
  [H_0,{\cal A}] = [H_1,{\cal A}] = 0.
  \label{cummutionAntiSym}
\end{equation}
Furthermore, ${\cal A}$ satisfies
\begin{equation}
  {\cal A}^2 = {\cal A},
  \label{AntiSymSquared}
\end{equation}
since every permutation of the Slater
determinant reproduces it. The expectation value of $\hat{H_0}$ 
\[
  \int \Phi^*\hat{H_0}\Phi d\mathbf{\tau} 
  = A! \int \Phi_H^*{\cal A}\hat{H_0}{\cal A}\Phi_H d\mathbf{\tau}
\]
is readily reduced to
\[
  \int \Phi^*\hat{H_0}\Phi d\mathbf{\tau} 
  = A! \int \Phi_H^*\hat{H_0}{\cal A}\Phi_H d\mathbf{\tau},
\]
where we have used eqs.~(\ref{cummutionAntiSym}) and
(\ref{AntiSymSquared}). The next step is to replace the antisymmetrization
operator by its definition Eq.~(\ref{HartreeFockPermutation}) and to
replace $\hat{H_0}$ with the sum of one-body operators
\[
  \int \Phi^*\hat{H_9}\Phi  d\mathbf{\tau}
  = \sum_{i=1}^A \sum_{p} (-)^p\int 
  \Phi_H^*\hat{h_i}\hat{P}\Phi_H d\mathbf{\tau}.
\]

The integral vanishes if two or more nucleons are permuted in only one
of the Hartree-functions $\Phi_H$ because the individual single-particle wave functions are
orthogonal. We obtain then
\[
  \int \Phi^*\hat{H}_0\Phi  d\mathbf{\tau}= \sum_{i=1}^A \int \Phi_H^*\hat{h}_i\Phi_H  d\mathbf{\tau}.
\]
Orthogonality of the single-particle functions allows us to further simplify the integral, and we
arrive at the following expression for the expectation values of the
sum of one-body Hamiltonians 
\begin{equation}
  \int \Phi^*\hat{H_0}\Phi  d\mathbf{\tau}
  = \sum_{\mu=1}^A \int \psi_{\mu}^*(\mathbf{r})\hat{h}\psi_{\mu}(\mathbf{r})
  d\mathbf{r}.
  \label{H1Expectation}
\end{equation}
We introduce the following shorthand for the above integral
\[
\langle \mu | h | \mu \rangle = \int \psi_{\mu}^*(\mathbf{r})\hat{h}\psi_{\mu}(\mathbf{r}),
\]
and rewrite Eq.~(\ref{H1Expectation}) as
\begin{equation}
  \int \Phi^*\hat{H_0}\Phi  d\mathbf{\tau}
  = \sum_{\mu=1}^A \langle \mu | h | \mu \rangle.
  \label{H1Expectation1}
\end{equation}

The expectation value of the two-body Hamiltonian is obtained in a
similar manner. We have
\begin{equation*}
  \int \Phi^*\hat{H_1}\Phi d\mathbf{\tau} 
  = A! \int \Phi_H^*{\cal A}\hat{H_2}{\cal A}\Phi_H d\mathbf{\tau},
\end{equation*}
which reduces to
\begin{equation*}
 \int \Phi^*\hat{H_1}\Phi d\mathbf{\tau} 
  = \sum_{i\le j=1}^A \sum_{p} (-)^p\int 
  \Phi_H^*V(r_{ij})\hat{P}\Phi_H d\mathbf{\tau},
\end{equation*}
by following the same arguments as for the one-body
Hamiltonian. Because of the dependence on the inter-nucleon distance $r_{ij}$,  permutations of
any two nucleons no longer vanish, and we get
\begin{equation*}
  \int \Phi^*\hat{H_1}\Phi d\mathbf{\tau} 
  = \sum_{i < j=1}^A \int  
  \Phi_H^*Vr_{ij})(1-P_{ij})\Phi_H d\mathbf{\tau}.
\end{equation*}
where $P_{ij}$ is the permutation operator that interchanges
nucleon $i$ and nucleon $j$. Again we use the assumption that the single-particle wave functions
are orthogonal. We obtain
\begin{equation}
\begin{split}
  \int \Phi^*\hat{H_1}\Phi d\mathbf{\tau} 
  = \frac{1}{2}\sum_{\mu=1}^A\sum_{\nu=1}^A
    &\left[ \int \psi_{\mu}^*(\mathbf{r}_i)\psi_{\nu}^*(\mathbf{r}_j)V(r_{ij})\psi_{\mu}(\mathbf{r}_i)\psi_{\nu}(\mathbf{r}_j)
    d\mathbf{r}_i\mathbf{r}_j \right.\\
  &\left.
  - \int \psi_{\mu}^*(\mathbf{r}_i)\psi_{\nu}^*(\mathbf{r}_j)
  V(r_{ij})\psi_{\nu}(\mathbf{r}_i)\psi_{\mu}(\mathbf{r}_i)
  d\mathbf{r}_i\mathbf{r}_j
  \right]. \label{H2Expectation}
\end{split}
\end{equation}
The first term is the so-called direct term. It is frequently also called the  Hartree term, 
while the second is due to the Pauli principle and is called
the exchange term or just the Fock term.
The factor  $1/2$ is introduced because we now run over
all pairs twice. 

The last equation allows us to  introduce some further definitions.  
The single-particle wave functions $\psi_{\mu}({\bf r})$, defined by the quantum numbers $\mu$ and ${\bf r}$
(recall that ${\bf r}$ also includes spin degree)   are defined as the overlap 
\[
   \psi_{\alpha}({\bf r})  = \langle {\bf r} | \alpha \rangle .
\]
Since we will work in what is called the occupation number representation  (or just second quantization)
of quantum mechanical states and operators\footnote{Loosely speaking often refered to
as Fock space or second quantization.},  we will not show the coordinate space dependence of various quantum mechanical objects. 
  
We introduce the following shorthands for the above two integrals
\[
\langle \mu\nu|V|\mu\nu\rangle =  \int \psi_{\mu}^*(\mathbf{r}_i)\psi_{\nu}^*(\mathbf{r}_j)V(r_{ij})\psi_{\mu}(\mathbf{r}_i)\psi_{\nu}(\mathbf{r}_j)
    d\mathbf{r}_i\mathbf{r}_j,
\]
and 
\[
\langle \mu\nu|V|\nu\mu\rangle = \int \psi_{\mu}^*(\mathbf{r}_i)\psi_{\nu}^*(\mathbf{r}_j)
  V(r_{ij})\psi_{\nu}(\mathbf{r}_i)\psi_{\mu}(\mathbf{r}_i)
  d\mathbf{r}_i\mathbf{r}_j.  
\]
Since the interaction is invariant under the interchange of two particles it means for example that we have
\[
\langle \mu\nu|V|\mu\nu\rangle =  \langle \nu\mu|V|\nu\mu\rangle,  
\]
or in the more general case
\[
\langle \mu\nu|V|\sigma\tau\rangle =  \langle \nu\mu|V|\tau\sigma\rangle.  
\]

The direct and exchange matrix elements can be  brought together if we define the antisymmetrized matrix element
\[
\langle \mu\nu|V|\mu\nu\rangle_{AS}= \langle \mu\nu|V|\mu\nu\rangle-\langle \mu\nu|V|\nu\mu\rangle,
\]
or for a general matrix element  
\[
\langle \mu\nu|V|\sigma\tau\rangle_{AS}= \langle \mu\nu|V|\sigma\tau\rangle-\langle \mu\nu|V|\tau\sigma\rangle.
\]
It has the symmetry property
\[
\langle \mu\nu|V|\sigma\tau\rangle_{AS}= -\langle \mu\nu|V|\tau\sigma\rangle_{AS}=-\langle \nu\mu|V|\sigma\tau\rangle_{AS}.
\]
The antisymmetric matrix element is also hermitian, implying 
\[
\langle \mu\nu|V|\sigma\tau\rangle_{AS}= \langle \sigma\tau|V|\mu\nu\rangle_{AS}.
\]

With these notations we rewrite Eq.~(\ref{H2Expectation}) as 
\begin{equation}
  \int \Phi^*\hat{H_1}\Phi d\mathbf{\tau} 
  = \frac{1}{2}\sum_{\mu=1}^A\sum_{\nu=1}^A \langle \mu\nu|V|\mu\nu\rangle_{AS}.
\label{H2Expectation2}
\end{equation}


Combining Eqs.~(\ref{H1Expectation1}) and
(\ref{H2Expectation2}) we obtain the energy functional 
\begin{equation}
  E[\Phi] 
  = \sum_{\mu=1}^A \langle \mu | h | \mu \rangle +
  \frac{1}{2}\sum_{{\mu}=1}^A\sum_{{\nu}=1}^A \langle \mu\nu|V|\mu\nu\rangle_{AS}.
\label{FunctionalEPhi}
\end{equation}
which we will use as our starting point for the Hartree-Fock calculations. 



\section{Examples of physical systems}

\subsection{Atomic physics}



In this section
we limit ourselves to studies of electronic systems such atoms, molecules and quantum dots, 
as discussed partly in chapter \ref{chap:mcvar} as well.
Using the Born-Oppenheimer approximation we rewrote Schr\"odinger's equation for $N$ electrons as 
\[
  \left[-\sum_{i=1}^N \frac{1}{2} \nabla^2_i 
    - \sum_{i=1}^N \frac{Z}{r_i} + \sum_{i<j}^N \frac{1}{r_{ij}} 
    \right] \Psi(\mathbf{R}) = E \Psi(\mathbf{R}), 
\]
where we let $\mathbf{R}$ represent the positions which the $N$ electrons can take, that is $\mathbf{R}=\left\{\mathbf{r}_1,\mathbf{r}_2,\dots,\mathbf{r}_N\right\}$. 
With more than one electron present we cannot find an solution on a closed form and must resort to numerical efforts. In this
chapter we will examine Hartree-Fock theory 
applied to the atomic problem. However, the machinery we expose can easily be extended to studies of molecules or two-dimensional systems like quantum dots.

For atoms and molecules, the electron-electron interaction 
is rather weak compared with the attraction from the nucleus. An independent particle picture
is therefore a viable first step towards the solution of Schr\"odinger's equation. 
We assume therefore that each electrons sees an effective field set up by the other electrons.
This leads to an integro-differential equation  and methods like Hartree-Fock theory and density functional theory.
Hartree-Fock theory and density functional theories are discussed in the next section.

In practical terms, for the Hartree-Fock method we end up solving a one-particle equation, as is the case for the hydrogen atom but modified due 
to the screening from the other electrons.  This modified single-particle equation reads (see Eq.~(\ref{eq:radialsl} for the hydrogen case)
in atomic units
\[
  -\frac{1}{2} \frac{d^2}{dr^2} u_{nl}(r) 
       + \left (\frac{l (l + 1)}{2r^2}-\frac{Z}{r}+ \Phi(r)+F_{nl}\right ) u_{nl}(r)  = e_{nl} u_{nl}(r) .
\]
The function $u_{nl}$ is the solution of the radial part of the Schr\"odinger equation and the functions $\Phi(r)$ and
$F_{nl}$ are the corrections due to the screening from the other electrons.  We will derive these equations in the next section.

The total one-particle wave function, see chapter \ref{chap:mcvar}  is 
\[
  \psi_{nlm_lsm_s} = \phi_{nlm_l}({\bf r})\xi_{m_s}(s)
\]
with $s$ is the spin ($1/2$ for electrons), $m_s$ is the spin projection $m_s=\pm 1/2$, and the spatial part is
\[
   \phi_{nlm_l}({\bf r}) =  R_{nl}(r)Y_{lm_l}(\hat{{\bf r}})
\]
with $Y$ the spherical harmonics discussed in chapter \ref{chap:mcvar} and $u_{nl} = rR_{nl}$.
The other quantum numbers are the orbital momentum  $l$ and its projection $m_l=-l,-l+1,\dots,l-1,l$ and the principal quantum
number $n=n_r+l+1$, with $n_r$ the number of nodes of a given single-particle wave function. 
All results are in atomic units, meaning that the energy is given by $e_{nl}=-Z^2/2n^2$ and the radius is dimensionless.


We obtain then a modified single-particle eigenfunction which in turn can be used
as an input in a variational Monte Carlo calculation of the ground state of a specific atom. 
This is the aim of the next chapter. Since Hartree-Fock theory does not treat 
correctly the role of many-body correlations, the hope is that
performing a Monte Carlo calculation we may improve our results by obtaining a better agreement with experiment.

In the next chapter we focus 
on the variational Monte Carlo method as a way to improve upon the Hartree-Fock results.  
The method was discussed in chapter \ref{chap:mcvar}. In chapter \ref{chap:mcvar} however, we limited ourselves 
to the implementation of a bruce force Metropolis algorithm. In chapter \ref{chap:improvedvmc} 
we present the concept of importance sampling and improved statistical data analysis methods such as
data blocking and the Jack-Knife method \cite{nb1999}.  We discuss also the conjugate gradient method
as a way to find variational minima.

Although the variational Monte Carlo approach will improve our agreement with experiment compared with the Hartree-Fock results, there are still further possibilities
for improvement. This is provided by Green's function Monte Carlo methods, which allow for an in principle exact calculation.
The diffusion Monte Carlo method is discussed in chapter 
\ref{chap:advancedqmc}, with an application to studies of Bose-Einstein condensation.

Other many-body methods such as large-scale diagonalization and coupled-cluster theories are 
discussed in Ref.~\cite{deanhj2009}.
Finally, chapter \ref{chap:quantinfo} demonstrates how algorithms from quantum information theory can be used to solve Schr\"odinger's equation
for many interacting particles. 




Hartree-Fock theory \cite{helgaker,bransden1983}
is one of the simplest approximate theories  
for solving the many-body Hamiltonian. It is based on a simple
approximation to the true many-body wave-function; that the
wave-function is given by a single Slater determinant of $N$ 
orthonormal single-particle wave functions\footnote{We limit ourselves to a restricted 
Hartree-Fock approach and assume that all the lowest-lying orbits are filled. This constitutes 
an approach suitable for systems with filled shells. 
The theory we outline is therefore applicable to systems which 
exhibit so-called magic numbers like the noble gases, closed-shell nuclei  
like $^{16}$O and $^{40}$Ca and quantum dots with magic number fillings.}
\[
  \psi_{nlm_lsm_s} = \phi_{nlm_l}({\bf r})\xi_{m_s}(s).
\]
We use hereafter the shorthand $\psi_{nlm_lsm_s}({\bf r}) = \psi_{\alpha}({\bf r})$,
where $\alpha$ now contains all the quantum numbers  needed to specify a particular single-particle orbital.

The Slater determinant can then be written as   
\begin{equation}
  \Phi(\mathbf{r}_1,\mathbf{r}_2,\dots,\mathbf{r}_N,\alpha,\beta,\dots,\nu)  = \frac{1}{\sqrt{N!}}\left| 
  \begin{array}{cccc}
    \psi_{\alpha}(\mathbf{r}_1)&\psi_{\alpha}(\mathbf{r}_2)&\dots&\psi_{\alpha}(\mathbf{r}_N) \\ [4pt]
    \psi_{\beta}(\mathbf{r}_1)&\psi_{\beta}(\mathbf{r}_2)&\dots&\psi_{\beta}(\mathbf{r}_N) \\[4pt] 
    \vdots              & \vdots            &\ddots&\vdots\\[4pt]
    \psi_{\nu}(\mathbf{r}_1)&\psi_{\beta}(\mathbf{r}_2)&\dots&\psi_{\beta}(\mathbf{r}_N)
  \end{array}
  \right|.
\label{HartreeFockDet}
\end{equation}
Here the variables $\mathbf{r}_i$ include the coordinates of 
spin and space of particle $i$. The quantum numbers $\alpha,\beta,\dots,\nu$ encompass all possible quantum numbers needed to specify a 
particular system. As an example, consider the Neon atom, with ten electrons which can fill the $1s$, $2s$ and $2p$ single-particle
orbitals. Due to the spin projections $m_s$ and orbital momentum projections $m_l$, the $1s$ and $2s$ states have  a degeneracy of $2(2l+1)=2$ while the $2p$ orbital has
a degeneracy of $2(2l+1) 2(2\cdot 1+1)= 6$.  This leads to ten possible values for  $\alpha,\beta,\dots,\nu$.  
Fig.~\ref{fig:tenfirstelements} shows the possible quantum numbers which the ten first elements can have.
%
\begin{figure}
%
\begin{center}
%
\begin{pspicture}(10,6)

%%%%%%%%%%

%%%     linje 1   %%%

\rput(0,4.5){
             \rput(0,0){
                \psframe(0,0)(0.5,0.5)
              }
              \multiput(0,0.5)(0.5,0){4}{ 
                 \psframe(0,0)(0.5,0.5)
              }
              \rput(0.2,1.1){s}
              \rput(1.2,1.1){p}
              \rput(-0.3,.2){K}
              \rput(-0.3,.7){L}
              \rput(1.2,.2){H}

              \psline{->}(0.25,0.05)(0.25,0.45)

} %%* end rput linje 1

%%%  linje 2   %%%

\rput(0,3){
          \multiput(0,0)(2.5,0){2}  {
             \rput(0,0){
                \psframe(0,0)(0.5,0.5)   
                 \psline{->}(0.15,0.05)(0.15,0.45)
                 \psline{<-}(0.35,0.05)(0.35,0.45)

                \rput(0.2,1.1){s}
                \rput(1.2,1.1){p}
             }
             \multiput(0,0.5)(0.5,0){4}{ 
                 \psframe(0,0)(0.5,0.5)
             }
          }
          \rput(-0.3,.2){K}
          \rput(-0.3,.7){L}
          \rput(1.2,.2){He}
          \rput(3.7,.2){Li}
          \psline{->}(2.75,0.55)(2.75,0.95)

}  %* end rput linje 2

%%%   linje 3  %%%%

\rput(0,1.5)  {
             \multiput(0,0)(2.5,0){4}  {
%%%%%%%%%%
                \rput(0,0){
                   \psframe(0,0)(0.5,0.5)
                   \psline{->}(0.15,0.05)(0.15,0.45)
                   \psline{<-}(0.35,0.05)(0.35,0.45) 
                }  
                \multiput(0,0.5)(0.5,0){4}  {
                   \psframe(0,0)(0.5,0.5)
                }
                \rput(0.2,1.1){s}
                \rput(1.2,1.1){p}
                \psline{->}(0.15,0.55)(0.15,0.95)
                \psline{<-}(0.35,0.55)(0.35,0.95)
             }
             \rput(-0.3,0.2){$n=1$}  
             \rput(-0.3,0.7){$n=2$}
             \rput(1.2,0.2){Be}     
             \rput(3.7,0.2){B}   
             \rput(6.2,0.2){C}
             \rput(8.7,0.2){N}
             \multiput(2.5,0.55)(2.5,0){3} {
               \psline{->}(0.75,0.05)(0.75,0.45)
             }
             \multiput(5,0.55)(2.5,0){2} {
                 \psline{->}(1.25,0.05)(1.25,0.45)
              }
              \psline{->}(9.25,0.55)(9.25,0.95)

}    %% end linje 3 

%%%   linje 4   %%%
\rput(0,0)  {
             \multiput(0,0)(2.5,0){3} {
%%%%%%%%%%
                \rput(0,0){
                   \psframe(0,0)(0.5,0.5)
                   \psline{->}(0.15,0.05)(0.15,0.45)
                   \psline{<-}(0.35,0.05)(0.35,0.45)
                }
                \multiput(0,0.5)(0.5,0){4}  {
                   \psframe(0,0)(0.5,0.5)
                }
                \psline{->}(0.15,0.55)(0.15,0.95)
                \psline{<-}(0.35,0.55)(0.35,0.95)
                \psline{->}(0.65,0.55)(0.65,0.95)
                \psline{<-}(0.85,0.55)(0.85,0.95)

                \rput(0.2,1.1){s}
                \rput(1.2,1.1){p}
             }

             \psline{->}(1.25,0.55)(1.25,0.95)
             \psline{->}(1.75,0.55)(1.75,0.95)

             \multiput(2.5,0.55)(2.5,0){2}  {
                \psline{->}(1.15,0)(1.15,0.45)
                \psline{<-}(1.35,0)(1.35,0.45)
             }
             \psline{->}(4.25,0.55)(4.25,0.95)
             \psline{->}(6.65,0.55)(6.65,0.95)
             \psline{<-}(6.85,0.55)(6.85,0.95)

             \rput(-0.3,0.2){$n=1$}
             \rput(-0.3,0.7){$n=2$}
             \rput(1.2,0.2){O}
             \rput(3.7,0.2){F}
             \rput(6.2,0.2){Ne}
 }   %% end linje 4 

\end{pspicture}
%
\end{center}
\caption{The electronic configurations for the ten first elements. We let an arrow which points upward to represent a state with $m_s=1/2$ while an arrow which points downwards
has $m_s=-1/2$. \label{fig:tenfirstelements} }
\end{figure}


If we consider the helium atom with two electrons in the $1s$ state, we can write the total Slater determinant as 
\be
   \Phi({\bf r}_1,{\bf r}_2,\alpha,\beta)=\frac{1}{\sqrt{2}}
\left| \begin{array}{cc} \psi_{\alpha}({\bf r}_1)& \psi_{\alpha}({\bf r}_2)\\\psi_{\beta}({\bf r}_1)&\psi_{\beta}({\bf r}_2)\end{array} \right|,
\ee 
with $\alpha=nlm_lsm_s=(1001/21/2)$ and $\beta=nlm_lsm_s=(1001/2-1/2)$  or using $m_s=1/2=\uparrow$ and $m_s=-1/2=\downarrow$ as 
$\alpha=nlm_lsm_s=(1001/2\uparrow)$ and $\beta=nlm_lsm_s=(1001/2\downarrow)$.
It is normal to skip the quantum number of the one-electron spin. We introduce therefore the shorthand
 $nlm_l\uparrow$ or $nlm_l\downarrow)$ for a particular state.
Writing out the Slater determinant
\be
\Phi({\bf r}_1,{\bf r}_2,\alpha,\beta)=
\frac{1}{\sqrt{2}}\left[
\psi_{\alpha}({\bf r}_1)\psi_{\beta}({\bf r}_2)-
\psi_{\beta}({\bf r}_1)\psi_{\gamma}({\bf r}_2)\right],
\ee
we see that the Slater determinant is antisymmetric 
with respect to the permutation of two particles, that is
\[
\Phi({\bf r}_1,{\bf r}_2,\alpha,\beta)=-\Phi({\bf r}_2,{\bf r}_1,\alpha,\beta),
\]
For three electrons we have  the general expression
\be
   \Phi({\bf r}_1,{\bf r}_2,{\bf r}_3,\alpha,\beta,\gamma)=\frac{1}{\sqrt{3!}}
\left| \begin{array}{ccc} \psi_{\alpha}({\bf r}_1)& \psi_{\alpha}({\bf r}_2)& \psi_{\alpha}({\bf r}_3)\\\psi_{\beta}({\bf r}_1)&\psi_{\beta}({\bf r}_2)&\psi_{\beta}({\bf r}_3)\\\psi_{\gamma}({\bf r}_1)&\psi_{\gamma}({\bf r}_2)&\psi_{\gamma}({\bf r}_3)\end{array} \right|.
\ee 
Computing the determinant gives 
\begin{eqnarray}
\Phi({\bf r}_1,{\bf r}_2,{\bf r}_3,\alpha,\beta,\gamma)&=
\frac{1}{\sqrt{3!}}\left[
\psi_{\alpha}({\bf r}_1)\psi_{\beta}({\bf r}_2)\psi_{\gamma}({\bf r}_3)+
\psi_{\beta}({\bf r}_1)\psi_{\gamma}({\bf r}_2)\psi_{\alpha}({\bf r}_3)+
\psi_{\gamma}({\bf r}_1)\psi_{\alpha}({\bf r}_2)\psi_{\beta}({\bf r}_3)-\right. \nonumber \\
&\left.\psi_{\gamma}({\bf r}_1)\psi_{\beta}({\bf r}_2)\psi_{\alpha}({\bf r}_3)-
\psi_{\beta}({\bf r}_1)\psi_{\alpha}({\bf r}_2)\psi_{\gamma}({\bf r}_3)-
\psi_{\alpha}({\bf r}_1)\psi_{\gamma}({\bf r}_2)\psi_{\beta}({\bf r}_3)
\right].
\end{eqnarray}
We note again that 
the wave-function is antisymmetric with respect to an
interchange of any two electrons, as required by the Pauli principle. For an $N$-body Slater determinant we have thus
(omitting the quantum numbers $\alpha,\beta,\dots,\nu$)
\[
  \Phi(\mathbf{r}_1, \mathbf{r}_2, \dots, \mathbf{r}_i, \dots,
  \mathbf{r}_j, \dots \mathbf{r}_N) = -
  \Phi(\mathbf{r}_1, \mathbf{r}_2, \dots, \mathbf{r}_j, \dots,
  \mathbf{r}_i, \dots \mathbf{r}_N).
\]

As another example, consider the Slater determinant for the ground state of beryllium. This system
is made up of four electrons and we assume that these electrons fill the $1s$ and $2s$ hydrogen-like
orbits.  
The radial part of the single-particle could also be represented by other single-particle wave functions
such as those given by the harmonic oscillator.

The ansatz for the Slater determinant can then be written as  
\[
   \Phi({\bf r}_1,{\bf r}_2,,{\bf r}_3,{\bf r}_4, \alpha,\beta,\gamma,\delta)=\frac{1}{\sqrt{4!}}
\left| \begin{array}{cccc} \psi_{100\uparrow}({\bf r}_1)& \psi_{100\uparrow}({\bf r}_2)& \psi_{100\uparrow}({\bf r}_3)&\psi_{100\uparrow}({\bf r}_4) \\
\psi_{100\downarrow}({\bf r}_1)& \psi_{100\downarrow}({\bf r}_2)& \psi_{100\downarrow}({\bf r}_3)&\psi_{100\downarrow}({\bf r}_4) \\
\psi_{200\uparrow}({\bf r}_1)& \psi_{200\uparrow}({\bf r}_2)& \psi_{200\uparrow}({\bf r}_3)&\psi_{200\uparrow}({\bf r}_4) \\
\psi_{200\downarrow}({\bf r}_1)& \psi_{200\downarrow}({\bf r}_2)& \psi_{200\downarrow}({\bf r}_3)&\psi_{200\downarrow}({\bf r}_4) \end{array} \right|.
\]
We choose an ordering where columns represent the spatial positions of various
electrons while rows refer to specific quantum numbers.

Note that the Slater determinant as written is zero since the spatial wave functions for the spin up and spin down 
states are equal.   However, we can rewrite
But we can rewrite it as the product of two Slater determinants, one for spin up and one for spin down.
In general we can rewrite it as 
\[
   \Phi({\bf r}_1,{\bf r}_2,,{\bf r}_3,{\bf r}_4, \alpha,\beta,\gamma,\delta)=Det\uparrow(1,2)Det\downarrow(3,4)-
Det\uparrow(1,3)Det\downarrow(2,4)
\]
\[
-Det\uparrow(1,4)Det\downarrow(3,2)+Det\uparrow(2,3)Det\downarrow(1,4)-Det\uparrow(2,4)Det\downarrow(1,3)
\]
\[
+Det\uparrow(3,4)Det\downarrow(1,2),
\]
where we have defined
\[
Det\uparrow(1,2)=\left| \frac{1}{\sqrt{2}}\begin{array}{cc} \psi_{100\uparrow}({\bf r}_1)& \psi_{100\uparrow}({\bf r}_2)\\
\psi_{200\uparrow}({\bf r}_1)& \psi_{200\uparrow}({\bf r}_2) \end{array} \right|,
\]
and 
\[
Det\downarrow(3,4)=\left| \frac{1}{\sqrt{2}}\begin{array}{cc} \psi_{100\downarrow}({\bf r}_3)& \psi_{100\downarrow}({\bf r}_4)\\
\psi_{200\downarrow}({\bf r}_3)& \psi_{200\downarrow}({\bf r}_4) \end{array} \right|.
\]
The total determinant is still zero!  In our variational Monte Carlo calculations this will obviously cause
problems.

We want to avoid to sum over spin variables, in particular when the interaction does not depend on spin.
It can be shown, see for example Moskowitz {\em et al} \cite{moskowitz1981,schmidt1982}, 
that for the variational energy
we can approximate the Slater determinant as  the product of a spin up and a spin down Slater determinant
\[
   \Phi({\bf r}_1,{\bf r}_2,,{\bf r}_3,{\bf r}_4, \alpha,\beta,\gamma,\delta) \propto Det\uparrow(1,2)Det\downarrow(3,4),
\]
or more generally as 
\[
   \Phi({\bf r}_1,{\bf r}_2,\dots {\bf r}_N) \propto Det\uparrow Det\downarrow,
\]
where we have the Slater determinant as the product of a spin up part involving the number of electrons 
with spin up only (two in beryllium
and five in neon) and a spin down part involving the electrons with spin down.

This ansatz is not antisymmetric under the exchange of electrons with  opposite spins but 
it can be shown that it gives the same
expectation value for the energy as the full Slater determinant
as long as the Hamiltonian is spin independent. It is left as an exercise to the reader to show this.
However, before  we can prove this need to set up the expectation value of a given two-particle Hamiltonian using a
Slater determinant.

\subsection{Quantum dots}



The interaction potential between two electrons is usually approximated proportional to the Coulomb repulsion in free space $V(\vec{r_i},\vec{r_j})=1/r_{ij}$. Other studies have investigated different forms of potential. For example Johnson and Payne \cite{johnsonPayne} assumed the interaction potential $V(\vec{r_i},\vec{r_j})$ between particles $i$ and $j$ moving in the confining potential to saturate at small particle separation and to decrease quadratically with increasing separation.
More recently, in order to investigate spin relaxation in quantum dots, Chaney and Maksym in~\cite{Chaney2007} built a model where the electron-electron interactions were designed to follow experimental data.

For sake of simplicity and since it is still in use in most studies of quantum dots, we will stick to the approximation of the interaction potential proportional to the Coulomb repulsion.

%model of QD in this book~\cite{Joyce2005} (including Auger effect, but complicated form of the interaction potential).

Defining the second potential that confines these electrons is a more difficult issue when modelling a quantum dot. Some numerical~\cite{KumarA1990,Macucci1993,Macucci1997,Stopa1996} and experimental~\cite{Kohn1961,Heitmann1993,Heitmann1995} studies have shown that for a small number of trapped electrons, the harmonic oscillator potential is a good approximation, at least to first approximation.
In~\cite{johnsonPayne}, the bare (i.e.\ unscreened) confining potential $V(\vec{r_i})$ for the $i^{th}$ particle is also modelled to be parabolic (i.e.\ the harmonic oscillator potential). It has been shown theoretically that for electrons contained in a parabolic potential there is a strong absorption of far-infrared light at the frequency corresponding to the bare parabola~\cite{Brey1989,Peeters1990,Yip1991,Li1991}. This theoretical prediction is consistent with some experimental measurements on quantum dots~\cite{Sikorski1989}. Further evidence that the bare potential in many quantum-dot samples is close to parabolic is provided by simple electrostatic models~\cite{Dempsey1990}.

%[\textit{\textcolor{red}{Waltersson discuss the limit  of this choice: ok only for small Nb of electrons $Ne<20$,etc}}]


We consider a system of electrons confined in a pure isotropic harmonic oscillator potential $V(\vec{r})=m^* \omega_0^2 r^2/2$, where $m^*$ is the effective mass of the electrons in the host semiconductor (as defined in section~\ref{semiconductors}), $\omega_0$ is the oscillator frequency of the confining potential, and $\vec{r}=(x,y,z)$ denotes the position of the particle.

The Hamiltonian of a single particle trapped in this harmonic oscillator potential simply reads
\begin{equation}
  \hat{H}= \frac{\textbf{p}^2}{2m^*}  + \frac{1}{2} m^* \omega_0^2 {\textbf{r}}^2
\end{equation}
where $\textbf{p}$ is the canonical momentum of the particle.

When considering several particles trapped in the same quantum dot, the Coulomb repulsion between those electrons has to be added to the single particle Hamiltonian which gives
\begin{equation}
\hat{H}=\sum_{i=1}^{N_e} \left( \frac{\mathbf{p_i}^2}{2m^*}+ \frac{1}{2} m^* \omega_0^2 {\mathbf{r_i}}^2 \right) + \frac{e^2}{4 \pi \epsilon_0 \epsilon_r} \sum_{i<j} \frac{1}{{\mathbf{r_i}-\mathbf{r_j}}},
\end{equation}
where $N_e$ is the number of electrons, $-e \;  (e>0)$ is the charge of the electron, $\epsilon_0$ and $\epsilon_r$ are respectively the free space permitivity and the relative permitivity of the host material (also called dielectric constant), and the index $i$ labels the electrons.


We assume that the magnetic field $\overrightarrow{B}$ is static and along the $z$ axis.
At first we ignore the spin-dependent terms. The Hamiltonian of these electrons in a magnetic field now reads ~\cite{Bransden2003}
\begin{align}
  \hat{H}&=&\sum_{i=1}^{N_e}\left(  \frac{(\mathbf{p_i}+e\mathbf{A})^2}{2m^*}  + \frac{1}{2} m^* \omega_0^2 {\mathbf{r_i}}^2  \right) + \frac{e^2}{4 \pi \epsilon_0 \epsilon_r} \sum_{i<j}\frac{1}{{\mathbf{r_i}-\mathbf{r_j}}}, \\
&=&\sum_{i=1}^{N_e}\left(  \frac{\mathbf{p_i}^2}{2m^*} + \frac{e}{2m^*}(\mathbf{A}\cdot \mathbf{p_i}+\mathbf{p_i}\cdot \mathbf{A}) + \frac{e^2}{2m^*}\mathbf{A}^2  + \frac{1}{2} m^* \omega_0^2 {\mathbf{r_i}}^2  \right) \\
& &+ \frac{e^2}{4 \pi \epsilon_0 \epsilon_r} \sum_{i<j}\frac{1}{{\mathbf{r_i}-\mathbf{r_j}}},
\end{align}
where $\mathbf{A}$ is the vector potential defined by $\mathbf{B}=\nabla \times \mathbf{A}$.

In coordinate space, $\mathbf{p_i}$ is the operator $-i \hbar \nabla_i$ and by applying the Hamiltonian on the total wave function $\Psi(\mathbf{r})$ in the Schr\"odinger equation, we obtain the following operator acting on $\Psi(\mathbf{r})$
\begin{align}
\mathbf{A}\cdot \mathbf{p_i}+\mathbf{p_i}\cdot \mathbf{A} &= - i \hbar \left( \mathbf{A}\cdot \nabla_i+\nabla_i\cdot \mathbf{A} \right) \Psi \\
&=  - i \hbar \left( \mathbf{A}\cdot  ( \nabla_i \Psi)+\nabla_i\cdot (\mathbf{A} \Psi) \right) 
\end{align}

We note that if we use the product rule and the Coulomb gauge $\nabla \cdot \mathbf{A} = 0$ (by choosing the vector potential as $\mathbf{A} = \frac{1}{2} \mathbf{B} \times \mathbf{r}$), $\mathbf{p_i}$ and $\nabla_i$ commute and we obtain
\begin{equation}
 \nabla_i \cdot (\mathbf{A}\Psi) = \mathbf{A} \cdot (\nabla_i\Psi) + (\underbrace{\nabla_i \cdot \mathbf{A})}_0 \Psi=\mathbf{A} \cdot (\nabla_i\Psi) 
\end{equation}

This leads us to the following Hamiltonian:
\begin{align}
\label{eq:Hamiltonian2}
  \hat{H}&=&\sum_{i=1}^{N_e}\left(  -\frac{ \hbar^2}{2m^*} \nabla_i^2- i \hbar \frac{e}{m^*} \mathbf{A}\cdot \nabla_i + \frac{e^2}{2m^*}\mathbf{A}^2  + \frac{1}{2} m^* \omega_0^2 {\mathbf{r_i}}^2  \right) \\
& &+ \frac{e^2}{4 \pi \epsilon_0 \epsilon_r} \sum_{i<j}\frac{1}{{\mathbf{r_i}-\mathbf{r_j}}},
\end{align}

The linear term in $\mathbf{A}$ becomes, in terms of $\mathbf{B}$:
\begin{align}
\label{eq:linearTermA}
\frac{-i \hbar e}{m^*} \mathbf{A} \cdot \nabla_i &= -\frac{i \hbar e}{2m^*} (\mathbf{B} \times \mathbf{r_i}) \cdot \nabla_i \\
&= \frac{-i \hbar e}{2m^*} \mathbf{B} \cdot( \mathbf{r_i} \times \nabla_i)  \\
&= \frac{ e}{2m^*} \mathbf{B} \cdot \mathbf{L} 
\end{align}
where $\mathbf{L}=-i \hbar (\mathbf{r_i} \times \nabla_i)$ is the orbital angular momentum operator of the electron $i$.

If we assume that the electrons are confined in the $xy$-plane, the quadratic term in $\mathbf{A}$ appearing in \ref{eq:Hamiltonian2} can be written under the form
\begin{align}
\frac{e^2}{2m^*} \mathbf{A}^2 &= \frac{e^2}{8m^*} (\mathbf{B} \times \mathbf{r})^2 \\
&= \frac{e^2}{8m^*} B^2 r_i^2
\end{align}

Until this point we have neglected the intrinsic magnetic moment of the electrons which is due to the electron spin in the host material. We will now add its effect to the Hamiltonian. This intrinsic magnetic moment is given by $\mathcal{M}_s=-g^*_s (e \mathbf{S})/(2m^*)$, where $\mathbf{S}$ is the spin operator of the electron and $g^*_s$ its effective spin gyromagnetic ratio (or effective \textit{g-factor} in the host material).% Dirac's relativistic theory predicts for $g_s$, the value $g_s=2$ which is in very good agreement with experiment~\cite{Bransden2003}.
We see that the spin magnetic moment $\mathcal{M}_s$ gives rise to an additional interaction energy~\cite{Bransden2003}, linear in the magnetic field,
\begin{equation}
\label{eq:spinHs}
  \hat{H_s}= - \mathcal{M}_s \cdot \mathbf{B} = g^*_s \frac{e }{2 m^*} B \hat{S_z}= g^*_s \frac{\omega_c}{2} \hat{S_z}
\end{equation}
where $\omega_c=e B/m^*$ is known as the cyclotron frequency.

The final Hamiltonian reads
\begin{align}
  \hat{H}&=\sum_{i=1}^{N_e} \bigg(  \frac{- \hbar^2}{2m^*} \nabla_i^2 + \overbrace{\frac{1}{2} m^* \omega_0^2 {\mathbf{r_i}}^2}^{\begin{smallmatrix}
  \text{Harmonic ocscillator} \\
  \text{potential}
\end{smallmatrix}} \bigg) + \overbrace{\frac{e^2}{4 \pi \epsilon_0 \epsilon_r} \sum_{i<j}\frac{1}{\abs{\mathbf{r_i}-\mathbf{r_j}}}}^{\begin{smallmatrix}
  \text{Coulomb} \\
  \text{interactions}
\end{smallmatrix}}  \nonumber \\
&+  \underbrace{\sum_{i=1}^{N_e} \left( \frac{1}{2} m^* \left( \frac{\omega_c}{2} \right)^2 {\mathbf{r_i}}^2 + \frac{1}{2}  \omega_c \hat{L}_z^{(i)}+ \frac{1  }{2} g_s^*  \omega_c \hat{S}_z^{(i)}\right)}_{\begin{smallmatrix}
  \text{single particle interactions} \\
  \text{with the magnetic field}
\end{smallmatrix}},
\end{align}

\subsection{Scaling the problem: Dimensionless form of $\hat{H}$ }
\label{sec:scaling}
%also look at what Simen did:  \url{http://folk.uio.no/simenkva/openfci/html/qdot_8cc.html}, he includes B field and this change lambda to lambda*.
In order to simplify the computation, the Hamiltonian can be rewritten on dimensionless form.
For this purpose, we introduce the following constants:
\begin{itemize}
 \item the oscillator frequency $\omega = \omega_0\sqrt{1+\omega_c^2/ (4\omega_0^2)}$,
 \item a new energy unit $\hbar \omega$,
\item 	a new length unit, the oscillator length defined by $l=\sqrt{\hbar /(m^* \omega)}$, also called the characteristic length unit.
\end{itemize}

We rewrite the Hamiltonian in dimensionless units using:\\
$$\mathbf{r} \longrightarrow \frac{\mathbf{r}}{l}, \quad \nabla \longrightarrow l \;\nabla \quad \text{and} \quad \hat{L}_z \longrightarrow \hat{L}_z$$ 


It leads to the following Hamiltonian:
\begin{align}
\hat{H}&=\sum_{i=1}^{N_e} \left(  -\frac{1}{2} \nabla_i^2 + \frac{1}{2} r_i^2 \right)  + \overbrace{\frac{e^2}{4 \pi \epsilon_0 \epsilon_r} \frac{1}{\hbar \omega l}}^{\begin{smallmatrix}
  \text{Dimensionless} \\
 \text{confinement } \\
  \text{strength ($\lambda$)}
\end{smallmatrix}}
\sum_{i<j}\frac{1}{r_{ij}}  \nonumber \\
&+  \sum_{i=1}^{N_e} \left(  \frac{1}{2}  \frac{\omega_c}{\hbar \omega} \hat{L}_z^{(i)}+ \frac{1  }{2} g_s^* \frac{\omega_c}{\hbar \omega} \hat{S}_z^{(i)}\right),
\end{align}
Lengths are now measured in units of $l=\sqrt{\hbar/(m^*\omega)}$, and energies in units of $\hbar \omega$.
 
A new dimensionless parameter $\lambda=l / a_0^*$ (where $a_0^*= 4 \pi \epsilon_0 \epsilon_r \hbar^2 / (e^2 m^*)$ is the effective Bohr radius) describes the strength of the electron-electron interaction.
Large $\lambda$ implies strong interaction and/or large quantum dot~\cite{Tavernier2003}. Since both $\hat{L_z}$ and $\hat{S_z}$ commute with the Hamiltonian we can perform the calculations separately in subspaces of given quantum numbers $L_z$ and $S_z$.
Figure \ref{fig:valuesLambda} displays values of the different parameters as a function of the magnetic field strength for a particular type of semiconductor: Gallium arsenide (GaAs) with know characteristics given in table~\ref{table:effectiveMass}.



The simplified dimensionless Hamiltonian becomes
\begin{equation}
\label{eq:dimlessH}
  \hat{H}=\sum_{i=1}^{N_e} \left[  -\frac{1}{2} \nabla_i^2 + \frac{1}{2} r_i^2  \right]+ \lambda \sum_{i<j}\frac{1}{r_{ij}} +  \sum_{i=1}^{N_e} \left(  \frac{1}{2}  \frac{\omega_c}{\hbar \omega} L_z^{(i)}+ \frac{1  }{2} g_s^* \frac{\omega_c}{\hbar \omega} S_z^{(i)}\right),
\end{equation}

The last sum which is proportional to the magnetic field involves only the quantum numbers $L_z$ and $S_z$ and not the operators themselves~\cite{Tavernier2003}. Therefore these terms can be put aside during the resolution, the squizzing effect of the magnetic field being included simply in the parameter $\lambda$. The contribution of these terms will be added when the other part has been solved. This brings us to the simple and general form of the Hamiltonian:
\begin{equation}
\label{eq:lambdaSimp}
\hat{H}=\sum_{i=1}^{N_e} \left(  -\frac{1}{2} \nabla_i^2 + \frac{1}{2} r_i^2  \right)+ \lambda \sum_{i<j}\frac{1}{r_{ij}}.
\end{equation}



\subsection{Hamiltonian for nuclei}

Nuclei are selfbound system with no naturally defined center-of-mass as electrons in atoms have.
Momentum conservation requires 
that a many-body wave function must factorize
as $\Phi(\bf{r}) = \phi(R)\psi(\bf{r}_{\rm rel})$ where 
$R$ is the center-of-mass (CoM) coordinate and $\bf{r}_{\rm rel}$ the 
relative coordinates. If we choose to expand our wave functions in
the harmonic oscillator basis, then we are able to exactly separate the
center-of-mass motion from the problem provided that we work in a model
space that includes all $n\hbar\Omega$ excitations. Our problem is however that we normally will never be fully able
to perform a calculation that includes all possible excitations. This means that there may be so-called spurious
center-of-mass contaminations in the final solutions. These contaminations are unphysical and one needs therefore
recipes for handling them. We will come back to this in chapters \ref{chap:coupledcluster}, \ref{chap:shellmodel}, \ref{chap:vmc},
\ref{chap:dmc} and \ref{chap:smmc}.  Here we limit  ourselves to setting up a  
a translationally invariant Hamiltonian.
The CoM momentum is
\begin{equation}
   P=\sum_{i=1}^A{\bf p}_i,
\end{equation}
and we have that
\begin{equation}
\sum_{i=1}^A{\bf p}_i^2 =
\frac{1}{A}\left[{\bf P}^2+\sum_{i<j}({\bf p}_i-{\bf p}_j)^2\right]
\end{equation}
meaning that
\begin{equation}
\left[\sum_{i=1}^A\frac{{\bf p}_i^2}{2m} -\frac{{\bf P}^2}{2mA}\right]
=\frac{1}{2mA}\sum_{i<j}({\bf p}_i-{\bf p}_j)^2.
\end{equation}
The last expression is explicitely translationally invariant.

In a similar fashion we can define the CoM coordinate
\begin{equation}
    {\bf R}=\frac{1}{A}\sum_{i=1}^{A}x_i,
\end{equation}
which yields
\begin{equation}
\sum_{i=1}^Ax_i^2 =
\frac{1}{A}\left[A^2{\bf R}^2+\sum_{i<j}^A(x_i-x_j)^2\right].
\end{equation}


 
 In nuclear physics  studies the translationally invariant one- and two-body 
 Hamiltonian reads
 for an A-nucleon system, assuming an averaged mass\footnote{Further definitions that we will use throughout the text are $m_p=939$ MeV for 
the mass of the proton and $m_n=988$ MeV for the mass of the neutron. (correct numbers)}  $m=(m_p+m_n)/2$,
 %
 \begin{equation}
\label{eq:ham}
 H=\left[\sum_{i=1}^A\frac{{\bf p}_i^2}{2m} -\frac{{\bf P}^2}{2mA}\right] +\sum_{i<j}^A V(r_{ij}), 
 \end{equation}
 %
 where $V(r_{ij})$ is the nucleon-nucleon interaction,
and  the center-of-mass  momentum is
${\bf P}=\sum_{i=1}^A{\bf p}_i$. 
Since we want to employ a harmonic oscillator single-particle basis, it is convenient to use the relation
 \begin{equation}
 \sum_{i=1}^A\frac{1}{2}m\Omega^2x_i^2-
 \frac{m\Omega^2}{2A}\left[A^2{\bf R}^2+\sum_{i<j}(x_i-x_j)^2\right]=0,
 \end{equation}
where ${\bf R}={1}/{A}\sum_{i=1}^{A}x_i$
is the center-of-mass coordinate and $\Omega$ is the oscillator frequency with dimension inverse length and units of fm$^{-1}$. 
Using this relation, we can rewrite the above Hamiltonian as
$H(\Omega)=H_0+H_1-H_{\mathrm{CoM}}$ with the single-particle part $H_0$ given by the harmonic
oscillator Hamiltonian
\begin{equation}
 H_0=\sum_{i=1}^A \left[ \frac{{\bf p}_i^2}{2m} +\frac{1}{2}m\Omega^2 {\bf r}^2_i\right], 
\end{equation}
the interaction part defined by
\begin{equation}
 H_1=\sum_{i<j}^A \left[ V_{ij}-\frac{m\Omega^2}{2A}(x_i-x_j)^2\right],
\end{equation}
and finally the center-of-mass term
 \begin{equation}
      H_{\mathrm{CoM}}= \frac{{\bf P}^2}{2mA}+\frac{mA\Omega^2{\bf R}^2}{2}.
 \end{equation}
We can then rewrite our Hamiltonian as
 \begin{equation}
\label{eq:nocoreph}
 H(\Omega)=\sum_{i=1}^A\left[ \frac{{\bf p}_i^2}{2m}
 +\frac{1}{2}m\Omega^2 {\bf r}^2_i
 \right] + \sum_{i<j}^A \left[ V_{ij}-\frac{m\Omega^2}{2A}
 (x_i-x_j)^2
 \right]_{\rm eff} 
 -H_{\mathrm{CoM}}.
 \end{equation}

It is useful to study the $A$-dependent two-nucleon Hamiltonian (omitting the center-of-mass term),
 \begin{equation}
 H_{A=2}(\Omega)= \frac{{\bf p}_1^2+{\bf p}_2^2}{2m}
 +\frac{1}{2}m\Omega^2 ({\bf r}^2_1+{\bf r}^2_2)
 + V(x_1-x_2)-\frac{m\Omega^2}{2A}(x_1-x_2)^2,
 \end{equation}
which can be conveniently separated in the coordinates of the relative and center-of-mass
motion as
 \begin{equation}
 H_2(\Omega)= \frac{{\bf p}^2}{m}+\frac{{\bf P}^2}{4m}
 +\frac{1}{4}m\Omega^2 {\bf r}^2
 +m\Omega^2 {\bf r}^2
 + V({\bf r})-\frac{m\Omega^2}{2A}{\bf r}^2,
 \end{equation}
with ${\bf p}$ and ${\bf r}$ being the relative momentum and coordinate, respectively.
The corresponding Schr\"odinger equation can in turn be separated in an equation for the center-of-mass motion 
and one for the relative motion. The equation for the center-of-mass motion 
reduces to a standard harmonic oscillator equation with eigenvalues $\hbar\Omega (2N_{\mathrm{CoM}}+L_{\mathrm{CoM}}+3/2)$,
where $N_{\mathrm{CoM}},L_{\mathrm{CoM}}$ are the respective quantum numbers of the center-of-mass motion.
What remains to be solved is an equation involving the relative motion only.  This can easily be done
by diagonalizing 
 \begin{equation}
 H_{\mathrm{relative}}(\Omega)= \frac{{\bf p}^2}{m}
 +\frac{1}{4}m\Omega^2{\bf r}^2 
 + V({\bf r})-\frac{m\Omega^2}{2A}{\bf r}^2,
 \end{equation}
in the harmonic oscillator basis of the relative motion.
The two-body basis for solving this equation should, in principle, let the quantum numbers of the relative
motion $n,l$ go to infinity. This problem can be solved to a very high degree of accurary numerically and you are going
to study the solution ofthis equation with different nucleon-nucleon interaction models.   
This is a very useful test of your codes, in particular, if you don't reproduce the experimental
energy of the $A=2$ problem, the deuteron, you know you have made some errors. 


An alternative to the above harmonic oscillator problem   is to 
rewrite the internal kinetic energy as
\begin{equation}
 \left[\sum_{i=1}^A\frac{{\bf p}_i^2}{2m} -\frac{{\bf P}^2}{2mA}\right] = \left( 1 - {1\over A}\right) \sum_{i=1}^A 
  { {\bf p}_i^2\over 2m } - \sum_{i<j}^A { {\bf p}_i \cdot {\bf p}_j \over mA }. 
\end{equation}  
The last term results in the introduction of an additional two-body term. It  yields a modified two-body interaction
\begin{equation}
 \hat{V}  = 
 \sum_{i<j}^A \left (V(i,j)-\frac{{\bf p}_i \cdot {\bf p}_j}{mA}\right),
\end{equation}
resulting in a total Hamiltonian given by 
\begin{equation}
  H = \left( 1 - {1\over A}\right) \sum_{i=1}^A { {\bf p}_i^2\over 2m } +
 \sum_{i<j}^A \left (V(i,j)-\frac{{\bf p}_i \cdot {\bf p}_j}{mA}\right), 
\end{equation}
These interactions can in turn be written out in terms of harmonic oscillator elements or  
more generalized complex wave functions.  

\section{Exercises}
\begin{prob}
%\subsection*{Exercise 1}
Consider the Slater determinant
\[
\Phi_{\lambda}^{AS}(x_{1}x_{2}\dots x_{N};\alpha_{1}\alpha_{2}\dots\alpha_{N})
=\frac{1}{\sqrt{N!}}\sum_{p}(-)^{p}P\prod_{i=1}^{N}\psi_{\alpha_{i}}(x_{i}).
\]
where $P$ is an operator which permutes the coordinates of two particles. We have assumed here that the 
number of particles is the same as the number of available single-particle states, represented by the
greek letters $\alpha_{1}\alpha_{2}\dots\alpha_{N}$.
\begin{enumerate}
\item[a)] Write  out $\Phi^{AS}$ for $N=3$.  
\item[b)] Show that
\[
\int dx_{1}dx_{2}\dots dx_{N}\left\vert
\Phi_{\lambda}^{AS}(x_{1}x_{2}\dots x_{N};\alpha_{1}\alpha_{2}\dots\alpha_{N})
\right\vert^{2} = 1.
\]
\item[c)] Define a general onebody operator $\hat{F} = \sum_{i}^N\hat{f}(x_{i})$ and a general 
twobody operator $\hat{G}=\sum_{i>j}^N\hat{g}(x_{i},x_{j})$
with $g$ being invariant under the interchange of the coordinates of particles $i$ and $j$.
Calculate the matrix elements for a two-particle Slater determinant
\[
\bra{\Phi_{\alpha_{1}\alpha_{2}}^{AS}}\hat{F}\ket{\Phi_{\alpha_{1}\alpha_{2}}^{AS}},
\]
and
\[
\bra{\Phi_{\alpha_{1}\alpha_{2}}^{AS}}\hat{G}\ket{\Phi_{\alpha_{1}\alpha_{2}}^{AS}}.
\]
Explain the short-hand notation for the Slater determinant.
Which properties do you expect these operators to have in addition to an eventual permutation
symmetry?
\item[d)] Compute the corresponding matrix elements for $N$ particles which can occupy $N$ single particle states.
\end{enumerate}
\end{prob}
\begin{prob}
We will now consider a simple three-level problem, depicted in the figure below. 
The single-particle states are labelled by the quantum number $p$ and can accomodate up to two single particles, 
viz., every single-particle state 
is doubly degenerate (you could think of this as one state having spin up and the other spin down). 
We let the spacing between the doubly degenerate single-particle states be constant, with value $d$.  The first state
has energy $d$. There are only three available single-particle states, $p=1$, $p=2$ and $p=3$, as illustrated
in the figure. 
\begin{enumerate}
\item[a)] How many two-particle Slater determinants can we construct in this space? 
\item[b)] We limit ourselves to a system with only the two lowest single-particle orbits and two particles, $p=1$ and $p=2$.
We assume that we can write the Hamiltonian as
\[
       \hat{H}=\hat{H}_0+\hat{H}_I,
\]
and that the onebody part of the Hamiltonian with single-particle operator $\hat{h}_0$ has the property
\[
\hat{h}_0\psi_{p\sigma} = p\times d \psi_{p\sigma},
\]
where we have added a spin quantum number $\sigma$. 
We assume also that the only two-particle states that can exist are those where two particles are in the 
same state $p$, as shown by the two possibilities to the left in the figure.
The two-particle matrix elements of $\hat{H}_I$ have all a constant value, $-g$.
\begin{figure}
\vspace{1.0cm}
 \setlength{\unitlength}{1cm}
 \begin{picture}(15,5)
 \thicklines
\put(-0.6,1){\makebox(0,0){$p=1$}}
\put(-0.6,2){\makebox(0,0){$p=2$}}
\put(-0.6,3){\makebox(0,0){$p=3$}}
% first 2-particle state
\put(0.8,1){\circle*{0.3}}
\put(1.7,1){\circle*{0.3}}
% second 2-particle state
\put(5.0,2){\circle*{0.3}}
\put(5.9,2){\circle*{0.3}}
% third 2-particle state
\put(9.2,1){\circle*{0.3}}
\put(10.1,3){\circle*{0.3}}
% third 2-particle state
\put(13.4,2){\circle*{0.3}}
\put(14.3,3){\circle*{0.3}}
\dashline[+1]{2.5}(0,1)(15,1)
\dashline[+1]{2.5}(0,2)(15,2)
\dashline[+1]{2.5}(0,3)(15,3)
 \end{picture}
\caption{Schematic plot of the possible single-particle levels with double degeneracy.
The filled circles indicate occupied particle states.
The spacing between each level $p$ is constant in this picture. We show some possible two-particle states.}
\end{figure}
Show then that the Hamiltonian matrix can be written as 
\[
\left(\begin{array}{cc}2d-g &-g \\
-g &4d-g \end{array}\right),
\]
and find the eigenvalues and eigenvectors.  What is mixing of the state with two particles in $p=2$ 
to the wave function with two-particles in $p=1$? Discuss your results in terms of a linear combination
of Slater determinants.  \\
\item[c)] Add the possibility that the two particles can be in the state with $p=3$ as well and find the Hamiltonian
matrix, the eigenvalues and the eigenvectors. We still insist that we only have two-particle states composed of two particles being in the same
level $p$. You can diagonalize numerically your $3\times 3$ matrix.\newline\newline
This simple model catches several birds with a stone. It demonstrates how we can build linear combinations
of Slater determinants and interpret these as different admixtures to a given state. It represents also the way we are going to interpret these contributions.  The two-particle states above $p=1$ will be interpreted as 
excitations from the ground state configuration, $p=1$ here.  The reliability of this ansatz for the ground state, 
with two particles in $p=1$,
depends on the strength of the interaction $g$ and the single-particle spacing $d$.
Finally, this model is a simple schematic ansatz for studies of pairing correlations and thereby superfluidity/superconductivity  
in fermionic systems. 
\end{enumerate}
\end{prob}
