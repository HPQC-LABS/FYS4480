\chapter{Slater determinants and Hartree-Fock theory}

\section{Introduction}

In this chapter we will derive the Hartree-Fock equations, providing us thereby with our first approach
to many-body theories. The Hartree-Fock approach 
results in  a set of coupled single-particle equations where every nucleon sees an
average field set up by all other nucleons. In the previous chapter we limited our attention to single-particle potentials
like the harmonic oscillator or the Woods-Saxon potential\footnote{For protons we need also to account for the Coulomb  interaction as 
well.} 
HF theory is an algorithm for a finding an approximative expression
for the ground state of a given Hamiltonian, yielding also a set of
coupled single-particle equations that are solved self-consistently.
The basic ingredients are
\begin{itemize}
\item Define a single-particle basis $\{\psi_{\alpha}\}$ so that
\[ \hat{h}^{\mathrm{HF}}\psi_{\alpha} = \varepsilon_{\alpha}\psi_{\alpha}\]
with
\[
\hat{h}^{\mathrm{HF}}=\hat{t}+\hat{u}_{\mathrm{ext}}+\hat{u}^{\mathrm{HF}}
\]
\item where $\hat{u}^{\mathrm{HF}}$ is a single-particle potential to
  be determined by the HF algorithm.
\item The HF algorithm means to choose $\hat{u}^{\mathrm{HF}}$ in
  order to have
\[ \langle \hat{H} \rangle = E^{\mathrm{HF}}= \langle \Phi_0 | \hat{H}|\Phi_0 \rangle\]
a local minimum with $\Phi_0$ being the SD ansatz for the ground
state.
\item The variational principle ensures that $E^{\mathrm{HF}} \ge
  E_0$, $E_0$ being the exact ground state energy.
\end{itemize}
In order to derive the Hartree-Fock equarions, we will need to
introduce our first ansatz for a many-particle state, the so-called
Slater determinant which is constructed using a set of orthogonal and
normalized single-particle states $\{\phi_{\alpha}\}$. The latter is
normally chosen as a the set of eigenstates of a given one-body
Hamiltonian $\hat{h}_0=\hat{t}+\hat{u}_{\mathrm{ext}}$. In nuclear
physics, as discussed in the previous chapter, the three-dimensional
harmonic oscillator is a standard choice for the external one-body
potential.  In atomic physics, the standard choice is the attractive
Coulomb potential set up by the atomic nucleus with charge $Z$.

\section{Slater determinants and Hamiltonians}
Our Hamiltonian is invariant under the permutation (interchange) of
two nucleons.  If we let $\hat{P}{ij}$ be an operator which
interchanges two nucleons.  Due to the symmetries we have ascribed to
our Hamiltonian, this operator commutes with the total Hamiltonian,
\[
[\hat{H},\hat{P}] = 0,
\]
meaning that $\Psi_{\lambda}(x_1, x_2, \dots , x_A)$ is an
eigenfunction of $\hat{P}$ as well, that is
\[
\hat{P}_{ij}\Psi_{\lambda}(x_1, x_2, \dots,x_i,\dots,x_j,\dots,x_A)=
\Psi_{\lambda}(x_1, x_2, \dots,x_j,\dots,x_i,\dots,x_A).
\]
We have introduced the
suffix $ij$ in order to indicate that we permute nucleons $i$ and $j$.
The Pauli principle tells us that the total wave function for a system
of fermions has to be antisymmetric, resulting in the eigenvalue
$\beta = -1$.

The Schr\"odinger equation reads
\begin{equation}
\hat{H}(x_1, x_2, \dots , x_A) \Psi_{\lambda}(x_1, x_2, \dots , x_A) =
E_\lambda \Psi_\lambda(x_1, x_2, \dots , x_A),
\label{eq:basicSE1}
\end{equation}
where the vector $x_i$ represents the coordinates (spatial and spin)
of nucleon $i$, $\lambda$ stands for all the quantum numbers needed to
classify a given $A$-nucleon state and $\Psi_{\lambda}$ is the
pertaining eigenfunction.  Throughout this course, $\Psi$ refers to
the exact eigenfunction, unless otherwise stated.

We write the Hamilton operator, or Hamiltonian, in a generic way
\[
	\hat{H} = \hat{T} + \hat{V}
\]
where $\hat{T}$ represents the kinetic energy of the system
\[
	\hat{T} = \sum_{i=1}^A \frac{\mathbf{p}_i^2}{2m_i} =
        \sum_{i=1}^A \left( -\frac{\hbar^2}{2m_i} \mathbf{\nabla_i}^2
        \right) = \sum_{i=1}^A t(x_i)
\]
while the operator $\hat{V}$ for the potential energy is given by
\begin{equation}
	\hat{V} = \sum_{i=1}^A \hat{u}_{\mathrm{ext}}(x_i) +
        \sum_{ji=1}^A v(x_i,x_j)+\sum_{ijk=1}^Av(x_i,x_j,x_k)+\dots
\label{eq:firstv}
\end{equation}
Hereafter we use natural units, viz.~$\hbar=c=e=1$, with $e$ the
elementary charge and $c$ the speed of light. This means that momenta
and masses have dimension energy.

If one does quantum chemistry, after having introduced the
Born-Oppenheimer approximation which effectively freezes out the
nucleonic degrees of freedom, the Hamiltonian for $n_e$ electrons
takes the following form
\[
  \hat{H} = \sum_{i=1}^{n_e} t(x_i) - \sum_{i=1}^{n_e} k\frac{Z}{r_i}
  + \sum_{i<j}^{n_e} \frac{k}{r_{ij}},
\]
with $k=1.44$ eVnm

 We can rewrite this as
\begin{equation}
    \hat{H} = \hat{H_0} + \hat{H_I} = \sum_{i=1}^{n_e}\hat{h}_0(x_i) +
    \sum_{i<j=1}^{n_e}\frac{1}{r_{ij}},
\label{H1H2}
\end{equation}
where we have defined $r_{ij}=| {\bf r}_i-{\bf r}_j|$ and
\begin{equation}
  \hat{h}_0(x_i) = \hat{t}(x_i) - \frac{Z}{x_i}.
\label{hi}
\end{equation}
The first term of eq.~(\ref{H1H2}), $H_0$, is the sum of the $N$
\emph{one-body} Hamiltonians $\hat{h}_0$. Each individual Hamiltonian
$\hat{h}_0$ contains the kinetic energy operator of an electron and
its potential energy due to the attraction of the nucleus. The second
term, $H_I$, is the sum of the $n_e(n_e-1)/2$ two-body interactions
between each pair of electrons. Note that the double sum carries a
restriction $i<j$.

The potential energy term due to the attraction of the nucleus defines
the one-body field $u_i=u_{\mathrm{ext}}(x_i)$ of
Eq.~(\ref{eq:firstv}).  We have moved this term into the $\hat{H}_0$
part of the Hamiltonian, instead of keeping it in $\hat{V}$ as in
Eq.~(\ref{eq:firstv}).  The reason is that we will hereafter treat
$\hat{H}_0$ as our non-interacting Hamiltonian. For a many-body
wavefunction $\Phi_{\lambda}$ defined by an appropriate single-nucleon
basis, we may solve exactly the non-interacting eigenvalue problem
\[
\hat{H}_0\Phi_{\lambda}= w_{\lambda}\Phi_{\lambda},
\]
with $w_{\lambda}$ being the non-interacting energy. This energy is
defined by the sum over single-nucleon energies to be defined below.
For atoms the single-nucleon energies could be the hydrogen-like
single-nucleon energies corrected for the charge $Z$. For nuclei and
quantum dots, these energies could be given by the harmonic oscillator
in three and two dimensions, respectively.

We will assume that the interacting part of the Hamiltonian can be
approximated by a two-body interaction.  This means that our
Hamiltonian is written as
\begin{equation}
    \hat{H} = \hat{H_0} + \hat{H_I} = \sum_{i=1}^A \hat{h}_0(x_i) +
    \sum_{i<j=1}^A \hat{v}(x_{ij}),
\label{Hnuclei}
\end{equation}
with
\begin{equation}
  H_0=\sum_{i=1}^A \hat{h}_0(x_i) = \sum_{i=1}^A\left(\hat{t}(x_i) +
  \hat{u}_{\mathrm{ext}}(x_i)\right).
\label{hinuclei}
\end{equation}
The one-body part $u_{\mathrm{ext}}(x_i)$ is normally approximated by
a harmonic oscillator potential or the Coulomb interaction an electron
feels from the nucleus. However, other potentials are fully possible,
such as one derived from the self-consistent solution of the
Hartree-Fock equations or so-called Woods-Saxon potentials to be
discussed this week (and later in this course).

We have defined
\[
    \hat{H} = \hat{H_0} + \hat{H_I} = \sum_{i=1}^A \hat{h}_0(x_i) +
    \sum_{i<j=1}^A \hat{v}(x_{ij}),
\]
with
\[
  H_0=\sum_{i=1}^A \hat{h}_0(x_i) = \sum_{i=1}^A\left(\hat{t}(x_i) +
  \hat{u}_{\mathrm{ext}}(x_i)\right).
\]

In nuclear physics the one-body part $u_{\mathrm{ext}}(x_i)$ is often
approximated by a harmonic oscillator potential or a Woods-Saxon
potential. However, this is not fully correct, because as we have
discussed, nuclei are self-bound systems and there is no external
confining potential. As we will see below, the Hamiltonian $H_0$ cannot be used to
  compute the total binding energy of a system like a nucleus since it is not based on a
  model for the nuclear forces. This applies to other many-particle systems as well, like atoms, molecules, or 
quantum dots. It means that the binding energy is not
the sum of the individual single-particle energies.

Nuclei are self-bound systems due to the strong nuclear force. A single-particle potential like the harmonic
oscillator is thus to be seen only as a potential that provides a basis of single-particle states that can be used
in computations of complicated matrix elements. In order to introduce the harmonic oscillator, we need 
to add and subtract a harmonic oscillator
potential to the total Hamiltonian. The external potential due to a harmonic oscillator reads
\[
\hat{u}_{\mathrm{ext}}(x_i)=\hat{u}_{\mathrm{ho}}(x_i)=
\frac{1}{2}m\omega^2 r_i^2,
\]
where $\omega$ is the oscillator frequency. Adding and subtracting the harmonic oscillator leads to 
\[
    \hat{H} = \hat{H_0} + \hat{H_I} = \sum_{i=1}^A \hat{h}_0(x_i) +
    \sum_{i<j=1}^A
    \hat{v}(x_{ij})-\sum_{i=1}^A\hat{u}_{\mathrm{ho}}(x_i),
\]
with
\[
  H_0=\sum_{i=1}^A \hat{h}_0(x_i) = \sum_{i=1}^A\left(\hat{t}(x_i) +
  \hat{u}_{\mathrm{ho}}(x_i)\right).
\]
Many practitioners use this as the standard Hamiltonian when doing
nuclear structure calculations with a harmonic oscillator basis.  
However, the one-body part of the total Hamiltonian $\hat{H}_0$ is no longer translationally invariant.
If the number of nucleons
is large, the breaking of this important invariance is mild or can be neglected. We can however remedy this deficiency from the following observations.
 In setting up a translationally invariant Hamiltonian, we note that the center-of-mass (CoM) momentum is given by
 \[
    P=\sum_{i=1}^A\vec{p}_i,
 \]
 and we have that
 \[
 \sum_{i=1}^A\vec{p}_i^2 =
 \frac{1}{A}\left[\vec{P}^2+\sum_{i<j}(\vec{p}_i-\vec{p}_j)^2\right]
 \]
 meaning that
 \[
 \left[\sum_{i=1}^A\frac{\vec{p}_i^2}{2m}
   -\frac{\vec{P}^2}{2mA}\right]
 =\frac{1}{2mA}\sum_{i<j}(\vec{p}_i-\vec{p}_j)^2,
 \]
which respects translational invariance.
 In a similar fashion we can define the CoM coordinate
 \[
     \vec{R}=\frac{1}{A}\sum_{i=1}^{A}\vec{r}_i,
 \]
 which yields
 \[
 \sum_{i=1}^A\vec{r}_i^2 =
 \frac{1}{A}\left[A^2\vec{R}^2+\sum_{i<j}(\vec{r}_i-\vec{r}_j)^2\right].
 \]
 If we then introduce the harmonic oscillator one-body Hamiltonian
 \[
      H_0= \sum_{i=1}^A\left(\frac{\vec{p}_i^2}{2m}+
      \frac{1}{2}m\omega^2\vec{r}_i^2\right),
 \]
 with $\omega$ the oscillator frequency, we can rewrite the latter as
 \[
      H_{\mathrm{HO}}=
      \frac{\vec{P}^2}{2mA}+\frac{mA\omega^2\vec{R}^2}{2}
      +\frac{1}{2mA}\sum_{i<j}(\vec{p}_i-\vec{p}_j)^2
      +\frac{m\omega^2}{2A}\sum_{i<j}(\vec{r}_i-\vec{r}_j)^2.
     \label{eq:obho}
 \]
 Or we could write
 \[
 H_{\mathrm{HO}}=
 H_{\mathrm{CoM}}+\frac{1}{2mA}\sum_{i<j}(\vec{p}_i-\vec{p}_j)^2
 +\frac{m\omega^2}{2A}\sum_{i<j}(\vec{r}_i-\vec{r}_j)^2,
 \]
 with
 \[
      H_{\mathrm{CoM}}=
      \frac{\vec{P}^2}{2mA}+\frac{mA\omega^2\vec{R}^2}{2}.
 \]
 The translationally invariant one- and two-body Hamiltonian read for
 an A-nucleon system, %
 \[\label{eq:ham}
\hat{H}=\left[\sum_{i=1}^A\frac{\vec{p}_i^2}{2m}-\frac{\vec{P}^2}{2mA}\right] +\sum_{i<j}^A V_{ij},
 \]
 where $V_{ij}$ is the nucleon-nucleon interaction. Adding zero in the following form
 \[
 \sum_{i=1}^A\frac{1}{2}m\omega^2\vec{r}_i^2-
 \frac{m\omega^2}{2A}\left[\vec{R}^2+\sum_{i<j}(\vec{r}_i-\vec{r}_j)^2\right]=0,
 \]
allows us to rewrite the Hamiltonian as
 \[
 \hat{H}=\sum_{i=1}^A \left[ \frac{\vec{p}_i^2}{2m}
   +\frac{1}{2}m\omega^2 \vec{r}^2_i \right] + \sum_{i<j}^A \left[
   V_{ij}-\frac{m\omega^2}{2A} (\vec{r}_i-\vec{r}_j)^2 \right]-H_{\mathrm{CoM}}.
 \]


In our case we assume that we can approximate the exact eigenfunction
with a Slater determinant 
\begin{equation} 
\Phi(x_1, x_2,\dots,x_A,\alpha,\beta,\dots, \sigma)=\frac{1}{\sqrt{A!}}
\left| \begin{array}{ccccc} \psi_{\alpha}(x_1)& \psi_{\alpha}(x_2)&
  \dots & \dots &
  \psi_{\alpha}(x_A)\\ \psi_{\beta}(x_1)&\psi_{\beta}(x_2)& \dots &
  \dots & \psi_{\beta}(x_A)\\ \dots & \dots & \dots & \dots & \dots
  \\ \dots & \dots & \dots & \dots & \dots
  \\ \psi_{\sigma}(x_1)&\psi_{\sigma}(x_2)& \dots & \dots &
  \psi_{\sigma}(x_A)\end{array} \right|,
\label{HartreeFockDet}
\end{equation} 
where $x_i$ stand for the coordinates and spin values of a nucleon
$i$ and $\alpha,\beta,\dots, \gamma$ are quantum numbers needed to
describe remaining quantum numbers.
\begin{enumerate}
 \item Show that the Slater determinant is normalized.
 \item Show that the density of electrons with coordinates
   $\mathbf{x}$, is given by
 \begin{equation}
  n(\mathbf{x}) = N \int d\mathbf{x}_2 \dots d\mathbf{x}_N
  |\Psi_{AS}(\mathbf{x},\mathbf{x}_2,\dots,\mathbf{x}_N)|^2
\end{equation}
can be written in terms of the $\psi_k$ as
\begin{equation}
 n(\mathbf{x}) = \sum_k|\psi_k(\mathbf{x})|^2.
\end{equation}
\end{enumerate}


The single-nucleon function $\psi_{\alpha}(x_i)$ are eigenfunctions of
the one-body Hamiltonian, that is
\[
\hat{h}_0(x_i)=\hat{t}(x_i) + \hat{u}_{\mathrm{ho}}(x_i),
\]
with eigenvalues
\[
\hat{h}_0(x_i) \psi_{\alpha}(x_i)=\left(\hat{t}(x_i) +
\hat{u}_{\mathrm{ho}}(x_i)\right)\psi_{\alpha}(x_i)=\varepsilon_{\alpha}\psi_{\alpha}(x_i).
\]
The energies $\varepsilon_{\alpha}$ are the so-called non-interacting
single-nucleon energies, or unperturbed energies.  The total energy is
in this case the sum over all single-nucleon energies, if no two-body
or more complicated many-body interactions are present.

Let us denote the ground state energy by $E_0$. According to the
variational principle we have
\begin{equation*}
  E_0 \le E[\Phi] = \int \Phi^*\hat{H}\Phi d\mathbf{\tau}
\end{equation*}
where $\Phi$ is a trial function which we assume to be normalized
\begin{equation*}
  \int \Phi^*\Phi d\mathbf{\tau} = 1,
\end{equation*}
where we have used the shorthand
$d\mathbf{\tau}=d\mathbf{x}_1d\mathbf{x}_2\dots d\mathbf{x}_A$.

In the Hartree-Fock method the trial function is the Slater
determinant of Eq.~(\ref{HartreeFockDet}) which can be rewritten as
\begin{equation}
  \Phi(x_1,x_2,\dots,x_A,\alpha,\beta,\dots,\nu) =
  \frac{1}{\sqrt{A!}}\sum_{P} (-)^P\hat{P}\psi_{\alpha}(x_1)
  \psi_{\beta}(x_2)\dots\psi_{\nu}(x_A)=\sqrt{A!}{\cal A}\Phi_H,
\label{HartreeFockPermutation}
\end{equation}
where we have introduced the antisymmetrization operator ${\cal A}$
defined by the summation over all possible permutations of two
nucleons.

It is defined as
\begin{equation}
  {\cal A} = \frac{1}{A!}\sum_{p} (-)^p\hat{P},
\label{antiSymmetryOperator}
\end{equation}
with $p$ standing for the number of permutations. We have introduced
for later use the so-called Hartree-function, defined by the simple
product of all possible single-nucleon functions
\begin{equation*}
  \Phi_H(x_1,x_2,\dots,x_A,\alpha,\beta,\dots,\nu) =
  \psi_{\alpha}(x_1) \psi_{\beta}(x_2)\dots\psi_{\nu}(x_A).
\end{equation*}

Both $\hat{H_0}$ and $\hat{\hat{H}_I}$ are invariant under all
possible permutations of any two nucleons and hence commute with
${\cal A}$
\begin{equation}
  [H_0,{\cal A}] = [H_I,{\cal A}] = 0.
  \label{cummutionAntiSym}
\end{equation}
Furthermore, ${\cal A}$ satisfies
\begin{equation}
  {\cal A}^2 = {\cal A},
  \label{AntiSymSquared}
\end{equation}
since every permutation of the Slater determinant reproduces it.

The expectation value of $\hat{H_0}$
\[
  \int \Phi^*\hat{H_0}\Phi d\mathbf{\tau} = A! \int \Phi_H^*{\cal
    A}\hat{H_0}{\cal A}\Phi_H d\mathbf{\tau}
\]
is readily reduced to
\[
  \int \Phi^*\hat{H_0}\Phi d\mathbf{\tau} = A! \int
  \Phi_H^*\hat{H_0}{\cal A}\Phi_H d\mathbf{\tau},
\]
where we have used eqs.~(\ref{cummutionAntiSym}) and
(\ref{AntiSymSquared}). The next step is to replace the
antisymmetrization operator by its definition
Eq.~(\ref{HartreeFockPermutation}) and to replace $\hat{H_0}$ with the
sum of one-body operators
\[
  \int \Phi^*\hat{H_0}\Phi d\mathbf{\tau} = \sum_{i=1}^A \sum_{p}
  (-)^p\int \Phi_H^*\hat{h}_0\hat{P}\Phi_H d\mathbf{\tau}.
\]
The integral vanishes if two or more nucleons are permuted in only one
of the Hartree-functions $\Phi_H$ because the individual
single-nucleon wave functions are orthogonal. We obtain then
\[
  \int \Phi^*\hat{H}_0\Phi d\mathbf{\tau}= \sum_{i=1}^A \int
  \Phi_H^*\hat{h}_0\Phi_H d\mathbf{\tau}.
\]
Orthogonality of the single-nucleon functions allows us to further
simplify the integral, and we arrive at the following expression for
the expectation values of the sum of one-body Hamiltonians
\begin{equation}
  \int \Phi^*\hat{H}_0\Phi d\mathbf{\tau} = \sum_{\mu=1}^A \int
  \psi_{\mu}^*(\mathbf{x})\hat{h}_0\psi_{\mu}(\mathbf{x}) d\mathbf{x}.
  \label{H1Expectation}
\end{equation}
We introduce the following shorthand for the above integral
\[
\langle \mu | \hat{h}_0 | \mu \rangle = \int
\psi_{\mu}^*(\mathbf{x})\hat{h}_0\psi_{\mu}(\mathbf{x})d\mathbf{x}.,
\]
and rewrite Eq.~(\ref{H1Expectation}) as
\begin{equation}
  \int \Phi^*\hat{H_0}\Phi d\mathbf{\tau} = \sum_{\mu=1}^A \langle \mu
  | \hat{h}_0 | \mu \rangle.
  \label{H1Expectation1}
\end{equation}
The expectation value of the two-body part of the Hamiltonian
(assuming a two-body Hamiltonian at most) is obtained in a similar
manner. We have
\begin{equation*}
  \int \Phi^*\hat{H_I}\Phi d\mathbf{\tau} = A! \int \Phi_H^*{\cal
    A}\hat{H_I}{\cal A}\Phi_H d\mathbf{\tau},
\end{equation*}
which reduces to
\begin{equation*}
 \int \Phi^*\hat{H_I}\Phi d\mathbf{\tau} = \sum_{i\le j=1}^A \sum_{p}
 (-)^p\int \Phi_H^*\hat{v}(x_{ij})\hat{P}\Phi_H d\mathbf{\tau},
\end{equation*}
by following the same arguments as for the one-body Hamiltonian.
Because of the dependence on the inter-nucleon distance $r_{ij}$,
permutations of any two nucleons no longer vanish, and we get
\begin{equation*}
  \int \Phi^*\hat{H_I}\Phi d\mathbf{\tau} = \sum_{i < j=1}^A \int
  \Phi_H^*\hat{v}(x_{ij})(1-P_{ij})\Phi_H d\mathbf{\tau}.
\end{equation*}
where $P_{ij}$ is the permutation operator that interchanges nucleon
$i$ and nucleon $j$. Again we use the assumption that the
single-nucleon wave functions are orthogonal.  We obtain
\begin{equation}
\begin{split}
  \int \Phi^*\hat{H_I}\Phi d\mathbf{\tau} =
  \frac{1}{2}\sum_{\mu=1}^A\sum_{\nu=1}^A &\left[ \int
    \psi_{\mu}^*(x_i)\psi_{\nu}^*(x_j)\hat{v}(x_{ij})\psi_{\mu}(x_i)\psi_{\nu}(x_j)
    dx_idx_j \right.\\ &\left.  - \int
    \psi_{\mu}^*(x_i)\psi_{\nu}^*(x_j)
    \hat{v}(x_{ij})\psi_{\nu}(x_i)\psi_{\mu}(x_j) dx_idx_j
    \right]. \label{H2Expectation}
\end{split}
\end{equation}
The first term is the so-called direct term. In Hartree-Fock theory it
leads to the so-called Hartree term, while the second is due to the
Pauli principle and is called the exchange term and in Hartree-Fock
theory it defines the so-called xFock term.  The factor $1/2$ is
introduced because we now run over all pairs twice.  The last equation
allows us to introduce some further definitions.  The single-nucleon
wave functions $\psi_{\mu}({\bf x})$, defined by the quantum numbers
$\mu$ and ${\bf x}$ (recall that ${\bf x}$ also includes spin degree,
later we will also add isospin) are defined as the overlap
\[
   \psi_{\alpha}(x) = \langle x | \alpha \rangle .
\]
We introduce the following shorthands for the above two integrals
\[
\langle \mu\nu|V|\mu\nu\rangle = \int
\psi_{\mu}^*(x_i)\psi_{\nu}^*(x_j)\hat{v}(x_{ij})\psi_{\mu}(x_i)\psi_{\nu}(x_j)
dx_idx_j,
\]
and
\[
\langle \mu\nu|V|\nu\mu\rangle = \int
\psi_{\mu}^*(x_i)\psi_{\nu}^*(x_j)
\hat{v}(x_{ij})\psi_{\nu}(x_i)\psi_{\mu}(x_j) dx_idx_j.
\]
The direct and exchange matrix elements can be brought together if we
define the antisymmetrized matrix element
\[
\langle \mu\nu|V|\mu\nu\rangle_{\mathrm{AS}}= \langle
\mu\nu|V|\mu\nu\rangle-\langle \mu\nu|V|\nu\mu\rangle,
\]
or for a general matrix element
\[
\langle \mu\nu|V|\sigma\tau\rangle_{\mathrm{AS}}= \langle
\mu\nu|V|\sigma\tau\rangle-\langle \mu\nu|V|\tau\sigma\rangle.
\]
It has the symmetry property
\[
\langle \mu\nu|V|\sigma\tau\rangle_{\mathrm{AS}}= -\langle
\mu\nu|V|\tau\sigma\rangle_{\mathrm{AS}}=-\langle
\nu\mu|V|\sigma\tau\rangle_{\mathrm{AS}}.
\]
The antisymmetric matrix element is also hermitian, implying
\[
\langle \mu\nu|V|\sigma\tau\rangle_{\mathrm{AS}}= \langle
\sigma\tau|V|\mu\nu\rangle_{\mathrm{AS}}.
\]

With these notations we rewrite Eq.~(\ref{H2Expectation}) as
\begin{equation}
  \int \Phi^*\hat{H_I}\Phi d\mathbf{\tau} =
  \frac{1}{2}\sum_{\mu=1}^A\sum_{\nu=1}^A \langle
  \mu\nu|V|\mu\nu\rangle_{\mathrm{AS}}.
\label{H2Expectation2}
\end{equation}
Combining Eqs.~(\ref{H1Expectation1}) and (\ref{H2Expectation2}) we
obtain the energy functional
\begin{equation}
  E[\Phi] = \sum_{\mu=1}^A \langle \mu | \hat{h}_0 | \mu \rangle +
  \frac{1}{2}\sum_{{\mu}=1}^A\sum_{{\nu}=1}^A \langle
  \mu\nu|V|\mu\nu\rangle_{\mathrm{AS}}.
\label{FunctionalEPhi}
\end{equation}
which we will use as our starting point for the Hartree-Fock
calculations.


\section{Deriving the Hartree-Fock equations}

The Hartree-Fock method is an algorithm for a finding an approximative expression
for the ground state of a given Hamiltonian. The basic ingredients are
\begin{itemize}
\item Define a single-particle basis $\{\psi_{\alpha}\}$ so that
\[ \hat{h}^{\mathrm{HF}}\psi_{\alpha} = \varepsilon_{\alpha}\psi_{\alpha}\]
with
\[
\hat{h}^{\mathrm{HF}}=\hat{t}+\hat{u}_{\mathrm{ext}}+\hat{u}^{\mathrm{HF}}
\]
\item where $\hat{u}^{\mathrm{HF}}$ is a single-particle potential to
  be determined by the HF algorithm.
\item The HF algorithm means to choose $\hat{u}^{\mathrm{HF}}$ in
  order to have
\[ \langle \hat{H} \rangle = E^{\mathrm{HF}}= \langle \Phi_0 | \hat{H}|\Phi_0 \rangle\]
a local minimum with $\Phi_0$ being the SD ansatz for the ground
state.
\item The variational principle ensures that $E^{\mathrm{HF}} \ge
  \tilde{E}_0$, $\tilde{E}_0$ the exact ground state energy.
\end{itemize}

The calculus of variations involves problems where the quantity to be
minimized or maximized is an integral.

In the general case we have an integral of the type
\[ E[\Phi]= \int_a^b f(\Phi(x),\frac{\partial \Phi}{\partial x},x)dx,\]
where $E$ is the quantity which is sought minimized or maximized.  The
problem is that although $f$ is a function of the variables $\Phi$,
$\partial \Phi/\partial x$ and $x$, the exact dependence of $\Phi$ on
$x$ is not known.  This means again that even though the integral has
fixed limits $a$ and $b$, the path of integration is not known. In our
case the unknown quantities are the single-particle wave functions and
we wish to choose an integration path which makes the functional
$E[\Phi]$ stationary. This means that we want to find minima, or
maxima or saddle points. In physics we search normally for minima.
Our task is therefore to find the minimum of $E[\Phi]$ so that its
variation $\delta E$ is zero subject to specific constraints. In our
case the constraints appear as the integral which expresses the
orthogonality of the single-particle wave functions.  The constraints
can be treated via the technique of Lagrangian multipliers

We assume the existence of an optimum path, that is a path for which
$E[\Phi]$ is stationary. There are infinitely many such paths.  The
difference between two paths $\delta\Phi$ is called the variation of
$\Phi$.
\subsection{Simple example of variational calculus}
Let us specialize to the expectation value of the energy for one
particle in three-dimensions.  This expectation value reads
\[
  E=\int dxdydz \psi^*(x,y,z) \hat{H} \psi(x,y,z),
\]
with the constraint
\[
 \int dxdydz \psi^*(x,y,z) \psi(x,y,z)=1,
\]
and a Hamiltonian
\[
\hat{H}=-\frac{1}{2}\nabla^2+\hat{v}(x,y,z).
\]
I will skip the variables $x,y,z$ below, and write for example
$\hat{v}(x,y,z)=V$.

The integral involving the kinetic energy can be written as, if we
assume periodic boundary conditions or that the function $\psi$
vanishes strongly for large values of $x,y,z$,
 \[
  \int dxdydz \psi^* \left(-\frac{1}{2}\nabla^2\right) \psi dxdydz =
  -\psi^*\nabla\psi|+\int dxdydz\frac{1}{2}\nabla\psi^*\nabla\psi.
\]
Inserting this expression into the expectation value for the energy
and taking the variational minimum we obtain
\[
\delta E = \delta \left\{\int dxdydz\left(
\frac{1}{2}\nabla\psi^*\nabla\psi+V\psi^*\psi\right)\right\} = 0.
\]
The constraint appears in integral form as
\[
 \int dxdydz \psi^* \psi=\mathrm{constant},
\]
and multiplying with a Lagrangian multiplier $\lambda$ and taking the
variational minimum we obtain the final variational equation
\[
\delta \left\{\int dxdydz\left(
\frac{1}{2}\nabla\psi^*\nabla\psi+V\psi^*\psi-\lambda\psi^*\psi\right)\right\}
= 0.
\]
Introducing the function $f$
\[
  f = \frac{1}{2}\nabla\psi^*\nabla\psi+V\psi^*\psi-\lambda\psi^*\psi=
  \frac{1}{2}(\psi^*_x\psi_x+\psi^*_y\psi_y+\psi^*_z\psi_z)+V\psi^*\psi-\lambda\psi^*\psi,
\]
where we have skipped the dependence on $x,y,z$ and introduced the
shorthand $\psi_x$, $\psi_y$ and $\psi_z$ for the various derivatives.
For $\psi^*$ the Euler equation results in
\[
\frac{\partial f}{\partial \psi^*}- \frac{\partial }{\partial
  x}\frac{\partial f}{\partial \psi^*_x}-\frac{\partial }{\partial
  y}\frac{\partial f}{\partial \psi^*_y}-\frac{\partial }{\partial
  z}\frac{\partial f}{\partial \psi^*_z}=0,
\] 
which yields
\[
    -\frac{1}{2}(\psi_{xx}+\psi_{yy}+\psi_{zz})+V\psi=\lambda \psi.
\]
We can then identify the Lagrangian multiplier as the energy of the
system. Then the last equation is nothing but the standard
Schr\"odinger equation and the variational approach discussed here
provides a powerful method for obtaining approximate solutions of the
wave function.  We have the Hamiltonian
\[
    \hat{H} = \hat{H_0} + \hat{H_I} = \sum_{i=1}^A\hat{h_0}(x_i) +
    \sum_{i<j=1}^A\hat{v}(r_{ij}),
\]

\[
  \hat{h}_0(x_i) = - \frac{1}{2} \nabla^2_i
  +\frac{1}{2}m\omega^2r_i^2.
\]
Let us denote the ground state energy by $E_0$. According to the
variational principle we have
\begin{equation*}
  E_0 \le E[\Phi] = \int \Phi^*\hat{H}\Phi d\mathbf{\tau}
\end{equation*}
where $\Phi$ is a trial function which we assume to be normalized
\begin{equation*}
  \int \Phi^*\Phi d\mathbf{\tau} = 1,
\end{equation*}
where we have used the shorthand $d\mathbf{\tau}=dx_1dx_2\dots dx_A$.
In the Hartree-Fock method the trial function is the Slater
determinant which can be rewritten as
\[
  \Psi(x_1,x_2,\dots,x_A,\alpha,\beta,\dots,\nu) =
  \frac{1}{\sqrt{A!}}\sum_{P} (-)^PP\psi_{\alpha}(x_1)
  \psi_{\beta}(x_2)\dots\psi_{\nu}(x_A)=\sqrt{A!}{\cal A}\Phi_H,
\]
where we have introduced the anti-symmetrization operator ${\cal A}$
defined by the summation over all possible permutations of two
fermions.  It is defined as
\[
  {\cal A} = \frac{1}{A!}\sum_{P} (-)^PP,
\]
with the the Hartree-function given by the simple product of all
possible single-particle function
\[
  \Phi_H(x_1,x_2,\dots,x_A,\alpha,\beta,\dots,\nu) =
  \psi_{\alpha}(x_1) \psi_{\beta}(x_2)\dots\psi_{\nu}(x_A).
\]
Both $\hat{H_0}$ and $\hat{H_I}$ are invariant under permutations of
fermions, and hence commute with ${\cal A}$
\[
  [H_0,{\cal A}] = [H_I,{\cal A}] = 0.
\]
Furthermore, ${\cal A}$ satisfies
\[
  {\cal A}^2 = {\cal A},
\]
since every permutation of the Slater determinant reproduces it.  Our
functional is written as
\[
  E[\Phi] = \sum_{\mu=1}^A \int
  \psi_{\mu}^*(x_i)\hat{h}_0(x_i)\psi_{\mu}(x_i) dx_i +
  \frac{1}{2}\sum_{\mu=1}^A\sum_{\nu=1}^A \left[ \int
    \psi_{\mu}^*(x_i)\psi_{\nu}^*(x_j)\hat{v}(r_{ij})\psi_{\mu}(x_i)\psi_{\nu}(x_j)
    dx_idx_j \right.
\]
\[ \left.
  - \int
  \psi_{\mu}^*(x_i)\psi_{\nu}^*(x_j)\hat{v}(r_{ij})\psi_{\nu}(x_i)\psi_{\mu}(x_j)
  dx_idx_j\right]
\]
The more compact version is
\[
  E[\Phi] = \sum_{\mu=1}^A \langle \mu | \hat{h}_0 | \mu\rangle+
  \frac{1}{2}\sum_{\mu=1}^A\sum_{\nu=1}^A\left[\langle \mu\nu
    |\hat{v}(r_{ij})|\mu\nu\rangle-\langle \mu\nu
    |\hat{v}(r_{ij})|\nu\mu\rangle\right].
\]
If we generalize the Euler-Lagrange equations to more variables and
introduce $A^2$ Lagrange multipliers which we denote by
$\epsilon_{\mu\nu}$, we can write the variational equation for the
functional of $E$
\[
  \delta E - \sum_{{\mu}=1}^A\sum_{{\nu}=1}^A \epsilon_{\mu\nu} \delta
  \int \psi_{\mu}^* \psi_{\nu} = 0.
\]
For the orthogonal wave functions $\psi_{\mu}$ this reduces to
\[
  \delta E - \sum_{{\mu}=1}^A \epsilon_{\mu} \delta \int \psi_{\mu}^*
  \psi_{\mu} = 0.
\]
Variation with respect to the single-particle wave functions
$\psi_{\mu}$ yields then
\begin{equation*}
\begin{split}
  \sum_{\mu=1}^A \int \delta\psi_{\mu}^*\hat{h_0}(x_i)\psi_{\mu} dx_i
  + \frac{1}{2}\sum_{{\mu}=1}^A\sum_{{\nu}=1}^A \left[ \int
    \delta\psi_{\mu}^*\psi_{\nu}^*\hat{v}(r_{ij})\psi_{\mu}\psi_{\nu}
    dx_idx_j- \int
    \delta\psi_{\mu}^*\psi_{\nu}^*\frac{1}{r_{ij}}\psi_{\nu}\psi_{\mu}
    dx_idx_j \right] & \\ + \sum_{\mu=1}^A \int
  \psi_{\mu}^*\hat{h_0}(x_i)\delta\psi_{\mu} dx_i +
  \frac{1}{2}\sum_{{\mu}=1}^A\sum_{{\nu}=1}^A \left[ \int
    \psi_{\mu}^*\psi_{\nu}^*\frac{1}
        {r_{ij}}\delta\psi_{\mu}\psi_{\nu} dx_idx_j- \int
        \psi_{\mu}^*\psi_{\nu}^*\hat{v}(r_{ij})\psi_{\nu}\delta\psi_{\mu}
        dx_idx_j \right] & \\ - \sum_{{\mu}=1}^A E_{\mu} \int
  \delta\psi_{\mu}^* \psi_{\mu}dx_i - \sum_{{\mu}=1}^A E_{\mu} \int
  \psi_{\mu}^* \delta\psi_{\mu}dx_i & = 0.
\end{split}
\end{equation*}
Although the variations $\delta\psi$ and $\delta\psi^*$ are not
independent, they may in fact be treated as such, so that the terms
dependent on either $\delta\psi$ and $\delta\psi^*$ individually may
be set equal to zero. To see this, simply replace the arbitrary
variation $\delta\psi$ by $i\delta\psi$, so that $\delta\psi^*$ is
replaced by $-i\delta\psi^*$, and combine the two equations. We thus
arrive at the Hartree-Fock equations
\[
  \begin{split}
    \left[ -\frac{1}{2}\nabla_i^2-u_{\mathrm{ext}}(x_i) +
      \sum_{{\nu}=1}^A \int \psi_{\nu}^*(x_j)\hat{v}(r_{ij})
      \psi_{\nu}(x_j)dx_j \right] \psi_{\mu}(x_i) & \\ - \left[
      \sum_{{\nu}=1}^A \int \psi_{\nu}^*(x_j)
      \hat{v}(r_{ij})\psi_{\mu}(x_j) dx_j \right] \psi_{\nu}(x_i) & =
    \epsilon_{\mu} \psi_{\mu}(x_i).
  \end{split}
\]
Notice that the integration $\int dx_j$ implies an integration over
the spatial coordinates $\mathbf{r_j}$ and a summation over the
spin-coordinate of fermion $j$.  The two first terms are the
expectation value of the one-body operator. The third or \emph{direct}
term is the averaged field set up by all other nucleons.  As written,
the term includes the 'self-interaction' of nucleons when $i=j$. The
self-interaction is cancelled in the fourth term, or the
\emph{exchange} term. The exchange term results from our inclusion of
the Pauli principle and the assumed determinantal form of the
wave-function. The effect of exchange is for nucleons of equal
single-particle quantum numbers to avoid each other.

  A theoretically convenient form of the Hartree-Fock equation is to
  regard the direct and exchange operator defined through
\begin{equation*}
  V_{\mu}^{d}(x_i) = \int
  \psi_{\mu}^*(x_j)\psi_{\mu}(x_j)\hat{v}(r_{ij}) dx_j
\end{equation*}
and
\begin{equation*}
  V_{\mu}^{ex}(x_i) g(x_i) = \left(\int
  \psi_{\mu}^*(x_j)\hat{v}(r_{ij})g(x_j) dx_j \right)\psi_{\mu}(x_i),
\end{equation*}
respectively.  The function $g(x_i)$ is an arbitrary function, and by
the substitution $g(x_i) = \psi_{\nu}(x_i)$ we get
\begin{equation*}
  V_{\mu}^{ex}(x_i) \psi_{\nu}(x_i) = \left(\int \psi_{\mu}^*(x_j)
  \hat{v}(r_{ij})\psi_{\nu}(x_j) dx_j\right)\psi_{\mu}(x_i).
\end{equation*}
We may then rewrite the Hartree-Fock equations as
\[
  \hat{h}^{HF}(x_i) \psi_{\nu}(x_i) = \epsilon_{\nu}\psi_{\nu}(x_i),
\]
with
\[
  \hat{h}^{HF}(x_i)= \hat{h}_0(x_i) + \sum_{\mu=1}^AV_{\mu}^{d}(x_i) -
  \sum_{\mu=1}^AV_{\mu}^{ex}(x_i),
\]
and where $\hat{h}_0(i)$ is the one-body part. The latter is normally
chosen as a part which yields solutions in closed form. The harmonic
oscillator is a classical problem thereof.  We normally rewrite the
last equation as
\[
  \hat{h}^{HF}(x_i)= \hat{h}_0(x_i) + \hat{u}^{HF}(x_i).
\]
The last equation
\[
  \hat{h}^{HF}(x_i)= \hat{h}_0(x_i) + \hat{u}^{HF}(x_i),
\]
allows us to rewrite the ground state energy (adding and subtracting
$\hat{u}^{HF}(x_i)$)
\[
  E_0^{HF} =\langle \Phi_0 | \hat{H} | \Phi_0\rangle = \sum_{i\le F}^A
  \langle i | \hat{h}_0 +\hat{u}^{HF}| j\rangle+ \frac{1}{2}\sum_{i\le
    F}^A\sum_{j \le F}^A\left[\langle ij |\hat{v}|ij \rangle-\langle
    ij|\hat{v}|ji\rangle\right]-\sum_{i\le F}^A \langle i
  |\hat{u}^{HF}| i\rangle,
\]
as
\[
  E_0^{HF} = \sum_{i\le F}^A \varepsilon_i + \frac{1}{2}\sum_{i\le
    F}^A\sum_{j \le F}^A\left[\langle ij |\hat{v}|ij \rangle-\langle
    ij|\hat{v}|ji\rangle\right]-\sum_{i\le F}^A \langle i
  |\hat{u}^{HF}| i\rangle,
\]
which is nothing but
\[
  E_0^{HF} = \sum_{i\le F}^A \varepsilon_i - \frac{1}{2}\sum_{i\le
    F}^A\sum_{j \le F}^A\left[\langle ij |\hat{v}|ij \rangle-\langle
    ij|\hat{v}|ji\rangle\right].
\]

\subsection{Deriving the Hartree-Fock equations by varying expansion coefficients}
Another possibility is to expand the single-particle
functions in a known basis and vary the coefficients, that is, the new
single-particle wave function is written as a linear expansion in
terms of a fixed chosen orthogonal basis (for example harmonic
oscillator, Laguerre polynomials etc) \be \psi_a = \sum_{\lambda}
C_{a\lambda}\psi_{\lambda}.
\label{eq:newbasis}
\ee In this case we vary the coefficients $C_{a\lambda}$. If the basis
has infinitely many solutions, we need to truncate the above sum.  In
all our equations we assume a truncation has been made.

The single-particle wave functions $\psi_{\lambda}({\bf r})$, defined
by the quantum numbers $\lambda$ and ${\bf r}$ are defined as the
overlap
\[
   \psi_{\lambda}({\bf r}) = \langle {\bf r} | \lambda \rangle .
\]
We will omit the radial dependence of the wave functions and introduce
first the following shorthands for the Hartree and Fock integrals
\[
\langle \mu\nu|V|\mu\nu\rangle = \int
\psi_{\mu}^*(\mathbf{r}_i)\psi_{\nu}^*(\mathbf{r}_j)V(r_{ij})\psi_{\mu}(\mathbf{r}_i)\psi_{\nu}(\mathbf{r}_j)
d\mathbf{r}_i\mathbf{r}_j,
\]
and
\[
\langle \mu\nu|V|\nu\mu\rangle = \int
\psi_{\mu}^*(\mathbf{r}_i)\psi_{\nu}^*(\mathbf{r}_j)
V(r_{ij})\psi_{\nu}(\mathbf{r}_i)\psi_{\mu}(\mathbf{r}_i)
d\mathbf{r}_i\mathbf{r}_j.
\]
Since the interaction is invariant under the interchange of two
particles it means for example that we have
\[
\langle \mu\nu|V|\mu\nu\rangle = \langle \nu\mu|V|\nu\mu\rangle,
\]
or in the more general case
\[
\langle \mu\nu|V|\sigma\tau\rangle = \langle
\nu\mu|V|\tau\sigma\rangle.
\]
The direct and exchange matrix elements can be brought together if we
define the antisymmetrized matrix element
\[
\langle \mu\nu|V|\mu\nu\rangle_{AS}= \langle
\mu\nu|V|\mu\nu\rangle-\langle \mu\nu|V|\nu\mu\rangle,
\]
or for a general matrix element
\[
\langle \mu\nu|V|\sigma\tau\rangle_{AS}= \langle
\mu\nu|V|\sigma\tau\rangle-\langle \mu\nu|V|\tau\sigma\rangle.
\]
It has the symmetry property
\[
\langle \mu\nu|V|\sigma\tau\rangle_{AS}= -\langle
\mu\nu|V|\tau\sigma\rangle_{AS}=-\langle
\nu\mu|V|\sigma\tau\rangle_{AS}.
\]
The antisymmetric matrix element is also hermitian, implying
\[
\langle \mu\nu|V|\sigma\tau\rangle_{AS}= \langle
\sigma\tau|V|\mu\nu\rangle_{AS}.
\]
We have for the interaction part
\begin{equation}
  \int \Phi^*\hat{H_1}\Phi d\mathbf{\tau} =
  \frac{1}{2}\sum_{\mu=1}^A\sum_{\nu=1}^A \langle
  \mu\nu|V|\mu\nu\rangle_{AS}.
\label{H2Expectation2}
\end{equation}
Combining Eqs.~(\ref{H1Expectation1}) and (\ref{H2Expectation2}) we
obtain the energy functional
\begin{equation}
  E[\Phi] = \sum_{\mu=1}^N \langle \mu | h | \mu \rangle +
  \frac{1}{2}\sum_{{\mu}=1}^A\sum_{{\nu}=1}^A \langle
  \mu\nu|V|\mu\nu\rangle_{AS}.
\label{FunctionalEPhi}
\end{equation}
 
If we vary the above energy functional with respect to the basis functions $|\mu \rangle$, this corresponds to 
what was done in the previous case. We are however interested in defining a new basis defined in terms of
a chosen basis as defined in Eq.~(\ref{eq:newbasis}). We can then rewrite the energy functional as
\begin{equation}
  E[\Phi^{\mathrm{HF}}] 
  = \sum_{i=1}^A \langle i | \hat{h}_0 | i \rangle +
  \frac{1}{2}\sum_{ij=1}^A\langle ij|V|ij\rangle_{AS},
\label{FunctionalEPhi2}
\end{equation}
where $\Phi^{\mathrm{HF}}$ is the new Slater determinant defined by the new basis of Eq.~(\ref{eq:newbasis}). 
Using Eq.~(\ref{eq:newbasis}) we can rewrite Eq.~(\ref{FunctionalEPhi2}) as 
\begin{equation}
  E[\Phi^{\mathrm{HF}}] 
  = \sum_{i=1}^A \sum_{\alpha\beta} C^*_{i\alpha}C_{i\beta}\langle \alpha | h | \beta \rangle +
  \frac{1}{2}\sum_{ij=1}^A\sum_{{\alpha\beta\gamma\delta}} C^*_{i\alpha}C^*_{j\beta}C_{i\gamma}C_{j\delta}\langle \alpha\beta|V|\gamma\delta\rangle_{AS}.
\label{FunctionalEPhi3}
\end{equation}

We wish now to minimize the above functional. We introduce again a set of Lagrange multipliers, noting that
since $\langle i | j \rangle = \delta_{i,j}$ and $\langle \alpha | \beta \rangle = \delta_{\alpha,\beta}$, 
the coefficients $C_{i\gamma}$ obey the relation
\[
 \langle i | j \rangle=\delta_{i,j}=\sum_{\alpha\beta} C^*_{i\alpha}C_{i\beta}\langle \alpha | \beta \rangle=
\sum_{\alpha} C^*_{i\alpha}C_{i\alpha},
\]
which allows us to define a functional to be minimized that reads
\begin{equation}
 F[\Phi^{\mathrm{HF}}]= E[\Phi^{\mathrm{HF}}] - \sum_{i=1}^A\epsilon_i\sum_{\alpha} C^*_{i\alpha}C_{i\alpha}.
\end{equation}
Minimizing with respect to $C^*_{i\alpha}$, remembering that $C^*_{i\alpha}$ and $C_{i\alpha}$
are independent, we obtain
\be
\frac{d}{dC^*_{i\alpha}}\left[  E[\Phi^{\mathrm{HF}}] - \sum_{j}^{A}\epsilon_j\sum_{\alpha} C^*_{j\alpha}C_{j\alpha}\right]=0,
\ee
which yields for every single-particle state $i$ the following Hartree-Fock equations
\be
\sum_{\gamma} C_{i\gamma}\langle \alpha | \hat{h}_0 | \gamma \rangle+
\sum_{j=1}^A\sum_{\beta\gamma\delta} C^*_{j\beta}C_{j\delta}C_{i\gamma}\langle \alpha\beta|V|\gamma\delta\rangle_{AS}=\epsilon_kC_{i\alpha}.
\ee
We can rewrite this equation as 
\be \sum_{\beta}
  \left\{\langle \alpha | \hat{h}_0 | \beta \rangle+
  \sum_{j}^A\sum_{\gamma\delta} C^*_{j\gamma}C_{j\delta}\langle
  \alpha\gamma|V|\beta\delta\rangle_{AS}\right\}C_{i\beta}=\epsilon_i^{\mathrm{HF}}C_{i\alpha}.
  \ee 
Note that the sums over greek indices run over the number of
  basis set functions (in principle an infinite number).  
Defining 
\[
\hat{h}_{\alpha\beta}^{HF}=\langle \alpha | \hat{h}_0 | \beta \rangle+
\sum_{j=1}^A\sum_{\gamma\delta} C^*_{j\gamma}C_{j\delta}\langle \alpha\gamma|V|\beta\delta\rangle_{AS},
\]
we can rewrite the new equations as 
\be
\sum_{\beta}h_{\alpha\beta}^{HF}C_{i\beta}=\epsilon_i^{\mathrm{HF}}C_{i\alpha}.
\label{eq:newhf}
\ee
Note again that the sums over greek indices run over the number of basis set functions (in principle an infinite number).

\subsubsection{Koopman's theorem and interpretation of Hartree-Fock results}
We have defined 
\[
  E[\Phi^{\mathrm{HF}}(A)] 
  = \sum_{i=1}^A \langle i | \hat{h}_0 | i \rangle +
  \frac{1}{2}\sum_{ij=1}^A\langle ij|V|ij\rangle_{AS},
\]
where $\Phi^{\mathrm{HF}}(A)$ is the new Slater determinant defined by
the new basis of Eq.~(\ref{eq:newbasis}) for $A$ nucleons.  If we
assume that the single-particle wave functions in the new basis do not
change from a nucleus with $A$ nucleons to a nucleus with $A-1$
nucleons, we can then define the corresponding energy for the $A-1$
systems as
\[
  E[\Phi^{\mathrm{HF}}(A-1)] 
  = \sum_{i=1; i\ne k}^A \langle i | \hat{h}_0 | i \rangle +
  \frac{1}{2}\sum_{ij=1;i,j\ne k}^A\langle ij|V|ij\rangle_{AS},
\]
where we have removed a single-particle state $k\le F$, that is a state below the Fermi level.  
Calculating the difference 
\[
  E[\Phi^{\mathrm{HF}}(A)]-   E[\Phi^{\mathrm{HF}}(A-1)] 
  = \langle k | \hat{h}_0 | k \rangle +
  \frac{1}{2}\sum_{i=1;i\ne k}^A\langle ik|V|ik\rangle_{AS}  \frac{1}{2}\sum_{j=1;j\ne k}^A\langle kj|V|kj\rangle_{AS},
\]
which becomes 
\[
  E[\Phi^{\mathrm{HF}}(A)]-   E[\Phi^{\mathrm{HF}}(A-1)] 
  = \langle k | \hat{h}_0 | k \rangle +
  \frac{1}{2}\sum_{j=1}^A\langle kj|V|kj\rangle_{AS}
\]
which is just our definition of the Hartree-Fock single-particle energy
\[
  E[\Phi^{\mathrm{HF}}(A)]-   E[\Phi^{\mathrm{HF}}(A-1)] 
  = \epsilon_k^{\mathrm{HF}} 
\]
Similarly, we can now compute the difference (recall that the single-particle states $abcd > F$)
\[
  E[\Phi^{\mathrm{HF}}(A+1)]-   E[\Phi^{\mathrm{HF}}(A)]= \epsilon_a^{\mathrm{HF}}. 
\]
This equation and the one on the previous slide, are linked to
Koopman's theorem.  In atomic physics it is used to define the
electron ionization or affinity energies.  Koopman's theorem states
that the ionization energy of closed-shell systems is given by the
energy of the highest occupied single-particle state.  If we then
recall that the binding energy differences
\[
BE(A)-BE(A-1) \hspace{0.5cm} \mathrm{and} \hspace{0.5cm} BE(A+1)-BE(A), 
\]
define the separation energies, we see that the Hartree-Fock single-particle energies can be used to
define separation energies. We have thus our first link between nuclear forces (included in the potential energy term). 

We have thus the following interpretations (if the single-particle field do not change)
\[
BE(A)-BE(A-1)\approx  E[\Phi^{\mathrm{HF}}(A)]-   E[\Phi^{\mathrm{HF}}(A-1)] 
  = \epsilon_k^{\mathrm{HF}}, 
\]
and 
\[
BE(A+1)-BE(A)\approx  E[\Phi^{\mathrm{HF}}(A+1)]-   E[\Phi^{\mathrm{HF}}(A)] =  \epsilon_a^{\mathrm{HF}}. 
\]
If we use the $^{16}$O as our closed-shell nucleus, we could then interpret the separation energy
\[
BE(^{16}\mathrm{O})-BE(^{15}\mathrm{O})\approx \epsilon_{0p^{\nu}_{1/2}}^{\mathrm{HF}}, 
\]
and 
\[
BE(^{16}\mathrm{O})-BE(^{15}\mathrm{N})\approx \epsilon_{0p^{\pi}_{1/2}}^{\mathrm{HF}}.
\]

Similarly, we could interpret
\[
BE(^{17}\mathrm{O})-BE(^{16}\mathrm{O})\approx \epsilon_{0d^{\nu}_{5/2}}^{\mathrm{HF}}, 
\]
and 
\[
BE(^{17}\mathrm{F})-BE(^{16}\mathrm{N})\approx\epsilon_{0d^{\pi}_{5/2}}^{\mathrm{HF}}.
\]
We can continue like this for all $A\pm 1$ nuclei where $A$ is a good closed-shell (or subshell closure)
nucleus. Examples are $^{4}$He, $^{22}$O, $^{24}$O, $^{40}$Ca, $^{48}$Ca, $^{52}$Ca, $^{54}$Ca, $^{56}$Ni, 
$^{68}$Ni, $^{78}$Ni, $^{90}$Zr, $^{88}$Sr, $^{100}$Sn, $^{132}$Sn and $^{208}$Pb, to mention some possile cases.
{\bf make figures}

If we also recall the so-called energy gap for neutrons (or protons) defined as
\[
\Delta S_n= 2BE(N,Z)-BE(N-1,Z)-BE(N+1,Z),
\]
for neutrons and the corresponding gap for protons
\[
\Delta S_p= 2BE(N,Z)-BE(N,Z-1)-BE(N,Z+1),
\]
we can define the neutron and proton energy gaps for $^{16}$O as
\[
\Delta S_{\nu}=\epsilon_{0d^{\nu}_{5/2}}^{\mathrm{HF}}-\epsilon_{0p^{\nu}_{1/2}}^{\mathrm{HF}}, 
\]
and 
\[
\Delta S_{\pi}=\epsilon_{0d^{\pi}_{5/2}}^{\mathrm{HF}}-\epsilon_{0p^{\pi}_{1/2}}^{\mathrm{HF}}. 
\]

\section{Bulding a Hartree-Fock program}
\subsection{Algorithm for solving the Hartree-Fock equations}

Our Hartree-Fock matrix  is thus
\[
\hat{h}_{\alpha\beta}^{HF}=\langle \alpha | \hat{h}_0 | \beta \rangle+
\sum_{j=1}^A\sum_{\gamma\delta} C^*_{j\gamma}C_{j\delta}\langle \alpha\gamma|V|\beta\delta\rangle_{AS}.
\]
The Hartree-Fock are solved in an iterative way starting with a guess
for the coefficients $C_{j\gamma}=\delta_{j,\gamma}$ and solving the
equations by diagonalization till the new single-particle energies
$\epsilon_i^{\mathrm{HF}}$ do not change anymore by a prefixed
quantity.

Normally we assume that the single-particle basis $|\beta\rangle$ forms an eigenbasis for the operator
$\hat{h}_0$, meaning that the Hartree-Fock matrix becomes  
\[
\hat{h}_{\alpha\beta}^{HF}=\epsilon_{\alpha}\delta_{\alpha,\beta}+
\sum_{j=1}^A\sum_{\gamma\delta} C^*_{j\gamma}C_{j\delta}\langle \alpha\gamma|V|\beta\delta\rangle_{AS}.
\]
The Hartree-Fock eigenvalue problem defined in Eq.~{\ref{eq:newhf}),
\[
\sum_{\beta}\hat{h}_{\alpha\beta}^{HF}C_{i\beta}=\epsilon_i^{\mathrm{HF}}C_{i\alpha},
\]
can be written out in a more compact form as
\[
\hat{h}^{HF}\hat{C}=\epsilon^{\mathrm{HF}}\hat{C}. 
\]
The Hartree-Fock equations are, in their simplest form, solved in an iterative way, starting with a guess for the
coefficients $C_{i\alpha}$. We label the coefficients as $C_{i\alpha}^{(n)}$, where the subscript $n$ stands for iteration $n$.
To set up the algorithm we can proceed as follows:
\begin{enumerate}
\item We start with a guess $C_{i\alpha}^{(0)}=\delta_{i,\alpha}$. Alternatively, we could have used random starting values as long as the vectors are normalized. Another possibility is to give states below the Fermi level a larger weight.
\item The Hartree-Fock matrix simplifies then to
\[
\hat{h}_{\alpha\beta}^{HF}=\epsilon_{\alpha}\delta_{\alpha,\beta}+
\sum_{j=1}^A\sum_{\gamma\delta} C^*_{j\gamma}^{(0)}C_{j\delta}^{(0)}\langle \alpha\gamma|V|\beta\delta\rangle_{AS}.
\]
Solving the Hartree-Fock eigenvalue problem yields then new eigenvectors $C_{i\alpha}^{(1)}$ and eigenvalues
$\epsilon_i^{\mathrm{HF}}^{(1)}$. 
\item With the new eigenvalues we can set up a new Hartree-Fock potential 
\[
\sum_{j=1}^A\sum_{\gamma\delta} C^*_{j\gamma}^{(1)}C_{j\delta}^{(1)}\langle \alpha\gamma|V|\beta\delta\rangle_{AS}.
\]
The diagonalization with the new Hartree-Fock potential yields new eigenvectors and eigenvalues.
This process is continued till for example
\[
\frac{\sum_{p} |\epsilon_i^{\mathrm{HF}}^{(n)}-\epsilon_i^{\mathrm{HF}}^{(n-1)}|}{m}\le \lambda.  
\]
\end{enumerate}

\begin{algorithm}
\DontPrintSemicolon % Some LaTeX compilers require you to use \dontprintsemicolon instead 
\KwIn{A set $C = \{c_1, c_2, \ldots, c_r\}$ of denominations of coins, where $c_i > c_2 > \ldots > c_r$ and a positive number $n$}
\KwOut{A list of coins $d_1,d_2,\ldots,d_k$, such that $\sum_{i=1}^k d_i = n$ and $k$ is minimized}
$C \gets \emptyset$\;
\For{$i \gets 1$ \textbf{to} $r$}{
  \While{$n \geq c_i$} {
    $C \gets C \cup \{c_i\}$\;
    $n \gets n - c_i$\;
  }
}
\Return{$C$}\;
\caption{{\sc Change} Makes change using the smallest number of coins}
\label{algo:change}
\end{algorithm}

Algorithm~\ref{algo:duplicate} and Algorithm~\ref{algo:duplicate2} will
find the first duplicate element in a sequence of integers.

\begin{algorithm}
\DontPrintSemicolon % Some LaTeX compilers require you to use \dontprintsemicolon instead
\KwIn{A sequence of integers $\langle a_1, a_2, \ldots, a_n \rangle$}
\KwOut{The index of first location witht he same value as in a previous location in the sequence}
$location \gets 0$\;
$i \gets 2$\;
\While{$i \leq n$ \textbf{and} $location = 0$}{
  $j \gets 1$\;
  \While{$j < i$ \textbf{and} $location = 0$}{
    % The "u" before the "If" makes it so there is no "end" after the statement, so the else will then follow
    \uIf{$a_i = a_j$}{
      $location \gets i$\;
    }
    \Else{
      $j \gets j + 1$\;
    }
  }
  $i \gets i + 1$\;
}
\Return{location}\;
\caption{{\sc FindDuplicate}}
\label{algo:duplicate}
\end{algorithm}

